\documentclass[twoside]{amsart}
\usepackage{amssymb,latexsym}
\usepackage{xspace}
\usepackage{enumitem}
\newcommand{\Rationals}{\mathbb{Q}{}}
\newcommand{\Reals}{\mathbb{R}{}}
\newcommand{\Integers}{\mathbb{Z}{}}
\newcommand{\Solution}{\textsc{Solution}\xspace}
\newcommand{\brk}{\vspace{5pt}}
\newcommand{\noindsol}{\noindent\Solution}
\newcommand{\Problem}{\textsc{Problem}\xspace}
\newcommand{\Blank}{\mathrel{\phantom{=}}}
\newcommand{\ltrue}{\top}
\newcommand{\lfalse}{\bot}
\begin{document}
\title[Answers to Chapter 2 Exercises]{Answers to Chapter 2 Excercises
  - A Book of Abstract Algebra} 
\author{Michael Welch}
\date{\today}
\maketitle


\noindent\large\textbf{A. $\quad$ Examples of Operations} 

\brk \noindent Which of the following rules are operations on the
indicated set?  ($\Integers$ designates the set of the integers,
$\Rationals$ the rational numbers, and $\Reals$ the real numbers.)
For each rule which is not an operation, explain why it is not.

\brk \noindent \textbf{Example} $a * b = \displaystyle \frac{a +
  b}{ab}$, on the set $\Integers$.

\brk \noindsol This is not an operation on $\Integers$. There are
integers $a$ and $b$ such that $(a+b)/ab$ is not an integer. (For
example,
% 
\[
\frac{2+3}{2 \cdot 3} = \frac{5}{6}
\]
% 
is not an integer.) Thus, $\Integers$ is not closed under $*$.

\brk \begin{enumerate}[topsep=1ex,label=\protect{\textbf{\arabic*}},
  leftmargin=1em]

\item $a*b = \sqrt{|ab|}$, on the set $\Rationals$.

  \brk \noindsol This is not an operation on $\Rationals$. Here is a
  counter example: $2*1 = \sqrt{|2 \cdot 1|} = \sqrt{2}$. The result
  is in irrational number, and therefore $\Rationals$ is not closed
  under $*$.  \brk



\item $a*b = a \ln b$, on the set $\{x \in \Reals : x > 0\}$.

  \brk \noindsol This is not an operation because it is not closed. A
  counter example is obtained by choosing $a=1$ and $b=0.5$: $1 * 0.5
  = \ln 0.5 \approx -0.693 < 0$.  \brk



\item $a*b$ is a root of the equation $x^2 - a^2 b^2 = 0$, on the set
  $\Reals$.

  \brk \noindsol This is not an operation because it is not uniquely
  defined.  If we pick $a=2,b=2$ then we have $x^2 - 2^2 2^2 = 0$. The
  roots of that equation are $x = \pm 4$. This means that $a*b$ is not
  uniquely defined, therefore $*$ is not an operation.  \brk


\item Subtraction, on the set $\Integers$.

  \brk \noindsol This is an operation. It is defined for all pairs
  $\{(a,b) : a,b \in \Integers\}$, it is uniquely defined for each
  pair, and it is closed.  \brk


\item Subtraction, on the set $\{ n \in \Integers : n \ge 0\}$.

  \brk \noindsol This is not an operation because it is not closed:
  $4*10 = 4 - 10 = -4 < 0$. \brk

\item $a * b = |a-b|$, on the set $\{ n \in \Integers : n \ge 0\}$.

  \brk \noindsol This is an operation. It is defined for all pairs
  $\{(a,b) : a,b \in \Integers, a \ge 0, b \ge 0\}$. It is uniquely
  defined for each pair, and it is closed.

\end{enumerate}

\noindent\large\textbf{B. $\quad$ Properties of Operations} 

\brk \noindent Each of the following is an operation $*$ on
$\Reals$. Indicate whether or not

\begin{enumerate}[label=\protect{(\roman*)}]
\item \label{list.comm} it is commutative,
\item \label{list.assoc} it is associative,
\item \label{list.identity} $\Reals$ has an identity element with respect to
$*$,
\item \label{list.inverse} every $x \in \Reals$ has an inverse with respect to
$*$.
\end{enumerate}

\brk \noindent \textbf{Instructions} For \ref{list.comm}, compute $x * y$ and
$y * x$, and verify whether or not they are equal. For \ref{list.assoc},
compute $x * (y
* z)$ and $(x * y) * z$, and verify whether or not they are
equal. For \ref{list.identity}, first solve the equation $x * e = x$ for $e$;
if the equation caonnot be solved, there is no identity element. If it
\emph{can} be solved, it is still necessary to check that $e*x=x*e=x$ for any
$x\in \Reals$. If it checks, then $e$ is an identity element. For
\ref{list.inverse}, first note that if there is no identity element, there can
be no inverses. If there is an identity element $e$, first solve the equation
$x*x'=e$ for $x'$; if the equation cannot be solved, $x$ does not have an
inverse. If it \emph{can} be solved, check to make sure that $x*x'=x'*x=e$. If
this checks, $x'$ is the inverse of $x$. \brk

\noindent \textbf{Example} $x * y = x + y + 1$ \brk

\begin{tabular}{cccc}
  \emph{Associative} & \emph{Commutative} & \emph{Identity} &
  \emph{Inverses} \\

  Yes $\boxtimes \quad$ No $\square$ & Yes $\boxtimes \quad$ No $\square$ &
  Yes $\boxtimes \quad$ No $\square$ & Yes $\boxtimes \quad$ No $\square$ 

\end{tabular}

\brk
\begin{enumerate}[label=\protect{(\roman*)}]
\item $x * y = x + y + 1;\quad y * x = y + x + 1 = x + y + 1$. \\
  (\emph{Thus, $*$ is commutative}.)

\item $x * (y * z) = x * (y + z + 1) = x + (y + z + 1) + 1 = x + y + z
  + 2$.  $(x * y) * z = (x + y + 1) * z = (x + y + 1) + z + 1 = x + y
  + z +
  2$. \\
  (\emph{$*$ is associative.})

\item Solve $x * e = x$ for $e$: $x * e = x + e + 1 = x$; therefore $e
  = -1$. Check: $x * (-1) = x + (-1) + 1 = x$; $(-1) * x = (-1) +
  x + 1 = x$. Therefore, -1 is the identity element. \\
  (\emph{$*$ has an identity element.})

\item Solve $x * x' = e$ for $x'$: $x * x' = x + x' + 1 = -1$;
  therefore $x' = -2 - x$. Check: $x * (-2 - x) = x + (-2 - x) + 1 =
  -1$; $(-2 - x) * x = (-2 - x) + x + 1 = -1$. Therefore, $-x - 2$
  is the inverse of $x$. \\
  (\emph{Every element has an inverse.})

\end{enumerate} \brk

\begin{enumerate}[label=\protect{\textbf{\arabic*}}, leftmargin=1em]

\item $x*y = x + 2y + 4$

  \brk \noindsol The solution follows:

  \begin{enumerate}[label=\protect{({\roman*})}]
  \item The operation is not commutative
    \begin{align*}
        x * y & = x + 2y + 4 \\
        y * x & = y + 2x + 4 \\
              & = 2x + y + 4
    \end{align*}

    Now choose $x=0$ and $y=1$ and we see that
    \begin{align*}
       0 * 1 & = 0 + 2 + 4 \\
             & = 6         \\
       1 * 0 & = 1 + 0 + 4 \\
             & = 5
    \end{align*}
    (\emph{$*$ is not commutative.}) \brk

  \item The operation is not associative
    \begin{align*}
        x * (y * z) & = x * (y + 2z + 4)      \\
                    & = x + 2(y + 2z + 4) + 4 \\
                    & = x + 2y + 4z + 12      \\
        (x * y) * z
                    & = (x + 2y + 4) * z      \\
                    & = (x + 2y + 4) + 2z + 4 \\
                    & = x + 2y + 2z + 8       \\
                    & = (x + 2y + 4z + 12) - 2z - 4 \\
                    & = x * (y * z) - 2z - 4  
    \end{align*}

    Now let's check:
    \begin{align*}
    \text{If} \quad x * (y * z)   & = (x * y) * z         \\
    \text{then} \quad x * (y * z) & = x * (y * z) - 2z -4 \\
    \text{therefore} \quad 0      & = -2z - 4             \\
    \text{This requires} \quad z  & = -2
    \end{align*}

    (\emph{$*$ is not associative}) \brk

  \item Check to see if their is an identity element:
    \begin{align*}
         x * e & = x           \\ 
         x * e & = x + 2e + 4  \\
	     x & = x + 2e + 4  \\
	     0 & =     2e + 4  \\
	    2e & =         -4  \\
	     e & =         -2  
    \end{align*}

    Now check to see if $x*e=e*x=x$:
    \begin{align*}
         x * -2 & = x + (2)(-2) + 4 \\
	        & = x - 4 + 4       \\
		& = x               \\
	 -2 * x & = -2 + 2x + 4     \\
	        & = 2x + 2          
    \end{align*}

    We can see $x*e \ne e*x$ therefore there is no identity
    element. \brk 

  \item No need to check for an inverse. There is no identity. \brk 
  \end{enumerate}

\item  $x * y = x + 2y - xy$

  \noindent \Solution The solution follows:

  \begin{enumerate}[label=\protect{(\roman*)}]
      
  \item The operation is not commutative:
    \begin{align*}
         x * y & = x + 2y - xy \\
	 y * x & = y + 2x - yx \\
	       & = 2x + y - xy \\
               & = (x + 2y - xy) + x - y \\
               & = (x * y) + x - y
    \end{align*}

    So if $x * y = y * x$ then $x * y = x * y + x - y$ which means $0 = x
    - y$ which requires that $x = y$. So in general $*$ \emph{is not
      commutative}. \brk

  \item The operation is not associative
    \begin{align*}
         x * (y * z) & = x * (y + 2z - yz)                          \\
	             & = x + 2(y + 2z - yz) - x(y + 2z - yz) \notag \\
		     & = x + 2y + 4z - 2yz - xy - 2xz + xyz  \notag \\
		     & = x + 2y + 4z - xy - 2xz - 2yz + xyz  \notag \\
	 (x * y) * z & = (x + 2y - xy) * z                          \\
	             & = (x + 2y - xy) + 2z - (x + 2y - xy)z \notag \\
		     & = (x + 2y - xy) + 2z - xz - 2yz - xyz \notag \\
		     & = x + 2y + 2z - xy - xz - 2yz - xyz   \notag
    \end{align*}


  \item From the following we can see that $e=0$.
    \begin{align*}
         x * e & = x                  \\
	 x * e & = x + 2e - xe        \\
	     x & = x + 2e - xe \notag \\
	     0 & =     2e - xe \notag \\
	e(2-x) & = 0           \notag \\
	     e & = 0 \quad\text{(when $x \ne 2$)} \notag 
    \end{align*}
    (We need to check what happens when $x = 2$: we have
    $2*y=2+2y-2y=2$. So when $x=2$, $x*y$ is always equal to 2 no
    matter what $y$ is.) 

    Now we must check that $x*e=e*x=x$:
    \begin{align*}
         e * x & = e + 2x - ex      \\
	       & = 0 + 2x - 0       \\
	       & = 2x               \\
	       & \ne x              
    \end{align*}
    Therefore, there is no identity element. \brk

  \item Since there is no identity element, there are no inverses. \brk

  \end{enumerate}

\item $x * y = | x + y |$

  \noindent \Solution The solution follows:

  \begin{enumerate}[label=(\roman*)]
  \item From the following we can see that the operation is
    commutative:
    \begin{align*}
         x * y & = | x + y |         \\
	 y * x & = | y + x |         \\
	       & = | x + y | \notag  
    \end{align*}

  \item It is not associative. Here is a counter example using $x = 7,
    y = -13, z = 1$

    \begin{align*}
         x * (y * z) & = 7 * | -13 + 1 | \\
	             & = | 7 + 12 | \notag \\
		     & = 19         \notag \\
	 (x * y) * z & = | 7 + -13 | * 1   \\
	             & = | -6 | * 1 \notag \\
		     & = | 6 + 1 |  \notag \\
		     & = 7          \notag \\
		     & \ne 19       \notag
    \end{align*}

  \item The value of $0$ works as an identity element for non-negative
    numbers. However, there is no possible identity element for
    negative numbers. This is because the result of the absolute value
    function always returns non-negative values. \brk

  \item Since there is no identity element, there are no inverses.\brk

  \end{enumerate}


\item $x * y = | x - y |$

  \noindent \Solution The solution is as follows:
  \begin{enumerate}[label=(\roman*)]

  \item The operation is commutative:
    \begin{align*}
         x * y & = | x - y |           \\
	       & =
	       \begin{cases}
	          x - y,     & \text{if $x \ge y$;} \\
		  y - x,     & \text{otherwise.}
	       \end{cases} \notag \\
	 y * x & = | y - x |           \\
	       & =
	       \begin{cases}
	          x - y,     & \text{if $x \ge y$;} \\
		  y - x,     & \text{otherwise.}
	       \end{cases} \notag 
    \end{align*}

  \item The operation is not associative. A counter example is
    provided by choosing $x = 1, y = -2, z = 3$.
    \begin{align*}
         x * (y * z) & = 1 * |(-2) - 3|    \\
	             & = 1 * 5       \notag \\
		     & = | 1 - 5 |   \notag \\
		     & = 4           \notag \\
	 (x * y) * z & = | 1 - (-2) | * 3  \\
	             & = 3 * 3       \notag \\
		     & = | 3 - 3 |   \notag \\
		     & = 0           \notag \\
		     & \ne 4         \notag
    \end{align*}

  \item Again, there can be no identity element, since the result of
    the operation is always non-negative. Therefore, there can be no
    number $x * e = x$ for $x < 0$. \brk

  \item Again, since there is no identity element, there can be no
    inverses.  However, $-x$ acts as an inverse for any number
    assuming that $0$ actually was an identity element. \brk
  \end{enumerate}

\item $x*y = xy + 1$

  \noindent \Solution The solution is as follows:

  \begin{enumerate}[label=(\roman*)]

  \item The operation is commutative as seen below.
    \begin{align*}
         x * y & = xy + 1  \\
	 y * x & = yx + 1 \\
	       & = xy + 1 \quad\text{(Multiplication is commutative.)}
    \end{align*}

  \item The operation is not associative as seen below.
    \begin{align*}
         x * (y * z) & = x * (yz + 1) \\
	             & = x(yz + 1) + 1 \\
		     & = x + xyz + 1 \\
	 (x * y) * z & = (xy + 1) * z \\
	             & = (xy + 1)z + 1 \\
		     & = z + xyz + 1 \\
		     & \ne x + xyz + 1
    \end{align*}

  \item We can see below that their is no constant value defined for
    the identity element. Our equations find a formula for $e$ which
    depens on $x$. This is not a constant value.
    \begin{align*}
         x * e & = x \\
	 x * e & = xe + 1 \\
	     x & = xe + 1 \\
	 x - 1 & = xe     \\
	 xe    & = x - 1  \\
	  e    & = \frac{x - 1}{x} \\
    \end{align*}

  \item There is no identity element, so there are no inverses.\brk
  \end{enumerate}

\item $x * y = \max\{x,y\} =$ the larger of the two numbers $x$
  and $y$.

  \noindent \Solution The solution is as follows.

  \begin{enumerate}
      
  \item The operation is commutative. 
%
    \begin{align*}
         x * y & = \max\{x,y\}    \\
	       & =
	       \begin{cases}
	          x,       & \text{if $x \ge y$;} \\
		  y,       & \text{otherwise.}
	       \end{cases} \\
	 y * x & = \max\{y,x\}   \\
	       & =
	       \begin{cases}
                 x,       & \text{if $x \ge y$;} \\
                 y, & \text{otherwise.}
	       \end{cases}
    \end{align*}

  \item The operation is associative. 
    \begin{align*}
         x * (y * z) & = x * \max\{y,z\} \\
	             &= \max\{x, \max\{y,z\}\} \\
		     &=
		     \begin{cases}
		        \max\{x, y\}, \text{if $y \ge z$;} \\
			\max\{x, z\}, \text{otherwise.}
		     \end{cases} \\
		     & =
		     \begin{cases}
		        x,  & \text{if $x \ge y \land y \ge z$;} \\
			y,  & \text{if $x < y \land y \ge z$;}   \\
			z,  & \text{otherwise.}
		     \end{cases} \\
                     & =
		     \begin{cases}
		        x,  & \text{if $x \ge y \land x \ge z$;} \\
			y,  & \text{if $x < y \land y \ge z$;}   \\
			z,  & \text{otherwise.}
		     \end{cases} \\
	 (x * y) * z & = \max\{x,y\} * z \\
	             & = \max\{\max\{x,y\}, z\} \\
		     & =
		     \begin{cases}
		        \max\{x,z\}, \text{if $x \ge y$;} \\
			\max\{y,z\}, \text{otherwise.}
		     \end{cases} \\
		     & =
		     \begin{cases}
		        x, & \text{if $x \ge y \land x \ge z$;} \\
			y, & \text{if $x < y \land y \ge z$;} \\
			z, & \text{otherwise.}
		     \end{cases}
  \end{align*}

\item There is no identity element. We will prove this by
  contradiction.  Assume that there is some identity element $e$. Then
  by definition $x * e = x$ forall $x$. Let us choose a value $m = e -
  1$. Then we have $m * e = (e-1) * e = \max(e-1,e) = e \ne
  m$. Therefore, e is not an identity element. \qed. \brk

\item Since there is no identity element, there are no inverses.\brk
      
\end{enumerate}

\item $\displaystyle x * y = \frac{xy}{x + y + 1}$ on the set
  of positive real numbers. \brk

  \noindent \Solution The solution is as follows
  \begin{enumerate}
  \item The operation is commutative.
    \begin{align*}
         x * y & = \frac{xy}{x + y + 1} \\
	 y * x & = \frac{yx}{y + x + 1} \\
	       & = \frac{xy}{x + y + 1} && \text{(+,$\cdot$ are commutative)}
    \end{align*}

  \item The operation is not associative. This can be demonstrated
    with the values $x = 2, y = 3, z = 4$.
    \begin{align*}
         2 * (3 * 4) & = 2 * \frac{3 \cdot 4}{3 + 4 + 1}   \\
	             & = 2 * (3/2) \\
		     & = \frac{2 \cdot (3/2)}{2 + (3/2) + 1} \\
		     & = \frac{6/2}{9/2} \\
		     & = \frac{6}{9} \\
		     & = \frac{2}{3} \\
	 (2 * 3) * 4 & = \frac{2 \cdot 3}{2 + 3 + 1} * 4 \\
	             & = \frac{6}{7} * 4 \\
		     & = \frac{(6/7) \cdot 4}{(6/7) + 4 + 1} \\
		     & = \frac{24/7}{41/7} \\
		     & = \frac{24}{41} 
    \end{align*}

  \item There is no identity element. As a matter of fact, the
    equation $x * e = x$ can only be solved when $x = 0$ or $x = -1$,
    as shown below.
    \begin{align*}
         \frac{ex}{e + x + 1} & = x \\
	 ex                   & = x^2 + ex + x \\
	 x^2 + ex + x         & = ex \\
	 x^2 + x              & = 0 \\
	 x(x+1)               & = 0 \\
	 x                    & = 0, -1
    \end{align*}

  \item Since there is no identity, there can be no inverses.
  \end{enumerate}
\end{enumerate}

\noindent \large\textbf{C. $\quad$ Operations on a Two-Element Set}

\noindent Let $A$ be the two-element set $A = \{a,b\}$.
\begin{enumerate}[label=\textbf{\arabic*}, leftmargin=1em]
   \item Write the tables of all 16 operations on $A$. (Use the format
   explained on page 20.) Label these operations $O_1$ to $O_{16}$. 

   \noindent \Solution The tables are shown in table~\ref{tab:boolean_ops} on 
   page~\pageref{tab:boolean_ops}.

   \begin{table}
   \caption{The sixteen different operations for a two element set.}
   \label{tab:boolean_ops}
   \begin{tabular}{cccc}
      & & & \\
      \begin{tabular}{c|c}    
         \multicolumn{2}{c}{$O_{16}$} \\ 
         $(x,y)$ & $x * y$ \\ \hline
	 $(a,a)$ & $a$     \\ 
	 $(a,b)$ & $a$     \\ 
	 $(b,a)$ & $a$     \\ 
	 $(b,b)$ & $a$     
      \end{tabular} & 
      \begin{tabular}{c|c}   
         \multicolumn{2}{c}{$O_{1}$} \\ 
         $(x,y)$ & $x * y$ \\ \hline
	 $(a,a)$ & $a$     \\ 
	 $(a,b)$ & $a$     \\ 
	 $(b,a)$ & $a$     \\ 
	 $(b,b)$ & $b$     
      \end{tabular} & 
      \begin{tabular}{c|c}  
         \multicolumn{2}{c}{$O_{2}$} \\ 
         $(x,y)$ & $x * y$ \\ \hline
	 $(a,a)$ & $a$     \\ 
	 $(a,b)$ & $a$     \\ 
	 $(b,a)$ & $b$     \\ 
	 $(b,b)$ & $a$     
      \end{tabular} & 
      \begin{tabular}{c|c} 
         \multicolumn{2}{c}{$O_{3}$} \\ 
         $(x,y)$ & $x * y$ \\ \hline
	 $(a,a)$ & $a$     \\ 
	 $(a,b)$ & $a$     \\ 
	 $(b,a)$ & $b$     \\ 
	 $(b,b)$ & $b$     
      \end{tabular}  \\

      & & & \\
      \begin{tabular}{c|c}
         \multicolumn{2}{c}{$O_{4}$} \\
         $(x,y)$ & $x * y$ \\ \hline
	 $(a,a)$ & $a$     \\ 
	 $(a,b)$ & $b$     \\ 
	 $(b,a)$ & $a$     \\ 
	 $(b,b)$ & $a$     
      \end{tabular} & 
      \begin{tabular}{c|c}
         \multicolumn{2}{c}{$O_{5}$} \\
         $(x,y)$ & $x * y$ \\ \hline
	 $(a,a)$ & $a$     \\ 
	 $(a,b)$ & $b$     \\ 
	 $(b,a)$ & $a$     \\ 
	 $(b,b)$ & $b$     
      \end{tabular} & 
      \begin{tabular}{c|c}
         \multicolumn{2}{c}{$O_{6}$} \\ 
         $(x,y)$ & $x * y$ \\ \hline
	 $(a,a)$ & $a$     \\ 
	 $(a,b)$ & $b$     \\ 
	 $(b,a)$ & $b$     \\ 
	 $(b,b)$ & $a$     
      \end{tabular} & 
      \begin{tabular}{c|c}
         \multicolumn{2}{c}{$O_{7}$} \\ 
         $(x,y)$ & $x * y$ \\ \hline
	 $(a,a)$ & $a$     \\ 
	 $(a,b)$ & $b$     \\ 
	 $(b,a)$ & $b$     \\ 
	 $(b,b)$ & $b$     
      \end{tabular}  \\

      & & & \\
      \begin{tabular}{c|c}
         \multicolumn{2}{c}{$O_{8}$} \\ 
         $(x,y)$ & $x * y$ \\ \hline
	 $(a,a)$ & $b$     \\ 
	 $(a,b)$ & $a$     \\ 
	 $(b,a)$ & $a$     \\ 
	 $(b,b)$ & $a$     
      \end{tabular} & 
      \begin{tabular}{c|c}
         \multicolumn{2}{c}{$O_{9}$} \\ 
         $(x,y)$ & $x * y$ \\ \hline
	 $(a,a)$ & $b$     \\ 
	 $(a,b)$ & $a$     \\ 
	 $(b,a)$ & $a$     \\ 
	 $(b,b)$ & $b$     
      \end{tabular} & 
      \begin{tabular}{c|c}
         \multicolumn{2}{c}{$O_{10}$} \\ 
         $(x,y)$ & $x * y$ \\ \hline
	 $(a,a)$ & $b$     \\ 
	 $(a,b)$ & $a$     \\ 
	 $(b,a)$ & $b$     \\ 
	 $(b,b)$ & $a$     
      \end{tabular} & 
      \begin{tabular}{c|c}
         \multicolumn{2}{c}{$O_{11}$} \\ 
         $(x,y)$ & $x * y$ \\ \hline
	 $(a,a)$ & $b$     \\ 
	 $(a,b)$ & $a$     \\ 
	 $(b,a)$ & $b$     \\ 
	 $(b,b)$ & $b$     
      \end{tabular}  \\

      & & & \\
      \begin{tabular}{c|c}
         \multicolumn{2}{c}{$O_{12}$} \\ 
         $(x,y)$ & $x * y$ \\ \hline
	 $(a,a)$ & $b$     \\ 
	 $(a,b)$ & $b$     \\ 
	 $(b,a)$ & $a$     \\ 
	 $(b,b)$ & $a$     
      \end{tabular} & 
      \begin{tabular}{c|c}
         \multicolumn{2}{c}{$O_{13}$} \\ 
         $(x,y)$ & $x * y$ \\ \hline
	 $(a,a)$ & $b$     \\ 
	 $(a,b)$ & $b$     \\ 
	 $(b,a)$ & $a$     \\ 
	 $(b,b)$ & $b$     
      \end{tabular} & 
      \begin{tabular}{c|c}
         \multicolumn{2}{c}{$O_{14}$} \\ 
         $(x,y)$ & $x * y$ \\ \hline
	 $(a,a)$ & $b$     \\ 
	 $(a,b)$ & $b$     \\ 
	 $(b,a)$ & $b$     \\ 
	 $(b,b)$ & $a$     
      \end{tabular} & 
      \begin{tabular}{c|c}
         \multicolumn{2}{c}{$O_{15}$} \\ 
         $(x,y)$ & $x * y$ \\ \hline
	 $(a,a)$ & $b$     \\ 
	 $(a,b)$ & $b$     \\ 
	 $(b,a)$ & $b$     \\ 
	 $(b,b)$ & $b$     
      \end{tabular}  \\
      & & & \\
   \end{tabular}
   \end{table}

   \item Identify which of the operations $O_1$ to $O_{16}$ are
   commutative.

   \noindent \Solution This can be solved very easily by looking at the
   second and third entries in each table to see if $a*b=b*a$. The commutative
   entries are $O_1, O_6, O_7, O_8, O_9, O_{14}, O_{15}, O_{16}$.
   commutative. \brk

   \item Identify which operations, among $O_1$ to $O_{16}$, are 
   associative.

   \noindent \Solution 
   In general there are eight cases to check. These cases are shown in
   table~\ref{tab:eight_cases} on page~\pageref{tab:eight_cases}.
%
   \begin{table}
   \caption{The eight cases of associativity.}
   \label{tab:eight_cases}
   \begin{align*}
      a * (a * a) & = (a * a) * a && 1\\
      a * (a * b) & = (a * a) * b && 2\\
      a * (b * a) & = (a * b) * a && 3\\
      a * (b * b) & = (a * b) * b && 4\\
      b * (a * a) & = (b * a) * a && 5\\
      b * (a * b) & = (b * a) * b && 6\\
      b * (b * a) & = (b * b) * a && 7\\
      b * (b * b) & = (b * b) * b && 8
   \end{align*}
   \end{table}

   Let's consider how many cases there are when $*$ is commutative. 
   It will be shown that only two cases need to be checked: 2 and 4. All
   the others are trivially true by commutativity, or else they are true
   if 2 or 4 is true. See table~\ref{tab:com_check} on 
   page~\pageref{tab:com_check} to see the list of proofs.
%
   \begin{table}
   \caption{Equations proving only 2 cases must be checked when operation is commutative}
   \label{tab:com_check}
   \begin{align*}
      a * (a * a) & = (a * a) * a    && \text{(Commutativity)} \\
      \\
      a * (a * b) & = (a * a) * b    && \text{(Must be checked)} \\
      \\
      a * (b * a) & = a * (a * b)    && \text{(Commutativity)} \\
                  & = (a * b) * a    && \text{(Commutativity)} \\
      \\
      a * (b * b) & = (a * b) * b    && \text{(Must be checked)} \\
      \\
      b * (a * a) & = (a * a) * b    && \text{(Commutativity)} \\
                  & = a * (a * b)    && \text{(By case 2)} \\
		  & = a * (b * a)    && \text{(Commutativity)} \\
		  & = (b * a) * a    && \text{(Commutativity)} \\
      \\
      b * (a * b) & = b * (b * a)    && \text{(Commutativity)} \\
                  & = (b * a) * b    && \text{(Commutativity)} \\
      \\
      b * (b * a) & = b * (a * b)    && \text{(Commutativity)} \\
                  & = (a * b) * b    && \text{(Commutativity)} \\
		  & = a * (b * b)    && \text{(By case 4)}     \\
		  & = (b * b) * a    && \text{(Commutativity)} \\
      \\
      b * (b * b) & = (b * b) * b    && \text{(Commutativity)}
   \end{align*}
   \end{table}

   Now we need to start checking all the cases.

   $\mathbf{O_1}$\textbf{:} This operation is commutative. Cases 2 and 4 
   are true, so this operation is associative. See figure~\ref{fig:o1}
   \begin{figure}
   \caption{The proofs for $O_1$.}
   \label{fig:o1}
   \begin{align*}
      a * (a * b) & = a * a  && \text{(Left hand side of case 2.)}\\
                  & = a      \\
      (a * a) * b & = a * b  && \text{(Right hand side of case 2.)}\\
                  & = a      && \text{(Case 2 is true.)} \\
      a * (b * b) & = a * b && \text{(Left hand side of case 4.)}\\
                  & = a \\
      (a * b) * b & = a * b && \text{(Right hand side of case 4.)}\\
                  & = a     && \text{(Case 4 is true.)}
   \end{align*}
   \end{figure}

   $\mathbf{O_2}$\textbf{:} This operation is not commutative. Therefore there are
   eight cases to check. Rather than check them all, we'll use boolean
   algebra. We assume $a$ is the value \verb=false= and $b$ is the value
   \verb=true=. The operation is equivalent to the boolean equation $x\land
   \lnot y$. We calculate the appropriate equations for $x * (y * z)$ and
   $(x * y) * z$ in figure~\ref{fig:o2_bool_eq}.

   \begin{figure}
   \caption{The 3 variable boolean equations for $O_2$.}
   \label{fig:o2_bool_eq}
   \begin{align*}
      x * (y * z) & = x * (y \land \lnot z) \\
		  & = x \land \lnot(y \land \lnot z)\\
		  & = x \land (\lnot y \lor z) \\
      (x * y) * z & = (x \land \lnot y) * z \\
                  & = (x \land \lnot y) \land \lnot z \\
		  & = x \land \lnot y \land \lnot z 
   \end{align*}
   \end{figure}

   Now we need to find a difference. So we'll calculate a triple of
   values where the first formula is true, but the second formula is not.
   We'll do this with an equation of the form $f \land \lnot s$ where $f$
   is the first equation and $s$ is the second equation.
%
   \begin{align*}
      f \land \lnot s & = x \land (\lnot y \lor z) \land \lnot 
                       (x \land \lnot y \land \lnot z))\\
         & = (x \land \lnot y \lor x \land z) \land 
	     (\lnot x \lor y \lor z) \\
	 & = (x \land y \land z) \lor (x \land \lnot y \land z) 
	      \lor (x \land z) \\
	 & = (x \land z) \land (y \lor \lnot y \lor \top) \\
	 & = x \land z
   \end{align*}

   So the two formulas differ when $x = b, z = b$. Let's check.
%
   \begin{align*}
       b * (a * b) & = b * a \\
                   & = b \\
       (b * a) * b & = b * b \\
                   & = a     \\
		   & \ne b
   \end{align*}

   $\mathbf{O_3}$\textbf{:} This operation is equivalent to the
   equation $x * y = x$. This can be seen by observing the table for
   $O_3$. With this information we can easily check to see if the
   operation is associative.
%
   \begin{align*}
      x * (y * z) & = x * y \\
                  & = x     \\
      (x * y) * z & = x * z \\
                  & = x
   \end{align*}

   Both equations evaluate to $x$ so we can see that $O_3$ is
   associative. 

   $\mathbf{O_4}$\textbf{:} This operation is not commutative. It is 
   equivalent to the equatioin $x * y = \lnot x \land y$.
%
   \begin{align*}
      x * (y * z) & = x * (\lnot y \land z)    \\
                  & = \lnot x \land \lnot y \land z \\
      (x * y) * z & = (\lnot x \land y) * z \\
                  & = \lnot (\lnot x \land y) \land z
   \end{align*}

   The first formula is more restrictive. Let's find a set of values
   where the second formula is true but the first is not: $s \land \lnot f$.
%
   \begin{align*}
      s \land \lnot f & = (\lnot (\lnot x \land y) \land z) \land \lnot
                 (\lnot x \land \lnot y \land z ) \\
         & = (x \lor \lnot y) \land z \land (x \lor y \lor \lnot z) \\
	 & = (x \lor \lnot y) \land ((x \land z) \lor (y \land z) \lor \bot)\\
	 & = (x \land z) \lor (x \land y \land z) \lor 
	     (\lnot y \land x \land z) \lor (\lnot y \land y \land z)  \\
	 & = (x \land z) \land (y \lor \lnot y) \\
	 & = x \land z
   \end{align*}

   When $x = b, z = b$ we find a difference, therefore $O_4$ is not associative.
%
   \begin{align*}
      b * (b * b) & = b * a \\
                  & = a \\
      (b * b) * b & = a * b \\
                  & = b \\
		  & \ne a
   \end{align*}

   $\mathbf{O_5}$\textbf{:} This is not commutative either. It is equivalent
   to $x * y = y$. It is associative. We can see this since both formulas
   evaluate to the same thing.
%   
   \begin{align*}
      x * (y * z) & = x * z \\
                  & = z \\
      (x * y) * z & = y * z \\
                  & = z 
   \end{align*}
   
   $\mathbf{O_6}$\textbf{:} This operation is commutative. We just have to
   check cases 2 and 4 from above. Below we see both cases are true, therefore
   this operation is associative.
%
   \begin{align*}
      a * (a * b) & = a * b && \text{(Left hand side of case 2.)}\\
                  & = b \\
      (a * a) * b & = a * b && \text{(Right hand side of case 2.)}\\
                  & = b     && \text{(Case 2 is true.)} \\
      a * (b * b) & = a * a && \text{(Left hand side of case 4.)}\\
                  & = a \\
      (a * b) * b & = b * b && \text{(Right hand side of case 4.)} \\
                  & = a     && \text{(Case 4 is true.)}
   \end{align*}

   $\mathbf{O_7}$\textbf{:} This operation is commutative. Cases 2 and
   4 are true, therefore this operation is associative.
%
   \begin{align*}
      a * (a * b) & = a * b && \text{(Left hand side of case 2.)}\\
                  & = b \\
      (a * a) * b & = a * b && \text{(Right hand side of case 2.)}\\
                  & = b     && \text{(Case 2 is true.)}\\
      a * (b * b) & = a * b && \text{(Left hand side of case 4.)} \\
                  & = b  \\
      (a * b) * b & = b * b && \text{(Right hand side of case 4.)} \\
                  & = b     && \text{(Case 4 is true.)}
   \end{align*}

   $\mathbf{O_8}$\textbf{:} This operation is commutative. Case 2 is 
   false, therefore this operation is not associative.
   %
   \begin{align*}
      a * (a * b) & = a * a && \text{(Left hand side of case 2.)} \\
                  & = b     \\
      (a * a) * b & = b * b && \text{(Right hand side of case 2.)} \\
                  & = a     && \text{(Case 2 is false.)}
   \end{align*}

   $\mathbf{O_9}$\textbf{:} This operation is commutative. Cases 2 and
   4 are true, therefore this operation is associative.
   %
   \begin{align*}
      a * (a * b) & = a * a  \\
                  & = b \\
      (a * a) * b & = b * b \\
                  & = b && \text{(Case 2 is true.)} \\
      a * (b * b) & = a * b \\
                  & = a \\
      (a * b) * b & = a * b \\
                  & = a && \text{(Case 4 is true.)} 
   \end{align*}

   $\mathbf{O_{10}}$\textbf{:} This operation is not commutative. This
   operation corresponds to the boolean equation $x * y = \lnot y$. See
   figure~\ref{fig:o10_assoc} for proof that this operation is not
   associative.

   \begin{figure}
      \caption{Proof that $O_{10}$ is not associative.}
      \label{fig:o10_assoc}
      \begin{align*}
         x * y & = \lnot y  && \text{(Boolean equivalent equation for $O_{10}$.)} \\
         x * (y * z) & = x * \lnot z \\
	             & = z \\
         (x * y) * z & = \lnot y * z \\
	             & = \lnot z && \text{(Doesn't equal $z$.)}.
      \end{align*}
   \end{figure}


   $\mathbf{O_{11}}$\textbf{:} This operation is not commutative. 
   The equivalent binary equation is $x * y = x \lor \lnot x \land \lnot y$. See
   the derivation of the boolean equivalent equations for $x * (y * z)$ and $(x
   * y) * z$ in figure~\ref{fig:o11_assoc} to see that they equal $x \lor \lnot
   * x \land \lnot y \land z$ and $x \lor \lnot x \land \lnot y$ respectively.

   \begin{figure}
      \caption{Derivation of the boolean equations $x * (y * z)$ and
      $(x * y) * z$ for $O_{11}$.}
      \label{fig:o11_assoc}
      \begin{align*}
         x * y & = x \lor (\lnot z \land \lnot y) \\
	 x * (y * z) & = x * (y \lor (\lnot y \land \lnot z)) \\
	             & = x \lor (\lnot x \land \lnot (y \lor (\lnot y 
		         \land \lnot z))) \\
	             & = x \lor (\lnot x \land (\lnot y \land \lnot (\lnot y
		         \land \lnot z)))  \\
	             & = x \lor (\lnot x \land (\lnot y \land (y \lor z))) \\
		     & = x \lor (\lnot x \land \lnot y \land y) 
		         \lor (\lnot x \land \lnot y \land z) \\
		     & = x \lor (\lnot x \land \lnot y \land z) \\
         (x * y) * z & = (x \lor (\lnot x \land \lnot y) * z \\
	             & = (x \lor (\lnot x \land \lnot y) \lor 
		         (z \land (x \lor (\lnot x \land \lnot y))) \\
		     & = x \lor (\lnot x \land \lnot y) \lor (x \land z)
		         \lor (\lnot x \land \lnot y \land z) \\
		     & = x \lor (\lnot x \land \lnot y)
      \end{align*}
   \end{figure}


   From the equations in figure~\ref{fig:o11_assoc} you can see that
   one of the equations is true for ${x = a, y = a, z = a}$ and the other
   one is false for the same set of values. So this is our counter example
   that $O_{11}$ is not associative. See figure~\ref{fig:o11_assoc2}
   for the derivation.

   \begin{figure}
      \caption{Derivation of ${x = a, y = a, z = a}$ to show $O_{11}$ 
      is not associative. One evaluates to $a$ and the other $b$.}
      \label{fig:o11_assoc2} 
      \begin{align*}
         a * (a * a) & = a * b \\
                     & = a \\
         (a * a) * a & = b * a \\
	             & = b        
      \end{align*}
   \end{figure}

   $\mathbf{O_{12}}$\textbf{:} This operation is not commutative. Its boolean
   equivalent equation is $x * y = \lnot x$. This operation is not associative
   The two equations evaluate to different values. See figure~\ref{fig:o12_assoc}.

   \begin{figure}
      \caption{Evaluation of binary equivalent equations for $O_{12}$ to show
      that it is not associative.}
      \label{fig:o12_assoc}
      \begin{align*}
         x * y & = \lnot x  \\
         x * (y * z) & = x * \lnot z \\
	             & = \lnot x \\
         (x * y) * z & = \lnot x * z \\
	             & = x
      \end{align*}
   \end{figure}

   $\mathbf{O_{13}}$\textbf{:} This operation is not commutative. It's boolean
   equivalent equation is $x * y = \lnot x \lor (x \land y)$. Using calculations
   similar to the other problems we find a set of values that shows
   this operation is not associative. Chose $x = a, y = a, z = a$. 
   See figure~\ref{fig:o13_assoc}.

   \begin{figure}
   \caption{Counter-example showing $O_{13}$ is not associative.}
   \label{fig:o13_assoc}
   \begin{align*}
      a * (a * a) & = a * b \\
                  & = b \\
      (a * a) * a & = b * a \\
                  & = a
   \end{align*}
   \end{figure}


   $\mathbf{O_{14}}$\textbf{:} This operation is commutative. Check cases
   2 and 4. See figure~\ref{fig:o14_assoc}. Case 2 is false. So this
   operation is not associative.

   \begin{figure}
      \caption{Case 2 for $O_{14}$. This operation is not associative.}
      \label{fig:o14_assoc}
      \begin{align*}
         a * (a * a) & = a * b \\
	             & = b \\
         (a * a) * a & = b * b \\
	             & = a
      \end{align*}
   \end{figure}

   $\mathbf{O_{15}}$\textbf{:} This operation is associative since it
   always evaluates to $b$.

   $\mathbf{O_{16}}$\textbf{:} This operation is associative since
   it always evaluates to $a$.

   Finally we have the following operations are associative: $O_1$, 
   $O_3$, $O_5$, $O_6$, $O_7$, $O_9$, $O_15$, $O_16$.

   \item For which of the operations $O_1$ to $O_{16}$ is there an 
   identity element?

   \noindent \Solution We can first rule out all operations that are not
   commutative. This is because we require that $x*e=e*x=x$. That leaves
   eight cases. Next we can rule out $O_{15}$ and $O_{16}$ because the
   former never returns a $b$ value, and the latter never returns an $a$
   value. Then we can rule out $O_8$ because there is no value of $y$ such
   that $b *_8 y = b$. Likewise, we can rule out $O_{14}$ because there
   is no value of $y$ such that $a *_{14} y = a$. That leaves four operations
   for consideration: $O_1, O_6, O_7$ and $O_9$. It will be shown
   that each of these has an identity element.

   The identity element of $O_1$ is $b$. This can be shown by noting
   $a *_{1} b = b *_{1} a = a$ and $b *_1 b = b$. The identity element
   of $O_6$ is $a$ because $a *_6 a = a$ and $b *_6 a = a *_6 b = b$.
   The identity element of $O_7$ is $a$. We can check that $a *_7 a = a$ and
   $b *_7 a = a *_7 b = a$. The identity element of $O_9$ is $b$.
   $a *_9 b = b *_9 a = a$ and $b *_9 b = b$. (These results are
   validated using boolean algebra in figure~\ref{fig:bool_ids}.)

   \begin{figure}
      \caption{List of identity elements for binary operations. In the figure
      $e_n$ corresponds to the identity element for $O_n$. For each operation
      we show that if we substitute $e_n$ in for the value of $y$ in the 
      equation for $O_n = x *_n y$ we derive the value $x$. The values
      $\lfalse, \ltrue$ correspond to the values $a, b$ respectively.}
      \label{fig:bool_ids}
      \begin{align*}
      O_1&=  x \land y  && \text{Operation 1}    \\
	 & = x \land \ltrue  && \text{Subst: } e_1 = \ltrue \\
	 & = x               && \qed                  \\
      O_6& = (\lnot x \land y) \lor (x \land \lnot y)
	                     && \text{Operation 6}    \\
	 & = (\lnot x \land \lfalse) \lor (x \land \lnot \lfalse)
	                     && \text{Subst: } e_6 = \lfalse \\
	 & = \lfalse \lor (x \land \ltrue)            \\
	 & = x               && \qed                  \\
      O_7& = x \lor y    && \text{Operation 7}    \\
	 & = x \lor \lfalse  && \text{Subst: } e_7 = \lfalse\\
	 & = x               && \qed                  \\
      O_9& = (\lnot x \land \lnot y) \lor (x \land y)
	                     && \text{Operation 9}    \\
	 & = (\lnot x \land \lnot \ltrue) \lor (x \land \ltrue)
	                     && \text{Subst: } e_9 = \ltrue \\
	 & = (\lnot x \land \lfalse) \lor x           \\
	 & = \lfalse \lor x                           \\
	 & = x               && \qed
      \end{align*}
   \end{figure}

   \item \Problem For which of the operations $O_1$ to $O_{16}$ does
   every element have an inverse?

   \noindent \Solution The answer is that only $O_6$ and $O_9$ provide
   inverses for every element.
   
   As we saw in the last problem only 4 operations
   have an identity: $O_1$, $O_6$, $O_7$ and $O_9$ so these are the only
   four operations we need to consider. $O_1$ does not have an inverse 
   for the value $a$, as there is no value of $y$ that makes the following
   equation true: $a * y = b$. Likewise, $O_7$ does not provide an inverse
   for the value $b$ as there is no value for $y$ which makes the 
   following equation true: $b * y = a$. 
   
   The remaining two operations do
   provide inverses for every element. For $O_6$ we have that 
   $a^{-1} = a \And b^{-1} = b$. For $O_9$ we have that 
   $a^{-1} = a \And b^{-1} = b$. In general for these two operations
   we have that $x^{-1} = x$.

\end{enumerate}

\noindent \large\textbf{D. $\quad$ Automata: The Algebra of Input/Output
  Sequences}

\brk \noindent Digital computers and related machines proces information
which is received in the form of input sequences. An \emph{input
  sequence} is a finite sequence of symbols from some alphabet
$A$. For instance, if $A=\{0,1\}$ (that is, if the alphabet consists
of only the two symbols 0 and 1), then examples of input sequences are
011010 and 1010111. If $A=\{a,b,c\}$, then examples of input sequences
are \emph{babbcac} and \emph{cccabaa}. \emph{Output sequences} are
defined in the same way as input sequences. The set of all sequences
of symbols in the alphabet $A$ is denoted by $A^*$.

There is an operation on $A^*$ called \emph{concatenation}: If $\mathbf{a}$ and
$\mathbf{b}$ are in $A^*$, say $\mathbf{a} = a_1 a_2 \ldots a_n$ and
$\mathbf{b} = b_1 b_2 \ldots b_m$, then \begin{align*} \mathbf{ab} & = a_1 a_2
\ldots a_n b_1 b_2 \ldots b_m \end{align*} In other words, the sequence
$\mathbf{ab}$ consists of the two sequences $\mathbf{a}$ and $\mathbf{b}$ end
to end.  For example, in the alphabet $A=\{0,1\}$, if $\mathbf{a} = 1001$ and
$\mathbf{b} = 010$, then $\mathbf{ab} = 1001010$.

The symbol $\lambda$ denotes the empty sequence.

\begin{enumerate}
   \item Prove that the operation above is associative.
   
   \begin{proof}
      Let $\mathbf{a} = a_1 a_2 \ldots a_r$, and $\mathbf{b} = b_1 b_2
      \ldots b_s$, and $\mathbf{c} = c_1 c_2 \ldots c_t$. Then we have
      \begin{align*}
        (\mathbf{ab})\mathbf{c} 
	               & = (a_1 a_2 \ldots a_r b_1 b_2 \ldots b_s)\mathbf{c}\\
	               & = a_1 a_2 \ldots a_r b_1 b_2 \ldots b_s c_1 c_2 
		           \ldots c_t \\
        \mathbf{a}(\mathbf{bc})
	               & = \mathbf{a}(b_1 b_2 \ldots b_s c_1 c_2 \ldots c_t)\\
	               & = a_1 a_2 \ldots a_r b_1 b_2 \ldots b_s c_1 c_2 
		           \ldots c_t \qedhere
      \end{align*}
   \end{proof}

   \item Explain why the operation is not commutative. 
   
   \noindent \Solution It is not commutative
   because commutativity changes the order of the symbols in the resulting
   output sequence. For example, let $\mathbf{a} = a_1 a_2 \ldots a_n$ and 
   $\mathbf{b} = b_1 b_2 \ldots b_m$. Then, as before, we have 
   $\mathbf{ab} = a_1 a_2 \ldots a_n b_1 b_2 \ldots b_m$. However, we
   get a different result for $\mathbf{ba}$: $\mathbf{ba} = b_1 b_2 \ldots
   b_m a_1 a_2 \ldots a_n$.

   \item Prove that there is an identity element for this operation.

   \begin{proof}
      We choose the identity element, $e$, to be the empty sequence $\lambda$,
      and again choose $\mathbf{a} = a_1 a_2 \ldots a_n$.
      Now we will demonstrate that $e$ is indeed the identity element.
      \begin{align*}
         \mathbf{a}\lambda & = a_1 a_2 \ldots a_n \lambda \\
	                   & = a_1 a_2 \ldots a_n 
			            && \text{Defn. of $\lambda$} \\
			   & = \mathbf{a} && \text{Defn. of $\mathbf{a}$} \\
	 \lambda\mathbf{a} & = \lambda a_1 a_2 \ldots a_n \\
	                   & = a_1 a_2 \ldots a_n 
			            && \text{Defn. of $\lambda$} \\
			   & = \mathbf{a} && \text{Defn. of
                             $\mathbf{a}$} \qedhere
      \end{align*} 
   \end{proof}

\end{enumerate}

\end{document}
