\documentclass{amsart}
\usepackage{amssymb,latexsym}
\usepackage{xspace}
\newcommand{\Rationals}{\mathbb{Q}{}}
\newcommand{\Reals}{\mathbb{R}{}}
\newcommand{\Integers}{\mathbb{Z}{}}
\newcommand{\Solution}{\textsc{Solution}\xspace}
\newcommand{\Problem}{\textsc{Problem}\xspace}
\begin{document}
These are my answers to the Chapter 2 exercises.

\section{Examples of Operations}
Which of the following rules are operations on the indicated set?  ($\Integers$
designates the set of the integers, $\Rationals$ the rational numbers, and
$\Reals$ the real numbers.) For each rule which is not an operation, explain
why it is not. 

\textbf{Example} $a * b = \displaystyle \frac{a + b}{ab}$, on the set
$\Integers$. 

\Solution This is not an operation on  $\Integers$. There are integers
$a$ and $b$ such that $(a+b)/ab$ is not an integer. (For example,

\[
\frac{2+3}{2 \cdot 3} = \frac{5}{6}
\]

is not an integer.) Thus, $\Integers$ is not closed under $*$. 

\begin{enumerate}

% #1
\item $a*b = \sqrt{|ab|}$, on the set $\Rationals$. 

\Solution This is not an operation on $\Rationals$. Here is a counter
example: $2*1 = \sqrt{|2 \cdot 1|} = \sqrt{2}$. The result is in irrational
number, and therefore $\Rationals$ is not closer under $*$.


% #2
\item $a*b = a \ln b$, on the set $\{x \in \Reals  : x > 0\}$. 

\Solution This is not an operation because it is not closed. A counter 
example is obtained by choosing $a=1$ and $b=0.5$: 
$1 \cdot 0.5 = \ln 0.5 \approx -0.693 < 0$. 


% #3
\item $a*b$ is a root of the equation $x^2 - a^2 b^2 = 0$, on the set $\Reals$.


\Solution This is not an operation because it is not uniquely defined.
$2*2$ is a root of the equation $x^2 - 2^2 2^2 = 0$, where $x = \pm 4$.


% #4
\item Subtraction, on the set $\Integers$.

\Solution This is an operation. It is defined for all pairs 
${(a,b) | a,b \in \Integers}$, it is uniquely defined for each pair, and it
is closed.


% #5
\item Subtraction, on the set $\{ n \in \Integers : n \ge 0\}$.

\Solution This is not an operation because it is not closed:
$4*10 = 4 - 10 = -4 < 0$.



\end{enumerate}


\section{Properties of Operations}

Each of the following is an operation $*$ on $\Reals$. Indicate whether or
not
\begin{enumerate}
   \item it is commutative,
   \item it is associative,
   \item $\Reals$ has an identity element with respect to $*$,
   \item every $x \in \Reals$ has an inverse with respect to $*$.
\end{enumerate}

\textbf{Instructions} For (1), computer $x * y$ and $y * x$, and verify whether
or not they are equal. For (2), compute $x * (y * z)$ and $(x * y) * z)$, and
verify whether or not they are equal. For (3), first solve the equation
$x * e = x$ for $e$; if the equation caonnot be solved, there is no identity
element. If it \emph{can} be solved, it is still necessary to check that
$e*x=x*e=x$ for any $x\in \Reals$. If it checks, then $e$ is an identity
element. For (4), first note that if there is no identity element, there can
be no inverses. If there is an identity element $e$, first solve the
equation $x*x'=e$ for $x'$; if the equation cannot be solved, $x$ does 
not have an inverse. If it \emph{can} be solved, check to make sure that
$x*x'=x'*x=e$. If this checks, $x'$ is the inverse of $x$.

\begin{enumerate}

   \item \Problem $x*y = x + 2y + 4$ 

   \noindent \Solution The solution follows:

   \begin{enumerate}
      \item The operation is not commutative
      \begin{align*}
         x * y & = x + 2y + 4 \\
	 y * x & = y + 2x + 4
      \end{align*}

      \item The operation is not associative
      \begin{align*}
         x * (y * z) & = x * (y + 2z + 4) \\
	             & = x + 2(y + 2z + 4) + 4 \\
		     & = x + 2y + 4z + 12  \\
\\
         (x * y) * z & = (x + 2y + 4) * z \\
	             & = (x + 2y + 4) + 2z + 4 \\
		     & = x + 2y + 2z + 8
      \end{align*}

      \item Check to see if their is an identity element:
      \begin{align*}
         x * e & = x           \\ 
         x * e & = x + 2e + 4  \\
	     x & = x + 2e + 4  \\
	     0 & =     2e + 4  \\
	    2e & =         -4  \\
	     e & =         -2  
      \end{align*}

      Now check to see if $x*e=e*x=x$:
      \begin{align*}
         x * -2 & = x + (2)(-2) + 4 \\
	        & = x - 4 + 4       \\
		& = x               \\
		\\
	 -2 * x & = -2 + 2x + 4     \\
	        & = 2x + 2          
      \end{align*}

      We can see $x*e \ne e*x$ therefore there is no identity element.

      \item No need to check for an inverse. There is no identity.
   \end{enumerate}

   \item \Problem $x * y = x + 2y - xy$

   \noindent \Solution The solution follows:

   \begin{enumerate}
      
      \item The operation is not commutative as we can see from
            (1) and (2) below:
      \begin{align*}
         x * y & = x + 2y - xy \\
	                       \\
	 y * x & = y + 2x - yx \\
	       & = 2x + y - xy \notag
      \end{align*}


      \item The operation is not associative
      \begin{align*}
         x * (y * z) & = x * (y + 2z - yz)                          \\
	             & = x + 2(y + 2z - yz) - x(y + 2z - yz) \notag \\
		     & = x + 2y + 4z - 2yz - xy - 2xz + xyz  \notag \\
		     & = x + 2y + 4z - xy - 2xz - 2yz + xyz  \notag \\
		     \notag \\
	 (x * y) * z & = (x + 2y - xy) * z                          \\
	             & = (x + 2y - xy) + 2z - (x + 2y - xy)z \notag \\
		     & = (x + 2y - xy) + 2z - xz - 2yz - xyz \notag \\
		     & = x + 2y + 2z - xy - xz - 2yz - xyz   \notag
      \end{align*}

      \item From the following we can see that $e=0$. 
      \begin{align*}
         x * e & = x                  \\
	 \\
	 x * e & = x + 2e - xe        \\
	     x & = x + 2e - xe \notag \\
	     0 & =     2e - xe \notag \\
	e(2-x) & = 0           \notag \\
	     e & = 0 \quad\text{(when $x \ne 2$)} \notag 
      \end{align*}
      We need to check what happens when $x = 0$: we have 
      $0*e=0+2e-0e$ which reduces to $0 = 2e$ and again we get $e = 0$.
      Now we must check that $x*e=e*x=x$
      \begin{align*}
         e * x & = e + 2x - ex      \\
	       & = 0 + 2x - 0       \\
	       & = 2x               \\
	       & \ne x              
      \end{align*}
      Therefore, there is no identity element.

      \item Since there is no identity element, there are no inverses.

   \end{enumerate}

   \item \Problem $x * y = | x + y |$

   \noindent \Solution The solution follows:

   \begin{enumerate}
      \item From the following we can see that the operation is commutative:
      \begin{align*}
         x * y & = | x + y |         \\
	 \\
	 y * x & = | y + x |         \\
	       & = | x + y | \notag  
      \end{align*}

      \item It is not associative. Here is a counter example using
      $x = 7, y = -13, z = 1$

      \begin{align*}
         x * (y * z) & = 7 * | -13 + 1 | \\
	             & = | 7 + 12 | \notag \\
		     & = 19         \notag \\
		                    \notag \\
	 (x * y) * z & = | 7 + -13 | * 1   \\
	             & = | -6 | * 1 \notag \\
		     & = | 6 + 1 |  \notag \\
		     & = 7          \notag \\
		     & \ne 19       \notag
      \end{align*}

      \item The value of $0$ works as an identity element for non-negative
         numbers. However, there is no possible identity element for negative
         numbers. This is because the result of the absolute value
	 function always returns non-negative values.

      \item Since there is no identity element, there are no inverses.

   \end{enumerate}


   \item \Problem $x * y = | x - y |$

   \noindent \Solution The solution is as follows:
   \begin{enumerate}

      \item The operation is commutative:
      \begin{align*}
         x * y & = | x - y |           \\
	       & =
	       \begin{cases}
	          x - y,     & \text{if $x \ge y$;} \\
		  y - x,     & \text{otherwise.}
	       \end{cases} \notag \\
	                                         \notag \\
	 y * x & = | y - x |           \\
	       & =
	       \begin{cases}
	          x - y,     & \text{if $x \ge y$;} \\
		  y - x,     & \text{otherwise.}
	       \end{cases} \notag 
      \end{align*}

      \item The operation is not associative. A counter example is provided
      by choosing $x = 1, y = -2, z = 3$.
      \begin{align*}
         x * (y * z) & = 1 * |(-2) - 3|    \\
	             & = 1 * 5       \notag \\
		     & = | 1 - 5 |   \notag \\
		     & = 4           \notag \\
		                     \notag \\
	 (x * y) * z & = | 1 - (-2) | * 3  \\
	             & = 3 * 3       \notag \\
		     & = | 3 - 3 |   \notag \\
		     & = 0           \notag \\
		     & \ne 4         \notag
      \end{align*}

      \item Again, there can be no identity element, since the result of
      the operation is always non-negative. Therefore, there can be 
      no number $x * e = x$ for $x < 0$.

      \item Again, since there is no identity element, there can be no inverses.
      However, $-x$ acts as an inverse for any number assuming that $0$ actually
      was an identity element.
   \end{enumerate}

   \item \Problem $x*y = xy + 1$

   \noindent \Solution The solution is as follows:

   \begin{enumerate}

      \item The operation is commutative as seen below.
      \begin{align*}
         x * y & = xy + 1  \\
	 \\
	 y * x & = yx + 1 \\
	       & = xy + 1 \quad\text{(Multiplication is commutative.)}
      \end{align*}

      \item The operation is not associative as seen below.
      \begin{align*}
         x * (y * z) & = x * (yz + 1) \\
	             & = x(yz + 1) + 1 \\
		     & = x + xyz + 1 \\
		     \\
	 (x * y) * z & = (xy + 1) * z \\
	             & = (xy + 1)z + 1 \\
		     & = z + xyz + 1 \\
		     & \ne x + xyz + 1
      \end{align*}

      \item We can see below that their is no constant value defined for
      the identity element. Our equations find a formula for $e$ which
      depens on $x$. This is not a constant value.
      \begin{align*}
         x * e & = x \\
	 \\
	 x * e & = xe + 1 \\
	 \\
	     x & = xe + 1 \\
	 x - 1 & = xe     \\
	 xe    & = x - 1  \\
	  e    & = \frac{x - 1}{x} \\
      \end{align*}

      \item There is no identity element, so there are no inverses.
   \end{enumerate}

   \item \Problem $x * y = \max(x,y)$, the larger of the two numbers $x$ and $y$.

   \noindent \Solution The solution is as follows.

   \begin{enumerate}
      
      \item The operation is commutative.
      \begin{align*}
         x * y & = \max(x,y)    \\
	       & =
	       \begin{cases}
	          x,       & \text{if $x \ge y$;} \\
		  y,       & \text{otherwise.}
	       \end{cases} \\
	       \\
	 y * x & = \max(y,x)   \\
	       & =
	       \begin{cases}
	          x,       & \text{if $x \ge y$;} \\
		  y,       & \text{otherwise.}
	       \end{cases}
      \end{align*}

      \item The operation is associative.
      \begin{align*}
         x * (y * z) & = x * \max(y,z) \\
	             &= \max(x, \max(y,z)) \\
		     &=
		     \begin{cases}
		        \max(x, y), \text{if $y >= z$;} \\
			\max(x, z), \text{otherwise.}
		     \end{cases} \\
		     & =
		     \begin{cases}
		        x,  & \text{if $x \ge y \wedge x \ge z$;} \\
			y,  & \text{if $x < y \wedge y \ge z$;}   \\
			z,  & \text{otherwise.}
		     \end{cases} \\
		     \\
	 (x * y) * z & = \max(x,y) * z \\
	             & = \max(\max(x,y), z) \\
		     & =
		     \begin{cases}
		        \max(x,z), \text{if $x >= y$;} \\
			\max(y,z), \text{otherwise.}
		     \end{cases} \\
		     & =
		     \begin{cases}
		        x, & \text{if $x >= y \wedge x >= z$;} \\
			y, & \text{if $x < y \wedge y >= z$;} \\
			z, & \text{otherwise.}
		     \end{cases}
      \end{align*}

      \item There is no identity element. We will prove this by contradiction.
      Assume that there is some identity element $e$. Then by definition 
      $x * e = x$ forall $x$. Let us choose a value $m = e - 1$. Then
      we have $m * e = (e-1) * e = \max(e-1,e) = e \ne m$. Therefore, e
      is not an identity element. \qed.

      \item Since there is no identity element, there are no inverses.
      
   \end{enumerate}

   \item \Problem $\displaystyle x * y = \frac{xy}{x + y + 1}$ on the set of
   positive real numbers.

   \noindent \Solution The solution is as follows
   \begin{enumerate}
      \item The operation is commutative.
      \begin{align*}
         x * y & = \frac{xy}{x + y + 1} \\
	 \\
	 y * x & = \frac{yx}{y + x + 1} \\
	       & = \frac{xy}{x + y + 1} && \text{(+,$\cdot$ are commutative)}
      \end{align*}

      \item The operation is not associative. This can be demonstrated 
      with the values $x = 2, y = 3, z = 4$.
      \begin{align*}
         2 * (3 * 4) & = 2 * \frac{3 \cdot 4}{3 + 4 + 1}   \\
	             & = 2 * (3/2) \\
		     & = \frac{2 \cdot (3/2)}{2 + (3/2) + 1} \\
		     & = \frac{6/2}{9/2} \\
		     & = \frac{6}{9} \\
		     & = \frac{2}{3} \\
		     \\
	 (2 * 3) * 4 & = \frac{2 \cdot 3}{2 + 3 + 1} * 4 \\
	             & = \frac{6}{7} * 4 \\
		     & = \frac{(6/7) \cdot 4}{(3/7) + 4 + 1} \\
		     & = \frac{24/7}{38/7} \\
		     & = \frac{24}{38} \\
		     & = \frac{12}{19}
      \end{align*}

      \item There is no identity element. As a matter of fact,
      the equation $x * e = x$ can only be solved when $x = 0$ or $x = -1$, as 
      shown below.
      \begin{align*}
         \frac{ex}{e + x + 1} & = x \\
	 ex                   & = x^2 + ex + x \\
	 x^2 + ex + x         & = ex \\
	 x^2 + x              & = 0 \\
	 x(x+1)               & = 0 \\
	 x                    & = 0, -1
      \end{align*}

      \item Since there is no identity, there can be no inverses.
   \end{enumerate}

\end{enumerate}

\section{Operations on a Two-Element Set}
Let $A$ be the two-element set $A = \{a,b\}$.
\begin{enumerate}
   \item \Problem Write the tables of all 16 operations on $A$. (Use the format
   explained on page 20.) Label these operations $O_1$ to $O_{16}$. 

   \noindent \Solution The tables are listed below.

   \begin{tabular}{cccc}
      & & & \\
      \begin{tabular}{c|c}    
         \multicolumn{2}{c}{$O_{16}$} \\ \hline
         $(x,y)$ & $x * y$ \\ \hline
	 $(a,a)$ & $a$     \\ 
	 $(a,b)$ & $a$     \\ 
	 $(b,a)$ & $a$     \\ 
	 $(b,b)$ & $a$     
      \end{tabular} & 
      \begin{tabular}{c|c}   
         \multicolumn{2}{c}{$O_{1}$} \\ \hline
         $(x,y)$ & $x * y$ \\ \hline
	 $(a,a)$ & $a$     \\ 
	 $(a,b)$ & $a$     \\ 
	 $(b,a)$ & $a$     \\ 
	 $(b,b)$ & $b$     
      \end{tabular} & 
      \begin{tabular}{c|c}  
         \multicolumn{2}{c}{$O_{2}$} \\ \hline
         $(x,y)$ & $x * y$ \\ \hline
	 $(a,a)$ & $a$     \\ 
	 $(a,b)$ & $a$     \\ 
	 $(b,a)$ & $b$     \\ 
	 $(b,b)$ & $a$     
      \end{tabular} & 
      \begin{tabular}{c|c} 
         \multicolumn{2}{c}{$O_{3}$} \\ \hline
         $(x,y)$ & $x * y$ \\ \hline
	 $(a,a)$ & $a$     \\ 
	 $(a,b)$ & $a$     \\ 
	 $(b,a)$ & $b$     \\ 
	 $(b,b)$ & $b$     
      \end{tabular}  \\

      & & & \\
      \begin{tabular}{c|c}
         \multicolumn{2}{c}{$O_{4}$} \\ \hline
         $(x,y)$ & $x * y$ \\ \hline
	 $(a,a)$ & $a$     \\ 
	 $(a,b)$ & $b$     \\ 
	 $(b,a)$ & $a$     \\ 
	 $(b,b)$ & $a$     
      \end{tabular} & 
      \begin{tabular}{c|c}
         \multicolumn{2}{c}{$O_{5}$} \\ \hline
         $(x,y)$ & $x * y$ \\ \hline
	 $(a,a)$ & $a$     \\ 
	 $(a,b)$ & $b$     \\ 
	 $(b,a)$ & $a$     \\ 
	 $(b,b)$ & $b$     
      \end{tabular} & 
      \begin{tabular}{c|c}
         \multicolumn{2}{c}{$O_{6}$} \\ \hline
         $(x,y)$ & $x * y$ \\ \hline
	 $(a,a)$ & $a$     \\ 
	 $(a,b)$ & $b$     \\ 
	 $(b,a)$ & $b$     \\ 
	 $(b,b)$ & $a$     
      \end{tabular} & 
      \begin{tabular}{c|c}
         \multicolumn{2}{c}{$O_{7}$} \\ \hline
         $(x,y)$ & $x * y$ \\ \hline
	 $(a,a)$ & $a$     \\ 
	 $(a,b)$ & $b$     \\ 
	 $(b,a)$ & $b$     \\ 
	 $(b,b)$ & $b$     
      \end{tabular}  \\

      & & & \\
      \begin{tabular}{c|c}
         \multicolumn{2}{c}{$O_{8}$} \\ \hline
         $(x,y)$ & $x * y$ \\ \hline
	 $(a,a)$ & $b$     \\ 
	 $(a,b)$ & $a$     \\ 
	 $(b,a)$ & $a$     \\ 
	 $(b,b)$ & $a$     
      \end{tabular} & 
      \begin{tabular}{c|c}
         \multicolumn{2}{c}{$O_{9}$} \\ \hline
         $(x,y)$ & $x * y$ \\ \hline
	 $(a,a)$ & $b$     \\ 
	 $(a,b)$ & $a$     \\ 
	 $(b,a)$ & $a$     \\ 
	 $(b,b)$ & $b$     
      \end{tabular} & 
      \begin{tabular}{c|c}
         \multicolumn{2}{c}{$O_{10}$} \\ \hline
         $(x,y)$ & $x * y$ \\ \hline
	 $(a,a)$ & $b$     \\ 
	 $(a,b)$ & $a$     \\ 
	 $(b,a)$ & $b$     \\ 
	 $(b,b)$ & $a$     
      \end{tabular} & 
      \begin{tabular}{c|c}
         \multicolumn{2}{c}{$O_{11}$} \\ \hline
         $(x,y)$ & $x * y$ \\ \hline
	 $(a,a)$ & $b$     \\ 
	 $(a,b)$ & $a$     \\ 
	 $(b,a)$ & $b$     \\ 
	 $(b,b)$ & $b$     
      \end{tabular}  \\

      & & & \\
      \begin{tabular}{c|c}
         \multicolumn{2}{c}{$O_{12}$} \\ \hline
         $(x,y)$ & $x * y$ \\ \hline
	 $(a,a)$ & $b$     \\ 
	 $(a,b)$ & $b$     \\ 
	 $(b,a)$ & $a$     \\ 
	 $(b,b)$ & $a$     
      \end{tabular} & 
      \begin{tabular}{c|c}
         \multicolumn{2}{c}{$O_{13}$} \\ \hline
         $(x,y)$ & $x * y$ \\ \hline
	 $(a,a)$ & $b$     \\ 
	 $(a,b)$ & $b$     \\ 
	 $(b,a)$ & $a$     \\ 
	 $(b,b)$ & $b$     
      \end{tabular} & 
      \begin{tabular}{c|c}
         \multicolumn{2}{c}{$O_{14}$} \\ \hline
         $(x,y)$ & $x * y$ \\ \hline
	 $(a,a)$ & $b$     \\ 
	 $(a,b)$ & $b$     \\ 
	 $(b,a)$ & $b$     \\ 
	 $(b,b)$ & $a$     
      \end{tabular} & 
      \begin{tabular}{c|c}
         \multicolumn{2}{c}{$O_{15}$} \\ \hline
         $(x,y)$ & $x * y$ \\ \hline
	 $(a,a)$ & $b$     \\ 
	 $(a,b)$ & $b$     \\ 
	 $(b,a)$ & $b$     \\ 
	 $(b,b)$ & $b$     
      \end{tabular}  \\
   \end{tabular}
\end{enumerate}

\end{document}
