\documentclass[draft,twoside]{amsart}
\usepackage{amssymb,latexsym}
\usepackage{xspace}
\newcommand{\Rationals}{\mathbb{Q}{}}
\newcommand{\Reals}{\mathbb{R}{}}
\newcommand{\Integers}{\mathbb{Z}{}}
\newcommand{\Solution}{\textsc{Solution}\xspace}
\newcommand{\Problem}{\textsc{Problem}\xspace}
\newcommand{\Blank}{\mathrel{\phantom{=}}}
\newcommand{\ltrue}{\top}
\newcommand{\lfalse}{\bot}
\begin{document}
\title{Answers to Chapter 3 Excercises - A Book of Abstract Algebra}
\author{Michael Welch}
\date{\today}
\maketitle

\section{Examples of Abelian Groups (Exercise A)}

Prove that each of the following sets, with the indicated operation, is an
abelian group.

\textbf{Instructions} Proceed as in Chapter 2, Exercise B.

\begin{enumerate}
   \item $x * y = x + y + k$ (k is a fixed constant), on the set 
   $\Reals$ of the real numbers.

   \begin{enumerate} % 4 parts: assoc, commut, identity, inverse
      \item 
      \begin{proof}[Proof that operation is commutative]
         \begin{align*}
	    x * y & = x + y + k \\
	    y * x & = y + x + k \\
	          & = x + y + k && \text{Commute addition} \qedhere
	 \end{align*}
      \end{proof}

      \item 
      \begin{proof}[Proof that operation is associative]
         \begin{align*}
	    x * (y * z) & = x * (y + z + k) \\
	                & = x + (y + z + k) + k \\
			& = x + y + z + 2k \\
	    (x * y) * z & = (x + y + k) * z \\
	                & = (x + y + k) + z + k \\
			& = x + y + z + 2k \qedhere
	 \end{align*}
      \end{proof}

      \item The identity element can be calculated:
      \begin{align*}
         x * e & = x         \\
         x * e & = x + e + k \\
	     x & = x + e + k \\
	     e & = -k
      \end{align*}

      We know by commuativity that $x * (-k) = (-k) * x$. Therefore,
      $-k$ is the identity element.

      \item We can caluclate the inverse of $x$ which we label $x'$.
      \begin{align*}
         x * x' & = -k          \\
	 x * x' & = x + x' + k  \\
	    -k  & = x + x' + k  \\
	     x' & = -x - 2k
      \end{align*}

   \end{enumerate}

   \item $\displaystyle x * y = \frac{xy}{2}$, on the set $\{x \in \Reals
   : x \ne 0\}$
   
      \begin{enumerate}
         \item
	 \begin{proof}[Proof the operation is commutative]
	    \begin{align*}
	       x * y & = \frac{xy}{2}         \\
               y * x & = \frac{yx}{2}         \\
	             & = \frac{xy}{2} \qedhere
	    \end{align*}
	 \end{proof}

	 \item
	 \begin{proof}[Proof the operation is associative]
	    \begin{align*}
	       x * (y * z) & = x * \frac{yz}{2}   \\
	                   & = \frac{x\displaystyle\frac{yz}{2}}{2} \\
			   & = \frac{xyz}{4} \\
	       (x * y) * z & = \frac{xy}{2} * z \\
	                   & = \frac{\displaystyle \frac{xy}{2}z}{2} \\
			   & = \frac{xyz}{4} \qedhere
	    \end{align*}
	 \end{proof}

	 \item The following calculation shows that the identity element
	 is 2.
	 \begin{align*}
	    x * e & = x              \\
	    x * e & = \frac{xe}{2}   \\
	        x & = \frac{xe}{2}   \\
		1 & = \frac{e}{2}    \\
		e & = 2
	 \end{align*}

	 \item The following calculation shows that the inverse of
	 $x$ is $4/x$.
	 \begin{align*}
	    x * x' & = 2             \\
	    x * x' & = \frac{xx'}{2} \\
	        2  & = \frac{xx'}{2} \\
		4  & = xx'           \\
		x' & = 4/x    && \text{$x \ne 0$ is given}
	 \end{align*}
      \end{enumerate}

      \item $x * y = x + y + xy$, on the set $\{x \in \Reals : x \ne -1\}$
      \begin{enumerate}
         \item
	 \begin{proof}[Proof the operation is commutative]
	    \begin{align*}
	       x * y & = x + y + xy \\
	       y * x & = y + x + yx \\
	             & = x + y + xy \qedhere
	    \end{align*}
	 \end{proof}

	 \item
	 \begin{proof}[Proof the operation is associative]
	    \begin{align*}
	       x * (y * z) & = x * (y + z + yz) \\
	                   & = x + (y + z + yz) + x(y + z + yz) \\
			   & = x + y + z + yz + xy + xz + xyz\\
			   & = x + y + xy + xz + yz + xyz \\
	       (x * y) * z & = (x + y + xy) * z \\
	                   & = (x + y + xy) + z + (x + y + xy)z \\
			   & = x + y + xy + z + xz + yz + xyz \\
			   & = x + y + xy + xz + yz + xyz \qedhere
	    \end{align*}
	 \end{proof}

	 \item The following calculation shows the identity element to
	 be 0.
	 \begin{align*}
	    x * e & = x           \\
	    x * e & = x + e + xe  \\
	        x & = x + e + xe  \\
		0 & =     e + xe  \\
		0 & =   e(1 + x)  \\
		e & = 0 && \text{$x\ne-1$ given}
	 \end{align*}

         \item The following calculation shows the inverse of $x$ is 
	 $\displaystyle \frac{-x}{1+x}$.
	 \begin{align*}
	    x * x' & = 0            \\
	    x * x' & = x + x' + xx' \\
	         0 & = x + x' + xx' \\
		 0 & = x + x'(1 + x) \\
		-x & = x'(1+x) \\
	    x'(1+x)& = -x \\
	        x' & = \frac{-x}{1+x} && \text{$x\ne-1$ given}
	 \end{align*}

      \end{enumerate}

      \item $x * y = \displaystyle \frac{x + y}{xy + 1}$, on the set
      $\{x \in \Reals : -1 < x < 1\}$.
      \begin{enumerate}
	 \item
         \begin{proof}[Proof the operation is commutative]
	    \begin{align*}
	       x * y & = \frac{x+y}{xy + 1} \\
	       y * x & = \frac{y+x}{yx + 1} \\
	             & = \frac{x+y}{xy + 1} \qedhere
	    \end{align*}
	 \end{proof}

	 \item
	 \begin{proof}[Proof the operation is associative]
	    \begin{align*}
	       x * (y * z) & = x * \frac{y + z}{yz + 1} \\
	                   & = \frac{x + \displaystyle \frac{y + z}{yz + 1}}
			       {x\displaystyle \frac{y + z}{yz + 1} + 1} \\
			   & = \frac{x(yz+1)+y+z}{xy+xz+yz+1} \\
			   & = \frac{x + y + z + xyz}{xy + xz + yz + 1}
			       \qedhere
	    \end{align*}
	 \end{proof}

	 \item The following caluclation shows the identity element to be 0.
	 \begin{align*}
	    x * e & = x \\
	    x * e & = \frac{x + e}{xe + 1} \\
	        x & = \frac{x + e}{xe + 1} \\
            x + e & = x(xe + 1) \\
	    x + e & = x^2e + x \\
	    e - x^2e & = 0  \\
	    e(1-x^2) & = 0    \\
	        e & = 0  && \text{$-1 < x < 1$ is given}
	 \end{align*}

	 \item The following caluclation shows the inverse of $x$ to be
	 $-x$.
	 \begin{align*}
	    x * x' & = 0                      \\
	    x * x' & = \frac{x + x'}{xx' + 1} \\
	        0  & = \frac{x + x'}{xx' + 1} \\
		0  & = x + x' \\
	        x' & = -x
	 \end{align*}
      \end{enumerate}
\end{enumerate}

\section{Groups on the Set $\Reals \times \Reals (Exercise B)$}

The symbol $\Reals \times \Reals$ represents the set of all ordered pairs
$(x,y)$ of real numbers.  $\Reals \times \Reals$ may therefore be identified
with the set of all the points in the plane.  Which of the following subsets of
$\Reals \times \Reals$, with the indicated operation, is a group?  Which is an
abelian group?

\textbf{Instructions} Proceed as in the preceding exercise. To find the
identity element, which in these problems is an ordered pair $(e_1, e_2)$ of
real numbers, solve the equation $(a,b) * (e_1, e_2) = (a,b)$ for $e_1$ and
$e_2$. To find the inverse $(a',b')$ of $(a,b)$, solve the equation
$(a,b)*(a',b')=(e_1,e_2)$ for $a'$ and $b'$. [Remember that $(x,y)=(x',y')$ if
and only if $x=x'$ and $y=y'$.]

   \begin{enumerate}
      \item $(a,b)*(c,d)=(ad+bc,bd)$, on the set $\{(x,y) \in \Reals
      \times \Reals : y \ne 0\}$.
      \begin{enumerate}
         \item The operation is commutative.
	 \begin{proof}
	    \begin{align*}
	       (a,b)*(c,d) & = (ad+bc,bd) \\
	       (c,d)*(a,b) & = (cb+da,db) \\
	                   & = (ad+bc,bd) \qedhere
	    \end{align*}
	 \end{proof}

	 \item The operation is associative.
	 \begin{proof}
	    \begin{align*}
	    \end{align*}
	 \end{proof}
      \end{enumerate}
   \end{enumerate}

\end{document}
