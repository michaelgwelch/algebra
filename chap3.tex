\documentclass[twoside]{amsart}
\usepackage{amssymb,latexsym}
\usepackage{xspace}
\usepackage{enumerate}
\usepackage{graphics}
\newcommand{\Rationals}{\mathbb{Q}{}}
\newcommand{\Reals}{\mathbb{R}{}}
\newcommand{\Integers}{\mathbb{Z}{}}
\newcommand{\Solution}{\textsc{Solution}\xspace}
\newcommand{\Problem}{\textsc{Problem}\xspace}
\newcommand{\Blank}{\mathrel{\phantom{=}}}
\newcommand{\ltrue}{\top}
\newcommand{\lfalse}{\bot}
\begin{document}
\title{Answers to Chapter 3 Exercises - A Book of Abstract Algebra}
\author{Michael Welch}
\date{\today}
\maketitle

This document contains selected answers to exercises from chapter 3
of A Book of Abstract Algebra.

\begin{enumerate}[A.]
   \item \textsc{Examples of Abelian Groups}

   Prove that each of the following sets, with the indicated operation, is an
   abelian group.

   \textbf{Instructions} Proceed as in Chapter 2, Exercise B.

   \begin{enumerate}[1.]
      \item $x * y = x + y + k$ (k is a fixed constant), on the set 
      $\Reals$ of the real numbers.

      \begin{enumerate} % 4 parts: assoc, commut, identity, inverse
	 \item 
	 \begin{proof}[Proof that operation is commutative]
	    \begin{align*}
	       x * y & = x + y + k \\
	       y * x & = y + x + k \\
		     & = x + y + k && \text{Commute addition} \qedhere
	    \end{align*}
	 \end{proof}

	 \item 
	 \begin{proof}[Proof that operation is associative]
	    \begin{align*}
	       x * (y * z) & = x * (y + z + k) \\
			   & = x + (y + z + k) + k \\
			   & = x + y + z + 2k \\
	       (x * y) * z & = (x + y + k) * z \\
			   & = (x + y + k) + z + k \\
			   & = x + y + z + 2k \qedhere
	    \end{align*}
	 \end{proof}

	 \item The identity element can be calculated:
	 \begin{align*}
	    x * e & = x         \\
	    x * e & = x + e + k \\
		x & = x + e + k \\
		e & = -k
	 \end{align*}

	 We know by commuativity that $x * (-k) = (-k) * x$. Therefore,
	 $-k$ is the identity element.

	 \item We can caluclate the inverse of $x$ which we label $x'$.
	 \begin{align*}
	    x * x' & = -k          \\
	    x * x' & = x + x' + k  \\
	       -k  & = x + x' + k  \\
		x' & = -x - 2k
	 \end{align*}

      \end{enumerate}

      \item $\displaystyle x * y = \frac{xy}{2}$, on the set $\{x \in \Reals
      : x \ne 0\}$
      
	 \begin{enumerate}
	    \item
	    \begin{proof}[Proof the operation is commutative]
	       \begin{align*}
		  x * y & = \frac{xy}{2}         \\
		  y * x & = \frac{yx}{2}         \\
			& = \frac{xy}{2} \qedhere
	       \end{align*}
	    \end{proof}

	    \item
	    \begin{proof}[Proof the operation is associative]
	       \begin{align*}
		  x * (y * z) & = x * \frac{yz}{2}   \\
			      & = \frac{x\displaystyle\frac{yz}{2}}{2} \\
			      & = \frac{xyz}{4} \\
		  (x * y) * z & = \frac{xy}{2} * z \\
			      & = \frac{\displaystyle \frac{xy}{2}z}{2} \\
			      & = \frac{xyz}{4} \qedhere
	       \end{align*}
	    \end{proof}

	    \item The following calculation shows that the identity element
	    is 2.
	    \begin{align*}
	       x * e & = x              \\
	       x * e & = \frac{xe}{2}   \\
		   x & = \frac{xe}{2}   \\
		   1 & = \frac{e}{2}    \\
		   e & = 2
	    \end{align*}

	    \item The following calculation shows that the inverse of
	    $x$ is $4/x$.
	    \begin{align*}
	       x * x' & = 2             \\
	       x * x' & = \frac{xx'}{2} \\
		   2  & = \frac{xx'}{2} \\
		   4  & = xx'           \\
		   x' & = 4/x    && \text{$x \ne 0$ is given}
	    \end{align*}
	 \end{enumerate}

	 \item $x * y = x + y + xy$, on the set $\{x \in \Reals : x \ne -1\}$
	 \begin{enumerate}
	    \item
	    \begin{proof}[Proof the operation is commutative]
	       \begin{align*}
		  x * y & = x + y + xy \\
		  y * x & = y + x + yx \\
			& = x + y + xy \qedhere
	       \end{align*}
	    \end{proof}

	    \item
	    \begin{proof}[Proof the operation is associative]
	       \begin{align*}
		  x * (y * z) & = x * (y + z + yz) \\
			      & = x + (y + z + yz) + x(y + z + yz) \\
			      & = x + y + z + yz + xy + xz + xyz\\
			      & = x + y + xy + xz + yz + xyz \\
		  (x * y) * z & = (x + y + xy) * z \\
			      & = (x + y + xy) + z + (x + y + xy)z \\
			      & = x + y + xy + z + xz + yz + xyz \\
			      & = x + y + xy + xz + yz + xyz \qedhere
	       \end{align*}
	    \end{proof}

	    \item The following calculation shows the identity element to
	    be 0.
	    \begin{align*}
	       x * e & = x           \\
	       x * e & = x + e + xe  \\
		   x & = x + e + xe  \\
		   0 & =     e + xe  \\
		   0 & =   e(1 + x)  \\
		   e & = 0 && \text{$x\ne-1$ given}
	    \end{align*}

	    \item The following calculation shows the inverse of $x$ is 
	    $\displaystyle \frac{-x}{1+x}$.
	    \begin{align*}
	       x * x' & = 0            \\
	       x * x' & = x + x' + xx' \\
		    0 & = x + x' + xx' \\
		    0 & = x + x'(1 + x) \\
		   -x & = x'(1+x) \\
	       x'(1+x)& = -x \\
		   x' & = \frac{-x}{1+x} && \text{$x\ne-1$ given}
	    \end{align*}

	 \end{enumerate}

	 \item $x * y = \displaystyle \frac{x + y}{xy + 1}$, on the set
	 $\{x \in \Reals : -1 < x < 1\}$.
	 \begin{enumerate}
	    \item
	    \begin{proof}[Proof the operation is commutative]
	       \begin{align*}
		  x * y & = \frac{x+y}{xy + 1} \\
		  y * x & = \frac{y+x}{yx + 1} \\
			& = \frac{x+y}{xy + 1} \qedhere
	       \end{align*}
	    \end{proof}

	    \item
	    \begin{proof}[Proof the operation is associative]
	       \begin{align*}
		  x * (y * z) & = x * \frac{y + z}{yz + 1} \\
			      & = \frac{x + \displaystyle \frac{y + z}{yz + 1}}
				  {x\displaystyle \frac{y + z}{yz + 1} + 1} \\
			      & = \frac{x(yz+1)+y+z}{xy+xz+yz+1} \\
			      & = \frac{x + y + z + xyz}{xy + xz + yz + 1}
				  \qedhere
	       \end{align*}
	    \end{proof}

	    \item The following caluclation shows the identity element to be 0.
	    \begin{align*}
	       x * e & = x \\
	       x * e & = \frac{x + e}{xe + 1} \\
		   x & = \frac{x + e}{xe + 1} \\
	       x + e & = x(xe + 1) \\
	       x + e & = x^2e + x \\
	       e - x^2e & = 0  \\
	       e(1-x^2) & = 0    \\
		   e & = 0  && \text{$-1 < x < 1$ is given}
	    \end{align*}

	    \item The following caluclation shows the inverse of $x$ to be
	    $-x$.
	    \begin{align*}
	       x * x' & = 0                      \\
	       x * x' & = \frac{x + x'}{xx' + 1} \\
		   0  & = \frac{x + x'}{xx' + 1} \\
		   0  & = x + x' \\
		   x' & = -x
	    \end{align*}
	 \end{enumerate}
   \end{enumerate}

   \item \textsc{Groups on the Set $\Reals \times \Reals$}

   The symbol $\Reals \times \Reals$ represents the set of all ordered pairs
   $(x,y)$ of real numbers.  $\Reals \times \Reals$ may therefore be identified
   with the set of all the points in the plane.  Which of the following subsets
   of $\Reals \times \Reals$, with the indicated operation, is a group?  Which
   is an abelian group?

   \textbf{Instructions} Proceed as in the preceding exercise. To find the
   identity element, which in these problems is an ordered pair $(e_1, e_2)$ of
   real numbers, solve the equation $(a,b) * (e_1, e_2) = (a,b)$ for $e_1$ and
   $e_2$. To find the inverse $(a',b')$ of $(a,b)$, solve the equation
   $(a,b)*(a',b')=(e_1,e_2)$ for $a'$ and $b'$. [Remember that $(x,y)=(x',y')$
   if and only if $x=x'$ and $y=y'$.]

      \begin{enumerate}[1.]
	 \item $(a,b)*(c,d)=(ad+bc,bd)$, on the set $\{(x,y) \in \Reals
	 \times \Reals : y \ne 0\}$.
	 \begin{enumerate}
	    \item The operation is commutative.
	    \begin{proof}
	       \begin{align*}
		  (a,b)*(c,d) & = (ad+bc,bd) \\
		  (c,d)*(a,b) & = (cb+da,db) \\
			      & = (ad+bc,bd) \qedhere
	       \end{align*}
	    \end{proof}

	    \item The operation is associative.
	    \begin{proof}
	       \begin{align*}
		  (a,b) * ((c,d) * (e,f)) & = (a,b) * (cf+de,df)  \\
					  & = (adf + b(cf+de), bdf) \\
					  & = (adf +bcf + bde, bdf) \\
		  ((a,b) * (c,d)) * (e,f) & = (ad+bc,bd) * (e,f)    \\
					  & = ((ad+bc)f+bde, bdf)   \\
					  & = (adf+bcf+bde,bdf) \qedhere
	       \end{align*}
	    \end{proof}

	    \item 
	    \begin{align*}
	       (a,b) * (e_1,e_2) & = (a,b)            \\
	       (a,b) * (e_1,e_2) & = (ae_2+be_1,be_2) \\
		       (a,b)     & = (ae_2+be_1,be_2) 
	    \end{align*}
	    This implies that $a=ae_2+be_1$ and $b=be_2$. So $e_2=1$
	    and $e_1=0$. So the identity element is $(0,1)$.

	    \item
	    \begin{align*}
	       (a,b) * (a',b') & = (0,1) \\
	       (a,b) * (a',b') & = (ab'+ba',bb') \\
		 (ab'+ba',bb') & = (0,1) \\
	    \end{align*}
	    This implies $ab'+ba'=0$ and $bb'=1$. So $b'=1/b$. Let's
	    solve for $a$:
	    \begin{align*}
	       ab'+ba' & = 0         \\
	       a(1/b) + ba' & = 0    \\
		    ba' & = -1(a/b)  \\
		    a' & = -a/b^2
	    \end{align*}
	    So $(a',b') = (-a/b^2,1/b)$. 

	 \end{enumerate}
      \end{enumerate} 
   \item \textsc{Groups of Subsets of a Set}

   If $A$ and $B$ are any two sets, their \emph{symmetric difference}
   is the set $A \ominus B$ defined as follows (see figure~\ref{fig:symmdif}):
   \begin{center}
   $$ A\ominus B = (A-B) \cup (B-A) $$
   \end{center}

   \textsc{Note}: $A - B$ represents the set obtained by removing from
   $A$ all the elements which are in $B$.

   \begin{figure}[ht]
      \includegraphics{img/symdiff.pdf}
      \caption{The shaded area is $A \ominus B$}
      \label{fig:symmdif}
   \end{figure}

   It is perfectly clear that $A \ominus B = B \ominus A$; hence this 
   operation is commutative. It is also associative, as the accompanying
   pictorial representation suggests: Let the union of $A$, $B$, and
   $C$ be divided into seven regions as illustrated.

   \begin{figure}[ht]
      \includegraphics{img/symdiff3.pdf}
   \end{figure}

   \begin{gather*}
      A \ominus B \text{ consists of the regions 1, 4, 3, and 6.} \\
      B \ominus C \text{ consists of the regions 2, 3, 4, and 7.} \\
      A \ominus (B \ominus C) \text{ consists of the regions 1, 3, 5, and 7.}
      \\
      (A \ominus B) \ominus C \text{ consists of the regions 1, 3, 5, and 7.} 
   \end{gather*}

   \noindent Thus, $A \ominus (B \ominus C) = (A \ominus B) \ominus C$.


   If $D$ is a set, then the \emph{power set} of $D$ is the set $P_D$
   of all the subsets of $D$. That is,
   \begin{center}
   $$ P_D = \{ A : A \subseteq D \} $$
   \end{center}

   The operation $\ominus$ is to be regarded as an operation on $P_D$.

   \begin{enumerate}[1.]

   \item Prove that there is an identity element with respect to the operation
   $\ominus$. 

   \begin{proof}
   The identity element of $\ominus$ is $\emptyset$. 
   \begin{align*}
      A \ominus \emptyset & = (A - \emptyset) \cup (\emptyset - A) \\
                          & = A \cup \emptyset \\
			  & = A
   \end{align*}
   By inspection, it's obvious that $\ominus$ is commutative. Therefore,
   $\emptyset$ is the identity element.
   \end{proof}

   \item Prove every subset $A$ of $D$ has an inverse with respect to 
   $\ominus$, thus showing $\langle P_D,\ominus \rangle$ is a group!

   \begin{proof}
      The inverse of $A$ is $A$.
      \begin{align*}
         A \ominus A & = (A - A) \cup (A - A) \\
	             & = \emptyset \cup \emptyset \\
		     & = \emptyset
      \end{align*}
   \end{proof}

   \item Let $D$ be the three-element set $D = \{a,b,c\}$. List the elements
   of $P_D$. (For example, one element is $\{a\}$, another is $\{a,b\}$ and 
   so on. Do not forget the empty set and the whole set $D$.) Then
   write the operation table for $\langle P_D,\ominus \rangle$.

   \begin{center}
      $$ P_D = \{ \emptyset, \{a\}, \{b\}, \{c\}, \{a,b\}, \{b,c\}, 
                  \{a,c\}, \{a,b,c\} \} $$
   \end{center}

   \begin{center}
   \scalebox{.9}
   {\begin{tabular}{c|cccccccc}
      $\ominus$ & $\emptyset$ & $\{a\}$ & $\{b\}$ & $\{c\}$ & $\{a,b\}$ & 
          $\{b,c\}$ & $\{a,c\}$ & $\{a,b,c\}$  \\ \hline
      $\emptyset$ & $\emptyset$ & $\emptyset$ & $\emptyset$ & $\emptyset$ 
                  & $\emptyset$ & $\emptyset$ & $\emptyset$ & $\emptyset$\\
      $\{a\}$ & $\emptyset$ & $\emptyset$ & $\{a,b\}$ & $\{a,c\}$ & $\{b\}$ &
         $\{a,b,c\}$ & $\{c\}$ & $\{b,c\}$\\
      $\{b\}$ & $\emptyset$ & $\{a,b\}$ & $\emptyset$ & $\{b,c\}$ & $\{a\}$ &
         $\{c\}$ & $\{a,b,c\}$ & $\{a,c\}$ \\
      $\{c\}$ & $\emptyset$ & $\{a,c\}$ & $\{b,c\}$ & $\emptyset$ 
         & $\{a,b,c\}$ & $\{b\}$ & $\{a\}$ & $\{a,b\}$ \\
      $\{a,b\}$ & $\emptyset$ & $\{b\}$ & $\{a\}$ & $\{a,b,c\}$ & $\emptyset$ 
         & $\{a,c\}$ & $\{b,c\}$ & $\{c\}$ \\
      $\{a,c\}$ & $\emptyset$ & $\{c\}$ & $\{a,b,c\}$ & $\{a\}$ & $\{b,c\}$ 
         & $\{a,b\}$ & $\emptyset$ & $\{b\}$\\
      $\{b,c\}$ & $\emptyset$ & $\{a,b,c\}$ & $\{c\}$ & $\{b\}$ & $\{a,c\}$
         & $\emptyset$ & $\{a,b\}$ & $\{a\}$ \\
      $\{a,b,c\}$ & $\emptyset$ & $\{b,c\}$ & $\{a,c\}$ & $\{a,b\}$ & $\{c\}$ 
         & $\{a\}$ & $\{b\}$ & $\emptyset$\\
   \end{tabular}}
   \end{center}
   
   \begin{verbatim}

   \end{verbatim}

   \end{enumerate}

   \item \textsc{A Checkerboard Game}
      \begin{figure}[ht]
      \includegraphics{img/checker.pdf}
   \end{figure}

   \noindent Our checkerboard has only four squares, numbered 1, 2, 3, and 4.
   There is a single checker on the board, and it has four possible moves:

   \begin{enumerate}
      \item[V:] Move vertically; that is, move from 1 to 3, or from 3 to 1, or
      from 2 to 4, or from 4 to 2. 
      \item[H:] Move horizontally; that is, move from 1 to 2 or vice versa,
      or from 3 to 4 or vice versa.
      \item[D:] Move diagonally; that is, move from 2 to 3 or vice versa, or
      move from 1 to 4 or vice versa.
      \item[I:] Stay put.
   \end{enumerate}

   \noindent We may consider an operation on the set of these four moves, which
   consists of performing moves successively. For example, if we move
   horizontally and then vertically, we end up with the same result as if we
   had moved diagonally:

   \begin{gather*}
      H * V = D
   \end{gather*}

   \noindent If we perform two horizontal moves in succession, we end up where
   we started: $H*H=I$. And so on. If $G=\{V,D,H,I\}$, and $*$ is the operation
   we have just described, write the table of $G$.

   \begin{center}
   \begin{tabular}{c|cccc}
      $*$ & $I$ & $V$ & $H$ & $D$ \\ \hline
      $I$ & $I$ & $V$ & $H$ & $D$ \\
      $V$ & $V$ & $I$ & $D$ & $H$ \\
      $H$ & $H$ & $D$ & $I$ & $V$ \\
      $D$ & $D$ & $H$ & $V$ & $I$
   \end{tabular}
   \end{center}
   
   \noindent Granting associativity, explain why $\langle G,* \rangle$ is a
   group.

   \emph{Explanation} $\langle G,* \rangle$ is a group because
   it has an identity element, $I$, and has an inverse for each element.
   We can see that for every element $M \in G$, $M * I = I * M = M$.
   Also, for every element $M$ we have an inverse $M^{-1} = M$.

   

   \item \textsc{A Coin Game}
   \begin{figure}[ht]
      \includegraphics{img/coingame.pdf}
   \end{figure}

   Imagine two coins on a table, at positions $A$ and $B$. In this game there
   are eight possible moves:

   \begin{table}[ht]
   \begin{tabular}{rlcrl}
    $M_1$:&Flip over the coin at $A$. &\verb=   =& $M_5$:&Flip coin at $A$; then switch.\\
    $M_2$:&Flip over the coin at $B$. &\verb=   =& $M_6$:&Flip coin at $B$; then switch.\\
    $M_3$:&Flip over both coins. &\verb=   =& $M_7$:&Flip both coins; then switch.\\
    $M_4$:&Switch the coins.     &\verb=   =& $I$:&Do not change anything.
   \end{tabular}
   \end{table}


   We may consider an operation on the set $\{I, M_1,\ldots,M_7\}$, which
   consists of performing any two moves in succession.  For example, if we
   switch coins, then flip over the coin at $A$, this is the same as first
   flipping over the coin at $B$ then switching:

   $$ M_4 * M_1 = M_2 * M_4 = M_6 $$

   If $G = \{I,M_1,\ldots,M_7\}$ and $*$ is the operation we have just
   described, write the table of $\langle G,* \rangle$.

   \begin{table}[ht]
   \begin{tabular}{c|cccccccc}
     $*$ & $I$ & $M_1$ & $M_2$ & $M_3$ & $M_4$ & $M_5$ & $M_6$ & $M_7$ \\ \hline
     $I$ & $I$ & $M_1$ & $M_2$ & $M_3$ & $M_4$ & $M_5$ & $M_6$ & $M_7$\\
     $M_1$ & $M_1$ & $I$ & $M_3$ & $M_2$ & $M_5$ & $M_4$ & $M_7$ & $M_6$\\
     $M_2$ & $M_2$ & $M_3$ & $I$ & $M_1$ & $M_6$ & $M_7$ & $M_4$ & $M_5$\\
     $M_3$ & $M_3$ & $M_2$ & $M_1$ & $I$ & $M_7$ & $M_6$ & $M_5$ & $M_4$\\
     $M_4$ & $M_4$ & $M_6$ & $M_5$ & $M_7$ & $I$ & $M_2$ & $M_1$ & $M_3$\\
     $M_5$ & $M_5$ & $M_7$ & $M_4$ & $M_6$ & $M_1$ & $M_3$ & $I$ & $M_2$\\
     $M_6$ & $M_6$ & $M_4$ & $M_7$ & $M_5$ & $M_2$ & $I$ & $M_3$ & $M_1$\\
     $M_7$ & $M_7$ & $M_5$ & $M_6$ & $M_4$ & $M_3$ & $M_1$ & $M_2$ & $I$\\
   \end{tabular}
   \end{table}

   Granting associativity, explain why $\langle G,* \rangle$ is a group.
   Is it commutative? If not, show why not.

   \noindent \textsc{Solution} It is a group because $I$ is an identity 
   element, and we can see from the table that every element has an
   inverse. However, the operation is not commutative. There are at least
   a dozen counter-examples, here is but one: $M_1*M_7=M_6\ne M_5=M7*M1$.

   \item \textsc{Groups in Binary Codes}
   \noindent The most basic way of transmitting information is to code it into
   strings of 0s and 1s, such as 0010111, 1010011, etc. Such strings are called
   \emph{binary words}, and the number of 0s and 1s in any binary word is
   called its \emph{length}. All information may be coded in this fashion.

   When information is transmitted, it is sometimes received incorrectly.
   One of the most important purposes of coding theory is to find ways of
   \emph{detecting errors}, and \emph{correcting} errors of transimission.

   If a word $\mathbf{a}=a_1 a_2 \cdots a_n$ is sent, but a word
   $\mathbf{b}=b_1 b_2 \cdots b_n$ is received (where the $a_i$ and $b_i$
   are 0s or 1s), then the \emph{error pattern} is the word
   $\mathbf{e}=e_1 e_2 \cdots e_n$ where
%
   \begin{gather*}
      e_i =
         \begin{cases}
	    0, & \text{if $a_i = b_i$} \\
	    1, & \text{if $1_i \ne b_i$}
	 \end{cases}
   \end{gather*}

   \noindent With this motivation, we define an operation of \emph{adding}
   words, as follows: If $\mathbf{a}$ and $\mathbf{b}$ are both of length $l$,
   we add them according to the rules
   %
   \begin{align*}
      0 + 0 &= 0   &   1 + 1 & = 0   &   0 + 1 & = 1   &   1 + 1 & = 0
   \end{align*}

   \newcommand{\mbf}[1]{\ensuremath{\mathbf{#1}}}
   \noindent If \mbf{a} and \mbf{b} are both of length \emph{n}, we add them by
   \emph{adding corresponding digits}. That is (let us introduce commas for
   convenience),
   %
   \begin{gather*}
     (a_1,a_2,\ldots,a_n) + (b_1,b_2,\ldots,b_n) = (a_1+b_1, a_2+b_2,
     \ldots,a_n+b_n)
   \end{gather*}

   \noindent Thus, the sum of \mbf{a} and \mbf{b} is the error pattern 
   \mbf{e}.

   For example,
   %
   \begin{table}[ht]
      \begin{tabular}{cc}
         \begin{tabular}{r}
	    0010110 \\
	   +0011010 \\ \hline
	   =0001100
	 \end{tabular} & 
	 \begin{tabular}{r}
	    10100111 \\
	   +11110111 \\ \hline
	   =01010000
	 \end{tabular}
      \end{tabular}
   \end{table}

   The symbol $\mathbb{B}^n$ will designate the set of all the binary words
   of length $n$. We will prove that the operation of word addition has the
   following properties on $\mathbb{B}^n$: 
   
   \begin{enumerate}[1.]
      \item It is commutative.
      \item It is associative.
      \item There is an identity element for word addition.
      \item Every word has an inverse under word addition.
   \end{enumerate}

   First, we verify the commutative law for words of length 1:
   \begin{gather*}
      0+1=1=1=1+1
   \end{gather*}

   \noindent \textbf{1} Show that $(a_1,a_2,\ldots,a_n) + (b_1,b_2,\ldots,b_n)
   = (b_1,b_2,\ldots,b_n) + (a_1,a_2,\ldots,a_n)$.
   
   \begin{proof}
   This is easy to show by using the commutativity property of words
   of length 1.
      \begin{align*}
              & \Blank (a_1,a_2,\ldots,a_n) + (b_1,b_2,\ldots,b_n) \\
	      & = (a_1+b_1,a_2+b_2,\ldots,a_n+b_n) \\
	      & = (b_1+a_1,b_2+a_2,\ldots,b_n+a_n) \\
	      & = (b_1,b_2,\ldots,b_n) + (a_1,a_2,\ldots,a_n) \qedhere
      \end{align*}
   \end{proof}

   \noindent \textbf{2} To verify the associative law, we first verify
   it for words of length 1:
   \begin{gather*}
      1 + (1 + 1) = 1 + 0 = 1 = 0 + 1 = (1 + 1) + 1 \\
      1 + (1 + 0) = 1 + 1 = 0 = 0 + 0 = (1 + 1) + 0
   \end{gather*}

   \noindent Check the remaining six cases.
   \begin{proof}
      \begin{gather*}
	 0 + (0 + 0) = 0 + 0 = 0 = 0 + 0 = (0 + 0) + 0 \\
	 0 + (0 + 1) = 0 + 1 = 1 = 0 + 1 = (0 + 0) + 1 \\
	 0 + (1 + 0) = 0 + 1 = 1 = 1 + 0 = (0 + 1) + 0 \\
	 0 + (1 + 1) = 0 + 0 = 0 = 1 + 1 = (0 + 1) + 1 \\
         1 + (0 + 0) = 1 + 0 = 1 = 1 + 0 = (1 + 0) + 0 \\
	 1 + (0 + 1) = 1 + 1 = 0 = 1 + 1 = (1 + 0) + 1 \qedhere
      \end{gather*}
   \end{proof}

   \noindent \textbf{3} Show that $(a_1,\ldots,a_n)+[(b_1,\ldots,b_n)+
   (c_1,\ldots,c_n)]=[(a_1,\ldots,a_n)+(b_1,\ldots,b_n)]+(c_1,\ldots,c_n)$.
   \begin{proof}
      \begin{align*}
         & \Blank (a_1,\ldots,a_n)+[(b_1,\ldots,b_n)+(c_1,\ldots,c_n)] \\
	 & = (a_1,\ldots,a_n)+[(b_1+c_1),\ldots,(b_n+c_n)] \\
	 & = [a_1+(b_1+c_1),\ldots,a_n+(b_n+c_n)] \\
	 & = [(a_1+b_1)+c_1,\ldots,(a_n+b_n)+c_n] \\
	 & = [(a_1+b_1),\ldots,(a_n+b_n)]+(c_1,\ldots,c_n)\\
	 & = [(a_1,\ldots,a_n)+(b_1,\ldots,b_n)]+(c_1,\ldots,c_n) \qedhere
      \end{align*}
   \end{proof}

   \noindent \textbf{4} The identity element of $\mathbb{B}^n$, that is,
   the identity element for adding words of length $n$, is: 
   $\mbf{z} = z_1 z_2 \cdots z_n$ where $z_i = 0$.

   \noindent \textbf{5} The inverse, with respect to word addition, of any
   word $(a_1,\ldots,a_n)$ is: itself.

   \noindent \textbf{6} Show that $a+b=a-b$ [where $a-b=a+(-b)$]. (Well, 
   unfortunately $(-b)$ is not defined. I will assume it means 
   $(-b_1,\ldots,-b_n)$ where $-0 = 0$ and $-1 = -1$.
   \begin{proof}
      First we show it to be true for words of length 1.
      \begin{gather*}
         0 + 0 = 0 = 0 + (-0) = 0 - 0 \\
	 0 + 1 = 1 = 0 + (-1) = 0 - 1 \\
	 1 + 0 = 1 = 1 + (-0) = 1 - 0 \\
	 1 + 1 = 0 = 1 + (-1) = 1 - 1 
      \end{gather*}

      Now we will show for words of length $n$.
      \begin{align*}
         & \Blank (a_1,\ldots, a_n) + (b_1,\ldots,b_n) \\
	 & = (a_1+b_1,\ldots,a_n+b_n) \\
	 & = (a_1-b_1,\ldots,a_n-b_n) \\
	 & = (a_1,\ldots,a_n) + (-b_1,\ldots,-b_n) \\
	 & = (a_1,\ldots,a_n) - (b_1,\ldots,b_n) \qedhere
      \end{align*}
   \end{proof}

   \noindent \textbf{7} If $a+b=c$, show that $a=b+c$.

   The proof is shown in figure~\ref{fig:f7proof}.
   
   \begin{figure}
      \caption{The proof for F7: If $a+b=c$ then $a=b+c$.}
      \label{fig:f7proof}
      \begin{align*}
         a + b & = c \\
	 a + (b + b) & = c + b \\
	 a + z & = c + b \\
	 a & = b + c \qed
      \end{align*}
   \end{figure}

   \item \textsc{Theory of Coding: Maximum Likelihood Decoding}
   See text pp.\ 34,35 for introductory material.

   Let $C_1$ contain the codewords listed in table~\ref{tab:c1codewords}.
   
   \begin{table}
   \caption{The codewords for $C_1$ of section G.}
   \label{tab:c1codewords}
   \begin{tabular}{ccc} 
      \mbf{c} & $c_1 + c_3$ & $c_1 + c_2 + c_3$ \\ \hline
      00000  & 0 & 0 \\
      00111  & 1 & 1 \\
      01001  & 0 & 1 \\
      01110  & 1 & 0 \\
      10011  & 1 & 1 \\
      10100  & 0 & 0 \\
      11010  & 1 & 0 \\
      11101  & 0 & 1 \\
   \end{tabular}
   \end{table}

      \noindent 1. Verify that every codeword $a_1 a_2 a_3 a_4 a_5$ in
      $C_1$ satisifies the following two parity-check equations:
      $a_4=a_1+a_3$; $a_5=a_1+a_2+a_3$. 

      This can be verified by inspecting table~\ref{tab:c1codewords} where
      the values for the equations are shown for each codeword along
      with the codeword.


      
      \noindent 2. Let $C_2$ be the following code in $\mathbb{B}^6$. The
      first three positions are the information positions, and every codeword 
      $a_1 a_2 a_3 a_4 a_5 a_6$ satisifies the parity-check equations
      $a_4 = a_2$, $a_5 = a_1 + a_2$ and $a_6 = a_1 + a_2 + a_3$.

      \begin{enumerate}[(a)]
         \item List the codewords of $C_2$.


	 \Solution The codewords are as follows:
	 \begin{table}[ht]
	    \begin{tabular}{l}
	       000000 \\
	       001001 \\
	       010111 \\
	       011110 \\
	       100011 \\
	       101010 \\
	       110100 \\
	       111101
	    \end{tabular}
	 \end{table}

         \item Find the minimum distance of the code $C_2$.

	 \noindent \Solution The minimum distance is 2.
      \end{enumerate}
      
      \noindent 3. Design a code in $\mathbb{B}^4$ where the first
      two positions are information positions. Give the parity-check
      equations, list the codewords, and find the minimum distance.

      \noindent \Solution My code uses the equations $a_3=a_2$ and
      $a_4 = a_1 + a_2$. The code words are listed below. The minimum
      distance is 2.
      \begin{table}[ht]
      \begin{tabular}{l}
        0000 \\
	0111 \\
	1001 \\
	1110
      \end{tabular}
      \end{table}

      If \mbf{a} and \mbf{b} are any two words, let $d(\mbf{a},\mbf{b})$
      denote the distance between \mbf{a} and \mbf{b}. To \emph{decode}
      a received word \mbf{x} (which may contain errors of transmission) means
      to find the codeword closes to \mbf{x}, that is, the codeword \mbf{a}
      such that $d(\mbf{a},\mbf{x})$ is a minimum. This is called
      \emph{maximum-likelihood decoding}.
\begin{verbatim}
\end{verbatim}
      \noindent 4. Decode the following words in $C_1$: 11111, 00101,
      11000, 10011, 10001, and 10111.

      \noindent \Solution The decoded words are 11101, 00111, 11010, 10011,
      10011, and (10011 or 00111).

      You may have noticed that the last two words in part 4 had ambigous
      decodings [welch: I can't find a second decoding for the second to last
      codeword in part 4]: for example, 10111 may be decoded as either
      10011 or 00111. This situation is clearly unsatisfactory. We shall see
      next what conditions will ensure that every word can be decoded into
      only \emph{one} possible codeword.

      In the remaining exercises, let $C$ be a code in $\mathbb{B}^n$,
      let $m$ denote the minimum distance in $C$, and let 
      $\mathbf{a}$ and $\mathbf{b}$ denote codewords in $C$.

      \noindent 5. Prove that it is possible to detect up to $m-1$ errors.
      (That is, if there are errors of transmission in $m-1$ or 
      fewer positions of a codeword, it can always be determined
      that the received word is incorrect.)

      \begin{proof}
	 Let $w$ be the sent word and $w'$ be the received word. Let 
	 $n$ be the number of errors in $w'$ such that $0 < n <= m - 1$. 
	 Assume that $w'$ is not determined to be incorrect. This means
	 that it was accepted as a codeword. However, the distance
	 between $w$ and $w'$ is $n$ and $n < m$. Therefore, the 
	 minimum distance of $C$ is $n$. But this contradicts
	 the definition of our code that states that $m$ is the minimum
	 distance. Therefore, our assumption is proved incorect, and
	 the word $w'$ will be detected to have errors.
      \end{proof}

      \noindent 6. By the \emph{sphere of radius} $k$ about a codeword
      $\mathbf{a}$ we mean the set of all words in $\mathbb{B}^n$
      whose distance from $\mathbf{a}$ is no greater than $k$. This set
      is denoted by $S_k(\mathbf{a})$; hence 
      \begin{center}
      $$ S_k(\mathbf{a}) = \{\mathbf{x} : d(\mathbf{a},\mathbf{x}) \le k \}$$
      \end{center}

      If $t=\frac{1}{2}(m-1)$, prove that any two spheres of radius $t$,
      say $S_t(\mathbf{a})$ and $S_t(\mathbf{b})$, have no elements in
      common. [\textsc{Hint}: Assume there is a word $\mathbf{x}$ such
      that $\mathbf{x} \in S_t(\mathbf{a})$ and $\mathbf{x}\in S_t(
      \mathbf{b})$. Using the definitions of $t$ and $m$, show that
      this is impossible.]

      \begin{proof}
         Assume there is a word $\mathbf{x}$ such that $\mathbf{x} \in
         S_t(\mathbf{a})$ and $\mathbf{x}\in S_t( \mathbf{b})$. 
	 This means that we need to flip at most $\frac{1}{2}(m-1)$ bits
	 to transform $\mathbf{a}$ into $\mathbf{x}$, and at most
	 $\frac{1}{2}(m-1)$ bits to tranform $\mathbf{x}$ into $\mathbf{b}$.
	 This implies that we need to flip at most 
	 $\frac{1}{2}(m-1) \times 2 = (m-1)$ bits to transform
	 $\mathbf{a}$ into $\mathbf{b}$. However, we know that the 
	 minimum distance between any two codewords is $m$. Therefore,
	 our assumption that $\mathbf{x}$ exists is false and there is
	 no common element between $S_t(\mathbf{a})$ and $S_t(\mathbf{b})$.
      \end{proof}

      \noindent 7. Deduce from part 6 that if there are $t$ or fewer errors
      of transmission in a codeword, the received word will be decoded 
      correctly.

      \begin{proof}
      By design we have a sphere of radius $t$ around every codeword.
      We know from part 6 that no two spheres have elements in common.
      If there are $t$ or fewer errors in a transmission, then the
      received word will be in the sphere surrounding the sent word and
      no other sphere. Therefore, we can correctly decode the received
      word to be the codeword in the sphere.
      \end{proof}

      \noindent 8. Let $C_2$ be the code described in part 2. Using the
      results of parts 5 and 7, explain why two errors in any codeword
      can always be detected, and why one error in any
      codeword can always be corrected.

      \noindent \Solution I claim that the author is incorrect in his claims.
      It is readily apparent that the minimum distance of $C_2$ is 2. This
      can be seen by looking at the first two elements 000000 and 001001.
      Since two errors in transmission can change 000000 into 001001, the claim
      that 2 errors can always be detected is false. Likewise, one error
      in transmission can transform 000000 into 001000. This error can
      be detected but it cannot be corrected. Therefore, both claims are
      false.

      Now, let's assume for the sake of this exercise that the minimum
      distance of $C_2$ is 3. Then the author's claims follow directly
      from parts 5-7. If there are 2 errors or less they will always 
      be detected because by part 5 it's impossible for the received
      word to be a codeword. If there is 1 error or less then the received
      word will fall within a sphere of radius of $t=1/2(3-1)=1$ and therefore
      be readily correctable (by parts 6 and 7).
     

\end{enumerate}
\end{document}
