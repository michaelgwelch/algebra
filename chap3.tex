\documentclass[twoside]{amsart}
\usepackage{amssymb,latexsym}
\usepackage{xspace}
\usepackage{enumerate}
\usepackage{graphics}
\newcommand{\Rationals}{\mathbb{Q}{}}
\newcommand{\Reals}{\mathbb{R}{}}
\newcommand{\Integers}{\mathbb{Z}{}}
\newcommand{\Solution}{\textsc{Solution}\xspace}
\newcommand{\Problem}{\textsc{Problem}\xspace}
\newcommand{\Blank}{\mathrel{\phantom{=}}}
\newcommand{\ltrue}{\top}
\newcommand{\lfalse}{\bot}
\begin{document}
\title{Answers to Chapter 3 Exercises - A Book of Abstract Algebra}
\author{Michael Welch}
\date{\today}
\maketitle

This document contains selected answers to exercises from chapter 3
of A Book of Abstract Algebra.

\begin{enumerate}[A.]
   \item \textsc{Examples of Abelian Groups}

   Prove that each of the following sets, with the indicated operation, is an
   abelian group.

   \textbf{Instructions} Proceed as in Chapter 2, Exercise B.

   \begin{enumerate}[1.]
      \item $x * y = x + y + k$ (k is a fixed constant), on the set 
      $\Reals$ of the real numbers.

      \begin{enumerate} % 4 parts: assoc, commut, identity, inverse
	 \item 
	 \begin{proof}[Proof that operation is commutative]
	    \begin{align*}
	       x * y & = x + y + k \\
	       y * x & = y + x + k \\
		     & = x + y + k && \text{Commute addition} \qedhere
	    \end{align*}
	 \end{proof}

	 \item 
	 \begin{proof}[Proof that operation is associative]
	    \begin{align*}
	       x * (y * z) & = x * (y + z + k) \\
			   & = x + (y + z + k) + k \\
			   & = x + y + z + 2k \\
	       (x * y) * z & = (x + y + k) * z \\
			   & = (x + y + k) + z + k \\
			   & = x + y + z + 2k \qedhere
	    \end{align*}
	 \end{proof}

	 \item The identity element can be calculated:
	 \begin{align*}
	    x * e & = x         \\
	    x * e & = x + e + k \\
		x & = x + e + k \\
		e & = -k
	 \end{align*}

	 We know by commuativity that $x * (-k) = (-k) * x$. Therefore,
	 $-k$ is the identity element.

	 \item We can caluclate the inverse of $x$ which we label $x'$.
	 \begin{align*}
	    x * x' & = -k          \\
	    x * x' & = x + x' + k  \\
	       -k  & = x + x' + k  \\
		x' & = -x - 2k
	 \end{align*}

      \end{enumerate}

      \item $\displaystyle x * y = \frac{xy}{2}$, on the set $\{x \in \Reals
      : x \ne 0\}$
      
	 \begin{enumerate}
	    \item
	    \begin{proof}[Proof the operation is commutative]
	       \begin{align*}
		  x * y & = \frac{xy}{2}         \\
		  y * x & = \frac{yx}{2}         \\
			& = \frac{xy}{2} \qedhere
	       \end{align*}
	    \end{proof}

	    \item
	    \begin{proof}[Proof the operation is associative]
	       \begin{align*}
		  x * (y * z) & = x * \frac{yz}{2}   \\
			      & = \frac{x\displaystyle\frac{yz}{2}}{2} \\
			      & = \frac{xyz}{4} \\
		  (x * y) * z & = \frac{xy}{2} * z \\
			      & = \frac{\displaystyle \frac{xy}{2}z}{2} \\
			      & = \frac{xyz}{4} \qedhere
	       \end{align*}
	    \end{proof}

	    \item The following calculation shows that the identity element
	    is 2.
	    \begin{align*}
	       x * e & = x              \\
	       x * e & = \frac{xe}{2}   \\
		   x & = \frac{xe}{2}   \\
		   1 & = \frac{e}{2}    \\
		   e & = 2
	    \end{align*}

	    \item The following calculation shows that the inverse of
	    $x$ is $4/x$.
	    \begin{align*}
	       x * x' & = 2             \\
	       x * x' & = \frac{xx'}{2} \\
		   2  & = \frac{xx'}{2} \\
		   4  & = xx'           \\
		   x' & = 4/x    && \text{$x \ne 0$ is given}
	    \end{align*}
	 \end{enumerate}

	 \item $x * y = x + y + xy$, on the set $\{x \in \Reals : x \ne -1\}$
	 \begin{enumerate}
	    \item
	    \begin{proof}[Proof the operation is commutative]
	       \begin{align*}
		  x * y & = x + y + xy \\
		  y * x & = y + x + yx \\
			& = x + y + xy \qedhere
	       \end{align*}
	    \end{proof}

	    \item
	    \begin{proof}[Proof the operation is associative]
	       \begin{align*}
		  x * (y * z) & = x * (y + z + yz) \\
			      & = x + (y + z + yz) + x(y + z + yz) \\
			      & = x + y + z + yz + xy + xz + xyz\\
			      & = x + y + xy + xz + yz + xyz \\
		  (x * y) * z & = (x + y + xy) * z \\
			      & = (x + y + xy) + z + (x + y + xy)z \\
			      & = x + y + xy + z + xz + yz + xyz \\
			      & = x + y + xy + xz + yz + xyz \qedhere
	       \end{align*}
	    \end{proof}

	    \item The following calculation shows the identity element to
	    be 0.
	    \begin{align*}
	       x * e & = x           \\
	       x * e & = x + e + xe  \\
		   x & = x + e + xe  \\
		   0 & =     e + xe  \\
		   0 & =   e(1 + x)  \\
		   e & = 0 && \text{$x\ne-1$ given}
	    \end{align*}

	    \item The following calculation shows the inverse of $x$ is 
	    $\displaystyle \frac{-x}{1+x}$.
	    \begin{align*}
	       x * x' & = 0            \\
	       x * x' & = x + x' + xx' \\
		    0 & = x + x' + xx' \\
		    0 & = x + x'(1 + x) \\
		   -x & = x'(1+x) \\
	       x'(1+x)& = -x \\
		   x' & = \frac{-x}{1+x} && \text{$x\ne-1$ given}
	    \end{align*}

	 \end{enumerate}

	 \item $x * y = \displaystyle \frac{x + y}{xy + 1}$, on the set
	 $\{x \in \Reals : -1 < x < 1\}$.
	 \begin{enumerate}
	    \item
	    \begin{proof}[Proof the operation is commutative]
	       \begin{align*}
		  x * y & = \frac{x+y}{xy + 1} \\
		  y * x & = \frac{y+x}{yx + 1} \\
			& = \frac{x+y}{xy + 1} \qedhere
	       \end{align*}
	    \end{proof}

	    \item
	    \begin{proof}[Proof the operation is associative]
	       \begin{align*}
		  x * (y * z) & = x * \frac{y + z}{yz + 1} \\
			      & = \frac{x + \displaystyle \frac{y + z}{yz + 1}}
				  {x\displaystyle \frac{y + z}{yz + 1} + 1} \\
			      & = \frac{x(yz+1)+y+z}{xy+xz+yz+1} \\
			      & = \frac{x + y + z + xyz}{xy + xz + yz + 1}
				  \qedhere
	       \end{align*}
	    \end{proof}

	    \item The following caluclation shows the identity element to be 0.
	    \begin{align*}
	       x * e & = x \\
	       x * e & = \frac{x + e}{xe + 1} \\
		   x & = \frac{x + e}{xe + 1} \\
	       x + e & = x(xe + 1) \\
	       x + e & = x^2e + x \\
	       e - x^2e & = 0  \\
	       e(1-x^2) & = 0    \\
		   e & = 0  && \text{$-1 < x < 1$ is given}
	    \end{align*}

	    \item The following caluclation shows the inverse of $x$ to be
	    $-x$.
	    \begin{align*}
	       x * x' & = 0                      \\
	       x * x' & = \frac{x + x'}{xx' + 1} \\
		   0  & = \frac{x + x'}{xx' + 1} \\
		   0  & = x + x' \\
		   x' & = -x
	    \end{align*}
	 \end{enumerate}
   \end{enumerate}

   \item \textsc{Groups on the Set $\Reals \times \Reals$}

   The symbol $\Reals \times \Reals$ represents the set of all ordered pairs
   $(x,y)$ of real numbers.  $\Reals \times \Reals$ may therefore be identified
   with the set of all the points in the plane.  Which of the following subsets
   of $\Reals \times \Reals$, with the indicated operation, is a group?  Which
   is an abelian group?

   \textbf{Instructions} Proceed as in the preceding exercise. To find the
   identity element, which in these problems is an ordered pair $(e_1, e_2)$ of
   real numbers, solve the equation $(a,b) * (e_1, e_2) = (a,b)$ for $e_1$ and
   $e_2$. To find the inverse $(a',b')$ of $(a,b)$, solve the equation
   $(a,b)*(a',b')=(e_1,e_2)$ for $a'$ and $b'$. [Remember that $(x,y)=(x',y')$
   if and only if $x=x'$ and $y=y'$.]

      \begin{enumerate}[1.]
	 \item $(a,b)*(c,d)=(ad+bc,bd)$, on the set $\{(x,y) \in \Reals
	 \times \Reals : y \ne 0\}$.
	 \begin{enumerate}
	    \item The operation is commutative.
	    \begin{proof}
	       \begin{align*}
		  (a,b)*(c,d) & = (ad+bc,bd) \\
		  (c,d)*(a,b) & = (cb+da,db) \\
			      & = (ad+bc,bd) \qedhere
	       \end{align*}
	    \end{proof}

	    \item The operation is associative.
	    \begin{proof}
	       \begin{align*}
		  (a,b) * ((c,d) * (e,f)) & = (a,b) * (cf+de,df)  \\
					  & = (adf + b(cf+de), bdf) \\
					  & = (adf +bcf + bde, bdf) \\
		  ((a,b) * (c,d)) * (e,f) & = (ad+bc,bd) * (e,f)    \\
					  & = ((ad+bc)f+bde, bdf)   \\
					  & = (adf+bcf+bde,bdf) \qedhere
	       \end{align*}
	    \end{proof}

	    \item 
	    \begin{align*}
	       (a,b) * (e_1,e_2) & = (a,b)            \\
	       (a,b) * (e_1,e_2) & = (ae_2+be_1,be_2) \\
		       (a,b)     & = (ae_2+be_1,be_2) 
	    \end{align*}
	    This implies that $a=ae_2+be_1$ and $b=be_2$. So $e_2=1$
	    and $e_1=0$. So the identity element is $(0,1)$.

	    \item
	    \begin{align*}
	       (a,b) * (a',b') & = (0,1) \\
	       (a,b) * (a',b') & = (ab'+ba',bb') \\
		 (ab'+ba',bb') & = (0,1) \\
	    \end{align*}
	    This implies $ab'+ba'=0$ and $bb'=1$. So $b'=1/b$. Let's
	    solve for $a$:
	    \begin{align*}
	       ab'+ba' & = 0         \\
	       a(1/b) + ba' & = 0    \\
		    ba' & = -1(a/b)  \\
		    a' & = -a/b^2
	    \end{align*}
	    So $(a',b') = (-a/b^2,1/b)$.

	 \end{enumerate}
      \end{enumerate}

   \item \textsc{Groups of Subsets of a Set}

   If $A$ and $B$ are any two sets, their \emph{symmetric difference}
   is the set $A \ominus B$ defined as follows (see figure~\ref{fig:symmdif}):
   \begin{center}
   $$ A\ominus B = (A-B) \cup (B-A) $$
   \end{center}

   \textsc{Note}: $A - B$ represents the set obtained by removing from
   $A$ all the elements which are in $B$.

   \begin{figure}[ht]
      \includegraphics{img/symmetric_diff.png}
      \caption{The colored area is $A \ominus B$}
      \label{fig:symmdif}
   \end{figure}

   If $D$ is a set, then the \emph{power set} of $D$ is the set $P_D$
   of all the subsets of $D$. That is,
   \begin{center}
   $$ P_D = \{ A : A \subseteq D \} $$
   \end{center}

   The operation $\ominus$ is to be regarded as an operation on $P_D$.

   \begin{enumerate}[1.]

   \item Prove that there is an identity element with respect to the operation
   $\ominus$, thus showing $\langle P_D,\ominus \rangle$ is a group!

   \begin{proof}
   The identity element of $\ominus$ is $\emptyset$. 
   \begin{align*}
      A \ominus \emptyset & = (A - \emptyset) \cup (\emptyset - A) \\
                          & = A \cup \emptyset \\
			  & = A
   \end{align*}
   By inspection, it's obvious that $\ominus$ is commutative. Therefore,
   $\emptyset$ is the identity element.
   \end{proof}

   \end{enumerate}

   \item \textsc{A Checkerboard Game}

   \item \textsc{A Coin Game}

   \item \textsc{Groups in Binary Codes}

   \item \textsc{Theory of Coding: Maximum Likelihood Decoding}

      \noindent 1.Verify
      
      \noindent 2. Let
      
      \noindent 3. Design
      
      \noindent 4. Decode

      In the remaining exercises, let $C$ be a code in $\mathbb{B}^n$,
      let $m$ denote the minimum distance in $C$, and let 
      $\mathbf{a}$ and $\mathbf{b}$ denote codewords in $C$.

      \noindent 5. Prove that it is possible to detect up to $m-1$ errors.
      (That is, if there are errors of transmission in $m-1$ or 
      fewer positions of a codeword, it can always be determined
      that the received word is incorrect.)

      \begin{proof}
	 Let $w$ be the sent word and $w'$ be the received word. Let 
	 $n$ be the number of errors in $w'$ such that $0 < n <= m - 1$. 
	 Assume that $w'$ is not determined to be incorrect. This means
	 that it was accepted as a codeword. However, the distance
	 between $w$ and $w'$ is $n$ and $n < m$. Therefore, the 
	 minimum distance of $C$ is $n$. But this contradicts
	 the definition of our code that states that $m$ is the minimum
	 distance. Therefore, our assumption is proved incorect, and
	 the word $w'$ will be detected to have errors.
      \end{proof}

      \noindent 6. By the \emph{sphere of radius} $k$ about a codeword
      $\mathbf{a}$ we mean the set of all words in $\mathbb{B}^n$
      whose distance from $\mathbf{a}$ is no greater than $k$. This set
      is denoted by $S_k(\mathbf{a})$; hence 
      \begin{center}
      $$ S_k(\mathbf{a}) = \{\mathbf{x} : d(\mathbf{a},\mathbf{x}) \le k \}$$
      \end{center}

      If $t=\frac{1}{2}(m-1)$, prove that any two spheres of radius $t$,
      say $S_t(\mathbf{a})$ and $S_t(\mathbf{b})$, have no elements in
      common. [\textsc{Hint}: Assume there is a word $\mathbf{x}$ such
      that $\mathbf{x} \in S_t(\mathbf{a})$ and $\mathbf{x}\in S_t(
      \mathbf{b})$. Using the definitions of $t$ and $m$, show that
      this is impossible.]

      \begin{proof}
      Assume there is a word $\mathbf{x}$ such that $\mathbf{x} \in
      S_t(\mathbf{a})$ and $\mathbf{x}\in S_t( \mathbf{b})$. 
      \end{proof}

\end{enumerate}
\end{document}
