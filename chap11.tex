\documentclass[twoside]{amsart}
\usepackage{amssymb,latexsym}
\usepackage{amsfonts}
\usepackage{xspace}
\usepackage{enumerate}
\usepackage{graphics}
\usepackage{fitch}
\newcommand{\Rationals}{\mathbb{Q}{}}
\newcommand{\Reals}{\ensuremath{\mathbb{R}}\xspace}
\newcommand{\Integers}{\ensuremath{\mathbb{Z}{}}\xspace}
\newcommand{\solution}{\textsc{Solution}\xspace}
\newcommand{\problem}{\textsc{Problem}\xspace}
\newcommand{\Blank}{\mathrel{\phantom{=}}}
\newcommand{\ltrue}{\top}
\newcommand{\lfalse}{\bot}
\newcommand{\fOfg}{\ensuremath{f \circ g}\xspace}
\newcommand{\gOff}{\ensuremath{g \circ f}\xspace}
\newcommand{\eps}{\ensuremath{\epsilon}\xspace}
\newcommand{\iso}{\cong}
\newcommand{\niso}{\ncong}
\newcommand{\blank}{\vspace{5pt}}
\newcommand{\ind}{\hspace{.35in}}
\newcommand{\degree}{\ensuremath{^\circ}}
\newcommand{\real}{\mathop{\mathrm{real}}}
\newcommand{\img}{\mathop{\mathrm{img}}}
\newcommand{\first}{\mathop{\mathrm{first}}}
\newcommand{\second}{\mathop{\mathrm{second}}}
\newcommand{\abs}{\mathop{\mathrm{abs}}}
%\renewcommand{\qed}{\blacksquare}
\newcommand{\itm}{\blank\item}
\newcommand{\sol}{\blank\noindent\solution}
\newcommand{\ord}{\mathop{\mathrm{ord}}}
\begin{document}
\title{Answers to Chapter 11 Exercises - A Book of Abstract Algebra}
\author{Michael Welch}
\date{\today}
\maketitle

This document contains selected answers to exercises from chapter 11
of A Book of Abstract Algebra.


\begin{enumerate}[A.]
   
   \item \textsc{Examples of Cyclic Groups}

   \begin{enumerate}[1)]
      \itm List the elements of $\langle 6 \rangle$ in $\Integers_{16}$.

      \sol $\langle 6 \rangle = \{ 6, 12, 2, 8, 14, 4, 10, 0 \}$.

      \itm List the elements of $\langle f \rangle$ in $S_6$ where
      \[
         f = 
            \begin{pmatrix}
               1 & 2 & 3 & 4 & 5 & 6 \\
               6 & 1 & 3 & 2 & 5 & 4
            \end{pmatrix}
      \]

      \sol Let's write $f = (1642)$. $f^2 = (14)(26)$. $f^3 = (1246)$.
      $f^4 = e$. So $\langle f \rangle = \{ e, (1642), (14)(26), (1246) \}$.

      \itm Describe the cyclic subgroup $\langle \frac{1}{2} \rangle$ in 
      $\Reals^*$. Describe the cyclic subgroup $\langle \frac{1}{2} \rangle$
      in $\Reals$.

      \sol The cyclic subgroup of $\langle \frac{1}{2} \rangle$ in 
      $\Reals^*$ is the set $\{ 2^n \mid n \in \Integers\}$.
      the cyclic subgroup of $\langle \frac{1}{2} \rangle$ in $\Reals$
      is the set $\{ \frac{n}{2} \mid n \in \Integers \}$.

      \itm If $f(x) = x + 1$, describe the cyclic subgroup $\langle f
      \rangle$ of $S_\Reals$.

      \sol $\langle f \rangle = \{ f(x) = x + n \mid n \in \Integers \}$.

      \itm If $f(x) = x + 1$, describe the cyclic subgroup $\langle f
      \rangle$ of $\mathbb{F}(\Reals)$.

      \sol I had to remind myself that on p45 $\mathbb{F}(\Reals)$ is 
      the set of all functions from $\Reals$ to $\Reals$. I will assume that
      because the operation is not given in this problem that $+$ is assumed.
      So the second multiple of $f$ is $f(x) + f(x) = 2x + 2$.  So
      the multiples of $f$ are easy to describe as being of the
      form $nx + n$. The inverses are a little trickier. We want 
      $f^n(x) + f^{-n}(x) = \epsilon(x) = x$. So we have 
      $f^{-n}(x) = (1-n)x - n $.
      $\langle f \rangle = \{ nx + n \mid n \in \Integers \text{ and } n \ge
      0\} \cup \{(1-n)x - n \mid n \in \Integers \text{ and } n \ge 0\}$.

      \itm Show that -1, as well as 1, is a generator of $\Integers$. Are there
      any other generators of $\Integers$? Explain! What are the generators
      of an arbitrary infinite cyclic group $\langle a \rangle$?

      \sol I don't know how to show this. To show that 1 is a generator is
      to show that -1 is a generator. 1 (by itself) generates all of the
      positive integers and combined with -1 generates 0 and -1 generates
      all of the negative integers. They are inverses of each other.
      There are no other generators of $\Integers$. Assume you had some
      element $n$ that is a generator of $\Integers$. How does it
      generate 1? Some multiple of $n$ must generate 1. So we say
      $nq = 1$ or $q = 1/n$. But then $q$ is a fraction which doesn't make
      sense when talking about multiples of $n$. Therefore $n$ can't
      be a generator. (Note, this is unlike a finite group where 
      multiples "wrap around").
      
      \ind The generators of an arbitrary infinite cyclic group $\langle a
      \rangle$ are $a$ and $a^{-1}$. (By the same reasoning.)

      \itm Is $\Reals^*$ cyclic? Try to prove your answer.

      \sol No it is not cyclic. Assume that it is and $k$ is a generator. First
      we can rule out $k=1$ since $1^n = 1$ for all $n$. But now assume that $k
      > 1$.  Then we have $k < k^2 < k^3$. But there are elements of $\Reals^*$
      between $k$ and $k^2$ (Like $(k + k^2)/2$ and infinitely many more).  How
      are they generated?  They can't be generated by $k$. Therefore $k$
      is not a generator. Same goes if $k< 1$. But we can also use Theorem 1.
      If it were cyclic (and obviously of order infinity) then it would
      be isomorphic with $\Integers$. But we know the positive reals are not
      countable, therefore there is no isomorphism. And therfore they 
      are not cyclic.
   \end{enumerate}

\end{enumerate}

\end{document}
