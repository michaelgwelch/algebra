\documentclass[twoside]{amsart}
\usepackage{amssymb,latexsym}
\usepackage{amsfonts}
\usepackage{xspace}
\usepackage{enumerate}
\usepackage{graphics}
\usepackage{fitch}
\newcommand{\Rationals}{\mathbb{Q}{}}
\newcommand{\Reals}{\ensuremath{\mathbb{R}}\xspace}
\newcommand{\Integers}{\ensuremath{\mathbb{Z}{}}\xspace}
\newcommand{\solution}{\textsc{Solution}\xspace}
\newcommand{\problem}{\textsc{Problem}\xspace}
\newcommand{\Blank}{\mathrel{\phantom{=}}}
\newcommand{\ltrue}{\top}
\newcommand{\lfalse}{\bot}
\newcommand{\fOfg}{\ensuremath{f \circ g}\xspace}
\newcommand{\gOff}{\ensuremath{g \circ f}\xspace}
\newcommand{\eps}{\ensuremath{\epsilon}\xspace}
\newcommand{\iso}{\cong}
\newcommand{\niso}{\ncong}
\newcommand{\blank}{\vspace{5pt}}
\newcommand{\ind}{\hspace{.35in}}
\newcommand{\degree}{\ensuremath{^\circ}}
\newcommand{\real}{\mathop{\mathrm{real}}}
\newcommand{\img}{\mathop{\mathrm{img}}}
\newcommand{\first}{\mathop{\mathrm{first}}}
\newcommand{\second}{\mathop{\mathrm{second}}}
\newcommand{\abs}{\mathop{\mathrm{abs}}}
%\renewcommand{\qed}{\blacksquare}
\newcommand{\itm}{\blank\item}
\newcommand{\sol}{\blank\noindent\solution}
\newcommand{\ord}{\mathop{\mathrm{ord}}}
\newcommand{\cgroup}[1]{\langle #1 \rangle}
\begin{document}
\title{Answers to Chapter 11 Exercises - A Book of Abstract Algebra}
\author{Michael Welch}
\date{\today}
\maketitle

This document contains selected answers to exercises from chapter 11
of A Book of Abstract Algebra.


\begin{enumerate}[A.]
   
   \item \textsc{Examples of Cyclic Groups}

   \begin{enumerate}[1)]
      \itm List the elements of $\cgroup{6}$ in $\Integers_{16}$.

      \sol $\cgroup{6} = \{ 6, 12, 2, 8, 14, 4, 10, 0 \}$.

      \itm List the elements of $\cgroup{f}$ in $S_6$ where
      \[
         f = 
            \begin{pmatrix}
               1 & 2 & 3 & 4 & 5 & 6 \\
               6 & 1 & 3 & 2 & 5 & 4
            \end{pmatrix}
      \]

      \sol Let's write $f = (1642)$. $f^2 = (14)(26)$. $f^3 = (1246)$.
      $f^4 = e$. So $\cgroup{f} = \{ e, (1642), (14)(26), (1246) \}$.

      \itm Describe the cyclic subgroup $\cgroup{\frac{1}{2}}$ in 
      $\Reals^*$. Describe the cyclic subgroup $\langle \frac{1}{2} \rangle$
      in $\Reals$.

      \sol It's been a while. First I had to recall by referencing
      Chapter 2 p.~26 that $\Reals$ refers to the \emph{additive group
        of the real numbers $\langle \Reals, +\rangle$}; and
      $\Reals^*$ refers to the group $\langle \Reals, *\rangle$.

      \ind The cyclic subgroup of $\langle \frac{1}{2} \rangle$ in 
      $\Reals^*$ is the set $\{ 2^n \mid n \in \Integers\}$ which is
      $\{\ldots, \frac{1}{8}, \frac{1}{4}, \frac{1}{2}, 1, 2, 4, 8,
      \ldots \}$. 
      
      \ind The cyclic subgroup of $\langle \frac{1}{2} \rangle$ in
      $\Reals$ is the set $\{ \frac{n}{2} \mid n \in \Integers \}$
      which is $\{\ldots, -\frac{3}{2}, -1, -\frac{1}{2}, 0,
      \frac{1}{2}, 1, \frac{3}{2}, \ldots\}$.

      \itm If $f(x) = x + 1$, describe the cyclic subgroup $\langle f
      \rangle$ of $S_\Reals$.

      \sol Recall that the operator in use when permutations make up
      the set in a group is function composition. Therefore $f^2(x) =
      f(x) \circ f(x) = f(f(x)) = f(x+1) = (x+1)+1 = x+2$. Likewise
      $f^3(x) = f(f(f(x))) = f(f(x+1)) = f(x+2) = (x+2)+1 = x+3$.

      \ind In general then, $\langle f \rangle =
      \{ f(x) = x + n \mid n \in \Integers \}$.

      \ind Recall the identity element is $\epsilon(x) = x$ and the
      inverse $g^{-1}(x)$ of an element $g(x) = x + n$ is $g^{-1}(x) =
      x - n$. Let's just double check that for any $g \in \langle f
      \rangle$, $g \circ e = e \circ g = g$ and $g \circ g^{-1} =
      g^{-1} \circ g = e$. In the calculations below let $g(x) = x + n$.

      \begin{align*}
        g(x) \circ \epsilon(x) &= g(\epsilon(x)) \\
        &= g(x)\\
        &= x+n\\
        \epsilon(x) \circ g(x) &= \epsilon(g(x)) \\
        &= g(x) \\
        &= x + n
      \end{align*}

      and now let $g(x) = x + n$ and $g^{-1}(x) = x - n$:

      \begin{align*}
        g(x) \circ g^{-1}(x) &= g(g^{-1}(x)) \\
        &= g(x - n) \\
        &= (x - n) + n\\
        &= x\\
        &= \epsilon(x)\\
        g^{-1}(x) \circ g(x) &= g^{-1}(g(x)) \\
        &= g^{-1}(x + n)\\
        &= (x + n) - n\\
        &= x\\
        &= \epsilon(x)
      \end{align*}

      \itm If $f(x) = x + 1$, describe the cyclic subgroup $\langle f
      \rangle$ of $\mathbb{F}(\Reals)$.

      \sol I had to remind myself that on p45 $\mathbb{F}(\Reals)$ is 
      the additive group of all functions from $\Reals$ to $\Reals$.
      So the second multiple of $f$ is $f(x) + f(x) = 2x + 2$.  So
      the multiples of $f$ are easy to describe as being of the
      form $nx + n$. Let's prove it by induction on $n$.

      Base case with $n = 1$:
      \begin{align*}
        f^1(x) &= x + 1 \qquad \text{Given}
      \end{align*}

      Inductive case assuming that $f^n(x) = n(x + 1)$ for all $1 <=
      n < k$, show that $f^k(x) = k(x + 1)$.

      \begin{align*}
        f^k{x} &= f^{k-1}(x) + f(x) & \text{Definition of +}\\
        &= (k-1)(x+1) + (x+1) & \text{Given for $k-1$} \\
        &= k(x+1) \qedhere
      \end{align*}

      Now this covers all cases of $n > 0$. Let's find the identity
      element (or indeed show that the identity element is $f^{0} = 0(x
      + 1) = 0$. We have that for some element $g(x) = x + n \in \langle f
      \rangle$ for some $n > 0$ $g(x) + \epsilon(x) = g(x)$ so
      $\epsilon(x) = 0 = 0(x+1)$. Therefore we have shown that $g^n{x}
      = n(x+1)$ for $n > 0$ and $\epsilon(x) = g^0(x) = 0$.


      Now we must find the inverses. Again for some $g(x) = n(x+1)$
      for $n > 0$ we have that $g(x) + g^{-1}(x) = 0$, therefore
      $n(x+1) + g^{-1}(x) = 0$ and $g^{-1}(x) = -n(x+1)$. So we now
      have that $\langle f \rangle = \{f(x) = n(x+1) \mid n \in \Integers\}$.


      \itm Show that -1, as well as 1, is a generator of $\Integers$. Are there
      any other generators of $\Integers$? Explain! What are the generators
      of an arbitrary infinite cyclic group $\langle a \rangle$?

      \sol I don't know how to show this. To show that 1 is a generator is
      to show that -1 is a generator. 1 (by itself) generates all of the
      positive integers and combined with -1 generates 0 and -1 generates
      all of the negative integers. They are inverses of each other.
      There are no other generators of $\Integers$. Assume you had some
      element $n$ that is a generator of $\Integers$. How does it
      generate 1? Some multiple of $n$ must generate 1. So we say
      $nq = 1$ or $q = 1/n$. But then $q$ is a fraction which doesn't make
      sense when talking about multiples of $n$. Therefore $n$ can't
      be a generator. (Note, this is unlike a finite group where 
      multiples "wrap around").
      
      \ind The generators of an arbitrary infinite cyclic group $\langle a
      \rangle$ are $a$ and $a^{-1}$. (By the same reasoning.)

      \itm Is $\Reals^*$ cyclic? Try to prove your answer.

      \sol No it is not cyclic. Assume that it is and $k$ is a generator. First
      we can rule out $k=1$ since $1^n = 1$ for all $n$. But now assume that $k
      > 1$.  Then we have $k < k^2 < k^3$. But there are elements of $\Reals^*$
      between $k$ and $k^2$ (Like $(k + k^2)/2$ and infinitely many more).  How
      are they generated?  They can't be generated by $k$. Therefore $k$
      is not a generator. Same goes if $k< 1$. But we can also use Theorem 1.
      If it were cyclic (and obviously of order infinity) then it would
      be isomorphic with $\Integers$. But we know the positive reals are not
      countable, therefore there is no isomorphism. And therfore they 
      are not cyclic.
   \end{enumerate}

   \itm \textsc{Elementary Properties of Cyclic Groups}

   \noindent Prove each of the following:

   \begin{enumerate}[1)]

      \itm If $G$ is a group of order $n$, $G$ is cyclic iff $G$ has an 
      element of order $n$.

      \sol I am given that $G$ is a group of order $n$. Now I must
      show that 1) $G$ is cyclic implies $G$ has an element of order $n$ 
      2) $G$ has an element of order $n$ implies $G$ is cyclic.

      1) $G$ is a group of order $n$ which means it has $n$ elements. 
      Also $G$ is cyclic which means that all its elements are generated
      by one element. Assume this element is $a$. Then we know that
      $G = \langle a \rangle = \{a^0=e,a,a^2,\ldots,a^{n-1}\}$. Now
      I just need to show that $a^n = e$. Assume that $a^n \ne e$.
      Then I have the case $a^n = a^i$ for some $0 < i < n$.

      \begin{align*}
         a^n &= a^i \\
         a^na^{-i} &= a^ia^{-i} \\
         a^{n-i} &= e
      \end{align*}

      Let $j = n-i$. I know that $0 < j < n$. So one of the multiples of
      $a$ is equal to $e$. But this implies that we don't have $n$ distinct
      elements. This contradicts our given. Therefore $a^n = e$ and also
      $n$ is the order of $a$.

      2) $G$ is a group of order $n$. Again, this means it has $n$ elements.
      Also the order of $a \in G$ is $n$. By Theorem 3 of Chapter 10 on p105
      we know that there are $n$ distinct powers of $a$. Since there
      are $n$ distinict powers of $a$ and $n$ elements in $G$ it must be
      the case that all the elements in $G$ are generated by $a$ and
      therefore $G$ is cyclic. $\qed$

      \itm Every cyclic group is abelian.

      \sol Recall an abelian group is a group whose elements 
      commute. Assume some cyclic group $G = \langle a \rangle$. Choose
      any two elements $a^i$, $a^j$ and show that $a^i a^j = a^j a^i$.
      \begin{align*}
         a^i a^j &= a^{i+j} \\
                 &= a^{j+i} \\
                 &= a^j a^i \qed
      \end{align*}

      \itm If $G = \langle a \rangle$ and $b \in G$, the order of $b$
      is a factor of the order of $a$.

      \sol Let the order of $a$ equal $n$. Let $b=a^m$ for some $0 \le m < n$.
      By exercise 10G5 we know that $\ord(b) = \ord(a^m) =
      \frac{\mathrm{lcm}(m,n)}{m}$.  Let's name this value $l$.
      So $\ord(b)=l$. But also we know that $b^n = (a^m)^n = (a^n)^m = e$.
      By theorem 5 of chapter 10, this tells us that $n$ is a multiple of
      $l$. In other words $l$ is a factor of $n$. $\qed$

      \itm In any cyclic group of order $n$, there are elements of order $k$
      for every integer $k$ which divides $n$.

      \sol Let $G = \langle a \rangle$ and let $G$ have order $n$. 
      This tells us that the order of $a$ is $n$.
      Choose some value of $k$ that divides $n$ such that 
      $n = kq$ for some integer value of $k$ and $q$. Then
      $(a^q)^k = (a^q)^{n/q} = a^n = e$. Now assume that there
      is some value $k' < k$ for which $(a^q)^{k'} = e$. Then
      $(a^{n/k})^{k'} = a^{(k'/k)n} = e$. But $k'/k < 1$. So this implies
      that there is a power of $a$ less than $n$ that equals $e$ which
      contradicts our givens. Therefore there is no $k'$. Therefore
      $k$ is the order of $a^q$. Therefore for any $k | n$ there is
      an element $a^q$ where $q = n/k$ that has order $k$. $\qed$.

      \itm Let $G$ be an abelian group of order $mn$, where $m$ and $n$
      are relatively prime. If $G$ has an element of order $m$
      and an element of order $n$, $G$ is cyclic. (See Chapter 10,
      Exercise E4.)

      \sol Let $a$ equal the element of order $m$ and let $b$ equal the
      element of order $n$. By Exercise 10E4 we know that since $a$ and
      $b$ commute that the order of $ab$ is $mn$. Since there is an element
      of order $mn$ in $G$ which is of order $mn$ we know by part 1
      that $G$ is cyclic.

      \itm Let $\cgroup{ a }$ be a cylic group of order $n$. 
      If $n$ and $m$ are relatively prime, then the function $f(x) =
      x^m$ is an automorphism of $\cgroup{ a }$. (\textsc{Hint}: Use
      Exercise B3 and Chapter 10, Theorem 5.)

      \sol Need to show that $f$ is injective and surjective.
      Let $b,c \in G$. Show that $f(b) = f(c) \Rightarrow b = c$.
      We can write $b = a^i$ and $c = a^j$. Assume 
      that $(a^i)^m = (a^j)^m$.  Then $(a^m)^i = (a^m)^j$.
      We can divide each by $n$: 
      $a^{im} = a^{nq_1}a^{r_1}$ and $a^{jm} = a^{nq_2}a^{r_2}$ where
      $0 \le r_1, r_2 < n$. So we have 
      $a^{nq_1}a^{r_1} = a^{nq_2}a^{r_2}$. But the powers of $n$ terms
      reduce to $e$ leaving $a^{r_1}=a^{r_2}$. By Theorem 3 of Chapter 10
      we know that $r_1 = r_2$. Let's just call the term $r$.
      So $a^{im} = a^r = a^{jm}$.

      Let's try this another way. Let's assume that $a^i \ne a^j$.
      $a^{im} = a^{jm}$. So we have $im = nq_1+r$ and $jm = nq_2 + r$.



      \ind Now to show $f$ is surjective. Let $y = a^i$ be
      some value in $G$. I must find some value of $x$
      such that $f(x) = x^m = a^i$.

      
   \end{enumerate}

   \itm \textsc{Generators of Cycle Groups}

   \noindent For any positive integer $n$, let $\phi(n)$ denote the
   number of positive integers less than $n$ and relatively prime to
   $n$. For example, 1, 2, 4, 5, 7, and 8 are relatively prime to 9,
   so $\phi(n) = 6$. Let $a$ have order $n$, and prove the following:

   \begin{enumerate}
     \item $a^r$ is a generator of $\langle a \rangle$ iff $r$ and $n$
       are relatively prime. (\textsc{Hint}: See Chapter 10 Exercise G2.)
   \end{enumerate}

\end{enumerate}

\end{document}
