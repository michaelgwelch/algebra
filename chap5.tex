\documentclass[twoside]{amsart}
\usepackage{amssymb,latexsym}
\usepackage{xspace}
\usepackage{enumerate}
\usepackage{graphics}
\newcommand{\Rationals}{\mathbb{Q}{}}
\newcommand{\Reals}{\mathbb{R}{}}
\newcommand{\Integers}{\mathbb{Z}{}}
\newcommand{\solution}{\textsc{Solution}\xspace}
\newcommand{\problem}{\textsc{Problem}\xspace}
\newcommand{\Blank}{\mathrel{\phantom{=}}}
\newcommand{\ltrue}{\top}
\newcommand{\lfalse}{\bot}
\begin{document}
\title{Answers to Chapter 5 Exercises - A Book of Abstract Algebra}
\author{Michael Welch}
\date{\today}
\maketitle

This document contains selected answers to exercises from chapter 5
of A Book of Abstract Algebra.


\begin{enumerate}[A.]
   \item \textsc{Recognizing Subgroups}

   \noindent In parts 1--6 below, determine whether or not $H$ is a
   subgroup of $G$. (Assume that the operation of $H$ is the same as
   that of $G$.)

   \textbf{Instructions} If $H$ is a subgroup of $G$, show that
   both conditions in the definition of ``subgroup'' are satisfied.
   If $H$ \emph{is not} a subgroup of $G$, explain which condition
   fails.
  

   \noindent 1. $G = \langle \mathbb{R},+ \rangle,\, H = 
   \{\log a : a \in \mathbb{Q}, a > 0 \}$.

   \noindent \solution $H$ is a subgroup of $G$:
   \begin{enumerate}[(i)]
      \item Assume $\log a, \log b \in H$. Since $a,b$ are rational numbers
      there are numbers $m_1,n_1,m_2,n_2$ such that $a = m_1/n_1, b = m_2/n_2$.
      So we have 
      \begin{align*}
         \log a + \log b & = \log \frac{m_1}{n_1} + \log \frac{m_2}{n_2} \\
	                 & = \log m_1 - \log n_1 + \log m_2 - \log n_2 \\
			 & = \log m_1m_2 - \log n_1n_2 \\
			 & = \log \frac{m_1m_2}{n_1n_2}
      \end{align*}

      Since $\displaystyle \frac{m_1m_2}{n_1n_2} \in \mathbb{Q}$ 
      we have that $H$ is closed with respect to addition.

      \item Assume $\log a \in H$. Again we know that there are numbers
      $m,n$ such that $a = m/n$. So we have
      \begin{align*}
         - \log a & = - \log \frac{m}{n} \\
	          & = - (\log m - \log n) \\
		  & = \log n - \log m \\
		  & = \log \frac{n}{m}
      \end{align*}

      Since $\displaystyle \frac{n}{m} \in \mathbb{Q}$ we have that 
      $H$ is closed with respect to inverses.
   \end{enumerate}

   \noindent 2. $G = \langle \mathbb{R}, + \rangle,\, H = 
   \{ \log n : n \in \mathbb{Z}, n > 0\}$.

   \noindent \solution It is rather easy to show that $H$ is closed
   with respect to addition (following a procedure similar to the
   previous example). However, it is not closed with respect to
   inverses:
      \begin{align*}
         - \log n & = - \log \frac{n}{1} \\
	          & = - (\log n - \log 1) \\
		  & = \log 1 - \log n \\
		  & = \log \frac{1}{n}
      \end{align*}

   Since $\displaystyle \frac{1}{n} \not \in \mathbb{Z}$, $H$ is
   not closed with respect to inverses.

   \noindent 3. $G = \langle \mathbb{R},+ \rangle,\, H = 
   \{ x \in \mathbb{R} : \tan x \in \mathbb{Q}\}$. \textsc{Hint}:
   Use the following formula from trigonometry:
   \[
      \tan (x+y) = \frac {\tan x + \tan y}{1 - \tan x \tan y}
   \]

   \noindent \solution Assume $x,y \in \mathbb{R}$ and $\tan x,
   \tan y \in \mathbb{Q}$. We know that $x+y \in \mathbb{R}$.
   We need to check that $\tan (x+y) \in \mathbb{Q}$. Choose 
   $m_1,n_1,m_2,n_2 \in \mathbb{Z}$ such that $\tan x = m_1/n_1$
   and $\tan y = m_2/n_2$.
   \begin{align*}
      \tan (x + y) & = \frac {\tan x + \tan y}{1 - \tan x \tan y}\\
                   & = \frac {m_1/n_1 + m_2/n_2}{1 - m_1m_2/n_1n_2} \\
   \end{align*}

   This shows us that $H$ is not closed with respect to addition. If
   $\tan x \tan y = 1$ then we have $\displaystyle \tan (x+y) =
   \frac{\tan x + \tan y}{0} \not \in \mathbb{Q}$.

   \noindent 4. $G = \langle \mathbb{R}^*,\cdot \rangle,\, H =
   \{2^n3^m : m,n \in \mathbb{Z}\}$.   
   
   \noindent \solution 
   \begin{enumerate}[(i)]
      \item Assume that we have two numbers $2^{n_1}3^{m_1}, 2^{n_2}3^{m_2} \in
      H$.  $2^{n_1}3^{m_1}\cdot 2^{n_2}3^{m_2} = 2^{n_1+n_2}3^{m_1+m_2}$, and
      $n_1+n_2,m_1+m_2 \in \mathbb{Z}$. So $H$ is closed with respect to
      multiplication.

      \item Assume that we have a number $2^n3^m \in H$. Then
      $1/2^n3^m = 2^{-n}3^{-m}$, and $-n, -m \in \mathbb{Z}$. So
      $H$ is closed with respect to multiplicative inverses.
   \end{enumerate}

   \noindent 5. $G = \langle \mathbb{R} \times \mathbb{R},+ \rangle,\, H =
   \{ (x,y) : y = 2x\}$.

   \noindent \solution 
   
   \begin{enumerate}[(i)]
      \item Assume we have two tuples $(x_1,y_1),(x_2,y_2) \in H$. Then
      $(x_1,y_1)+(x_2,y_2) = (x_1+x_2,y_1+y_2)$. Since we know that $y_1 =
      2x_1$ and $y_2 = 2x_2$ we have $y_1+y_2 = 2x_1 + 2x_2 = 2(x_1+x_2)$.
      Therefore, $H$ is closed with respect to addition.

      \item Assume we have a tuple $(x,y)\in H$. The inverse is
      $(-x,-y)$. We know $y=2x$, so we have $-y = -2x = 2 \cdot -x$.
      Therefore, $H$ is closed with respect to inverses.

   \end{enumerate}

   \noindent 6. $ G =\langle \mathbb{R} \times \mathbb{R}, + \rangle,\, H=
   \{(x,y) : x^2 + y^2 > 0\}$.

   \noindent \solution 
   \begin{enumerate}[(i)]
      \item Assume we have two tuples $(x_1,y_1),(x_2,y_2)\in H$. Then
      $(x_1,y_1) + (x_2,y_2) = (x_1+x_2,y_1+y_2)$. $(x_1+x_2)^2 +
      (y_1+y_2)^2 = x_1^2 + 2x_1x_2 + x_2^2 + y_1^2 + 2y_1y_2 + y_2^2$.
      We can find values of $x_1,x_2,y_1,y_2$ such that 
      $ x_1^2 + 2x_1x_2 + x_2^2 + y_1^2 + 2y_1y_2 + y_2^2 = 0$. For example,
      choose the tuples $(-\frac{1}{4},\frac{1}{4})$ and $(\frac{1}{4},
      - \frac{1}{4})$. When we add them together we get $(0,0)$. 
      $0^2 + 0^2 \not > 0$.
      Therefore, $H$ is not closed with respect to addition.
   \end{enumerate}

   \noindent 7. Let $C$ and $D$ be sets, with $C \subseteq D$. Prove
   that $P_C$ is a subgroup of $P_D$. (See Chapter 3, Exercise C.)
   As a reminder the operation on $P_D$ is the \emph{symmetric difference},
   $\ominus$.

   \noindent \solution  $\langle P_C,\ominus \rangle$ is a subgroup of
   $\langle P_D,\ominus \rangle$.
   
   \begin{enumerate}[(i)]
      \item Assume we have two sets $A,B \in P_C$. We know $P_C = \{ A' : A'
      \subseteq C\}$. Therefore, $A \subseteq C$ and $B \subseteq C$. If we
      remove any elements from $A$ or $B$ it is obvious the result is also a
      subset of $C$. Next I need to mention that the union of any two subsets
      of $C$ is also a subset of $C$.  We have $A \ominus B = (A - B) \cup (B -
      A)$. Since $A-B\subseteq C$ and $B-A\subseteq C$ and $(A-B)\cup (B-A)
      \subseteq C$ we have that $P_C$ is closed with respect to $\ominus$.


      \item From Chapter 3 Exercise C we know that the inverse of A is A. 
      Therefore $\langle P_C, \ominus \rangle$ is closed with respect
      to inverses.
   \end{enumerate}

   \item \textsc{Subgroups of Functions}

   \noindent In each of the following, show that $H$ is a subgroup of $G$.

   \noindent 1. \problem $G = \langle \mathcal{F}(\mathbb{R}), + \rangle,\, H=
   \{ f \in \mathcal{F}(\mathbb{R}) : f(x) = 0 \text{ for every } 
   x \in [0,1]\}$.

   \noindent \solution We will show $H$ is a subgroup of $G$.
   
   \begin{enumerate}[(i)]
      \item Assume we have $f,g \in H$; then $f(x) = 0$ for $x \in [0,1]$ and
      $g(x) = 0$ for $x \in [0,1]$. So $[f+g](x) = f(x) + g(x) = 0 + 0 = 0$ for
      $x \in [0,1]$. Thus $f+g \in H$.

      \item If $f \in H$, then $f(x) = 0$ for $x \in [0,1]$. Thus,
      $[-f](x)=-f(x)=-0=0$ for $x \in [0,1]$. So $-f \in H$.

   \end{enumerate}

   \noindent 2. \problem $G = \langle \mathcal{F}(\mathbb{R}), + \rangle,\, H =
   \{ f \in \mathcal{F}(\mathbb{R}) : f(-x) = -f(x)\}$.

   \noindent \solution We will show $H$ is a subgroup of $G$.

   \begin{enumerate}[(i)]
      \item Suppose $f,g\in H$; then $f(-x) = -f(x)$ and $g(-x) = -g(x)$.
      So $[f+g](-x) = f(-x) + g(-x) = -f(x) - g(x) = -(f(x) + g(x)) = 
      - [f+g](x)$. Thus, $f+g\in H$.

      \item Suppose $f \in H$; then $f(-x)= -f(x)$. Thus 
      $[-f](-x) = -f(-x) = -(-f(x)) = f(x)$. Thus, $f\in H$.
   \end{enumerate}

   \noindent \problem 3. $G=\langle \mathcal{F}(\mathbb{R}),+\rangle,\, H=
   \{ f \in \mathcal{F}(\mathbb{R}) : f \text{ is periodic of period $\pi$}\}$

\end{enumerate}




\end{document}
