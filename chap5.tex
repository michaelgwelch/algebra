\documentclass[twoside]{amsart}
\usepackage{amssymb,latexsym}
\usepackage{xspace}
\usepackage{enumerate}
\usepackage{graphics}
\newcommand{\Rationals}{\mathbb{Q}{}}
\newcommand{\Reals}{\mathbb{R}{}}
\newcommand{\Integers}{\mathbb{Z}{}}
\newcommand{\solution}{\textsc{Solution}\xspace}
\newcommand{\problem}{\textsc{Problem}\xspace}
\newcommand{\Blank}{\mathrel{\phantom{=}}}
\newcommand{\ltrue}{\top}
\newcommand{\lfalse}{\bot}
\begin{document}
\title{Answers to Chapter 5 Exercises - A Book of Abstract Algebra}
\author{Michael Welch}
\date{\today}
\maketitle

This document contains selected answers to exercises from chapter 5
of A Book of Abstract Algebra.


\begin{enumerate}[A.]
   \item \textsc{Recognizing Subgroups}

   \noindent In parts 1--6 below, determine whether or not $H$ is a
   subgroup of $G$. (Assume that the operation of $H$ is the same as
   that of $G$.)

   \textbf{Instructions} If $H$ is a subgroup of $G$, show that
   both conditions in the definition of ``subgroup'' are satisfied.
   If $H$ \emph{is not} a subgroup of $G$, explain which condition
   fails.
  

   \noindent 1. $G = \langle \mathbb{R},+ \rangle,\, H = 
   \{\log a : a \in \mathbb{Q}, a > 0 \}$.

   \noindent \solution $H$ is a subgroup of $G$:
   \begin{enumerate}[(i)]
      \item Assume $\log a, \log b \in H$. Since $a,b$ are rational numbers
      there are numbers $m_1,n_1,m_2,n_2$ such that $a = m_1/n_1, b = m_2/n_2$.
      So we have 
      \begin{align*}
         \log a + \log b & = \log \frac{m_1}{n_1} + \log \frac{m_2}{n_2} \\
	                 & = \log m_1 - \log n_1 + \log m_2 - \log n_2 \\
			 & = \log m_1m_2 - \log n_1n_2 \\
			 & = \log \frac{m_1m_2}{n_1n_2}
      \end{align*}

      Since $\displaystyle \frac{m_1m_2}{n_1n_2} \in \mathbb{Q}$ 
      we have that $H$ is closed with respect to addition.

      \item Assume $\log a \in H$. Again we know that there are numbers
      $m,n$ such that $a = m/n$. So we have
      \begin{align*}
         - \log a & = - \log \frac{m}{n} \\
	          & = - (\log m - \log n) \\
		  & = \log n - \log m \\
		  & = \log \frac{n}{m}
      \end{align*}

      Since $\displaystyle \frac{n}{m} \in \mathbb{Q}$ we have that 
      $H$ is closed with respect to inverses.
   \end{enumerate}

   \noindent 2. $G = \langle \mathbb{R}, + \rangle,\, H = 
   \{ \log n : n \in \mathbb{Z}, n > 0\}$.

   \noindent \solution It is rather easy to show that $H$ is closed
   with respect to addition (following a procedure similar to the
   previous example). However, it is not closed with respect to
   inverses:
      \begin{align*}
         - \log n & = - \log \frac{n}{1} \\
	          & = - (\log n - \log 1) \\
		  & = \log 1 - \log n \\
		  & = \log \frac{1}{n}
      \end{align*}

   Since $\displaystyle \frac{1}{n} \not \in \mathbb{Z}$, $H$ is
   not closed with respect to inverses.

   \noindent 3. $G = \langle \mathbb{R},+ \rangle,\, H = 
   \{ x \in \mathbb{R} : \tan x \in \mathbb{Q}\}$. \textsc{Hint}:
   Use the following formula from trigonometry:
   \[
      \tan (x+y) = \frac {\tan x + \tan y}{1 - \tan x \tan y}
   \]

   \noindent \solution Assume $x,y \in \mathbb{R}$ and $\tan x,
   \tan y \in \mathbb{Q}$. We know that $x+y \in \mathbb{R}$.
   We need to check that $\tan (x+y) \in \mathbb{Q}$. Choose 
   $m_1,n_1,m_2,n_2 \in \mathbb{Z}$ such that $\tan x = m_1/n_1$
   and $\tan y = m_2/n_2$.
   \begin{align*}
      \tan (x + y) & = \frac {\tan x + \tan y}{1 - \tan x \tan y}\\
                   & = \frac {m_1/n_1 + m_2/n_2}{1 - m_1m_2/n_1n_2} \\
   \end{align*}

   This shows us that $H$ is not closed with respect to addition. If
   $\tan x \tan y = 1$ then we have $\displaystyle \tan (x+y) =
   \frac{\tan x + \tan y}{0} \not \in \mathbb{Q}$.

   \noindent 4. $G = \langle \mathbb{R}^*,\cdot \rangle,\, H =
   \{2^n3^m : m,n \in \mathbb{Z}\}$.   
   
   \noindent \solution 
   \begin{enumerate}[(i)]
      \item Assume that we have two numbers $2^{n_1}3^{m_1}, 2^{n_2}3^{m_2} \in
      H$.  $2^{n_1}3^{m_1}\cdot 2^{n_2}3^{m_2} = 2^{n_1+n_2}3^{m_1+m_2}$, and
      $n_1+n_2,m_1+m_2 \in \mathbb{Z}$. So $H$ is closed with respect to
      multiplication.

      \item Assume that we have a number $2^n3^m \in H$. Then
      $1/2^n3^m = 2^{-n}3^{-m}$, and $-n, -m \in \mathbb{Z}$. So
      $H$ is closed with respect to multiplicative inverses.
   \end{enumerate}

   \noindent 5. $G = \langle \mathbb{R} \times \mathbb{R},+ \rangle,\, H =
   \{ (x,y) : y = 2x\}$.

   \noindent \solution 
   
   \begin{enumerate}[(i)]
      \item Assume we have two tuples $(x_1,y_1),(x_2,y_2) \in H$. Then
      $(x_1,y_1)+(x_2,y_2) = (x_1+x_2,y_1+y_2)$. Since we know that $y_1 =
      2x_1$ and $y_2 = 2x_2$ we have $y_1+y_2 = 2x_1 + 2x_2 = 2(x_1+x_2)$.
      Therefore, $H$ is closed with respect to addition.

      \item Assume we have a tuple $(x,y)\in H$. The inverse is
      $(-x,-y)$. We know $y=2x$, so we have $-y = -2x = 2 \cdot -x$.
      Therefore, $H$ is closed with respect to inverses.

   \end{enumerate}

   \noindent 6. $ G =\langle \mathbb{R} \times \mathbb{R}, + \rangle,\, H=
   \{(x,y) : x^2 + y^2 > 0\}$.

   \noindent \solution 
   \begin{enumerate}[(i)]
      \item Assume we have two tuples $(x_1,y_1),(x_2,y_2)\in H$. Then
      $(x_1,y_1) + (x_2,y_2) = (x_1+x_2,y_1+y_2)$. $(x_1+x_2)^2 +
      (y_1+y_2)^2 = x_1^2 + 2x_1x_2 + x_2^2 + y_1^2 + 2y_1y_2 + y_2^2$.
      We can find values of $x_1,x_2,y_1,y_2$ such that 
      $ x_1^2 + 2x_1x_2 + x_2^2 + y_1^2 + 2y_1y_2 + y_2^2 = 0$. For example,
      choose the tuples $(-\frac{1}{4},\frac{1}{4})$ and $(\frac{1}{4},
      - \frac{1}{4})$. When we add them together we get $(0,0)$. 
      $0^2 + 0^2 \not > 0$.
      Therefore, $H$ is not closed with respect to addition.
   \end{enumerate}

   \noindent 7. Let $C$ and $D$ be sets, with $C \subseteq D$. Prove
   that $P_C$ is a subgroup of $P_D$. (See Chapter 3, Exercise C.)
   As a reminder the operation on $P_D$ is the \emph{symmetric difference},
   $\ominus$.

   \noindent \solution  $\langle P_C,\ominus \rangle$ is a subgroup of
   $\langle P_D,\ominus \rangle$.
   
   \begin{enumerate}[(i)]
      \item Assume we have two sets $A,B \in P_C$. We know $P_C = \{ A' : A'
      \subseteq C\}$. Therefore, $A \subseteq C$ and $B \subseteq C$. If we
      remove any elements from $A$ or $B$ it is obvious the result is also a
      subset of $C$. Next I need to mention that the union of any two subsets
      of $C$ is also a subset of $C$.  We have $A \ominus B = (A - B) \cup (B -
      A)$. Since $A-B\subseteq C$ and $B-A\subseteq C$ and $(A-B)\cup (B-A)
      \subseteq C$ we have that $P_C$ is closed with respect to $\ominus$.


      \item From Chapter 3 Exercise C we know that the inverse of A is A. 
      Therefore $\langle P_C, \ominus \rangle$ is closed with respect
      to inverses.
   \end{enumerate}

   \item \textsc{Subgroups of Functions}

   \noindent In each of the following, show that $H$ is a subgroup of $G$.

   \noindent 1. \problem $G = \langle \mathcal{F}(\mathbb{R}), + \rangle,\, H=
   \{ f \in \mathcal{F}(\mathbb{R}) : f(x) = 0 \text{ for every } 
   x \in [0,1]\}$.

   \noindent \solution We will show $H$ is a subgroup of $G$.
   
   \begin{enumerate}[(i)]
      \item Assume we have $f,g \in H$; then $f(x) = 0$ for $x \in [0,1]$ and
      $g(x) = 0$ for $x \in [0,1]$. So $[f+g](x) = f(x) + g(x) = 0 + 0 = 0$ for
      $x \in [0,1]$. Thus $f+g \in H$.

      \item If $f \in H$, then $f(x) = 0$ for $x \in [0,1]$. Thus,
      $[-f](x)=-f(x)=-0=0$ for $x \in [0,1]$. So $-f \in H$.

   \end{enumerate}

   \noindent 2. \problem $G = \langle \mathcal{F}(\mathbb{R}), + \rangle,\, H =
   \{ f \in \mathcal{F}(\mathbb{R}) : f(-x) = -f(x)\}$.

   \noindent \solution We will show $H$ is a subgroup of $G$.

   \begin{enumerate}[(i)]
      \item Suppose $f,g\in H$; then $f(-x) = -f(x)$ and $g(-x) = -g(x)$.
      So $[f+g](-x) = f(-x) + g(-x) = -f(x) - g(x) = -(f(x) + g(x)) = 
      - [f+g](x)$. Thus, $f+g\in H$.

      \item Suppose $f \in H$; then $f(-x)= -f(x)$. Thus 
      $[-f](-x) = -f(-x) = -(-f(x)) = f(x)$. Thus, $f\in H$.
   \end{enumerate}

   \noindent 3. \problem  $G=\langle \mathcal{F}(\mathbb{R}),+\rangle,\, H=
   \{ f \in \mathcal{F}(\mathbb{R}) : f \text{ is periodic of period $\pi$}\}$

   \noindent \textsc{Remark:} A function $f$ is said to be \emph{periodic}
   of period $a$ if there is a number $a$, called the period of $f$,
   such that $f(x) = f(x+na)$ for every $x \in \mathbb{R}$ and 
   $n \in \mathbb{Z}$.

   \noindent \solution 
   
   \begin{enumerate}[(i)]
      \item  Suppose $f,g \in H$; then $f(x) = f(x+na)$ and 
      $g(x) = g(x+na)$ for every $x \in \mathbb{R}$ and $n \in \mathbb{Z}$.
      So $[f+g](x) = f(x) + g(x) = f(x + na) + g(x+na)$ which equals
      $[f+g](x + na)$ for all $x\in \mathbb{R}$ and $n \in \mathbb{Z}$. 
      Thus, $f+g \in H$.
   \end{enumerate}

   \noindent 4. \problem $G = \langle \mathcal{F}(\mathbb{R}),+\rangle,\, H=
   \{ f \in \mathcal{F}(\mathbb{R}) : \int_0^1 f(x) dx = 0\}$

   \noindent \solution 
   \begin{enumerate}[(i)]
      \item Suppose $f,g \in H$; then $\int_0^1 f(x) dx = 0$ and
      $\int_0^1 g(x)dx = 0$. So $[f+g](x) = f(x) + g(x)$ and
      $\int_0^1 (f(x)+g(x)) dx = \int_0^1 f(x)dx + \int_0^1 g(x)dx = 0$.
      Thus, $f+g \in H$.

      \item Suppose $f \in H$; then $[-f](x) = -f(x)$ and
      $\int_0^1 -f(x)dx = -\int_0^1 f(x) dx = -0 = 0$. Thus, $-f \in H$.
   \end{enumerate}

   \noindent 5. \problem $G = \langle \mathcal{F}(\mathbb{R}),+\rangle,\, H =
   \{ f \in \mathcal{F}(\mathbb{R}) : df/dx \text{ is constant}\}$

   \noindent \solution 
   \begin{enumerate}[(i)]
      \item Suppose $f,g \in H$; then $[f+g](x) = f(x) + g(x)$, and
      $df/dx = k_1$ and $dg/dx = k_2$ and 
      $d(f+g)/dx = df/dx + dg/dx = k_1 + k_2 = k_3$. So $k_3$ is a constant
      and $f+g\in H$.

      \item Suppose $f \in H$; then $[-f](x) = -f(x)$, and
      $d(-f)/dx = -df/dx = -k$. Thus, $-f \in H$.
   \end{enumerate}

   \noindent 6. \problem $G=\langle \mathcal{F}(\mathbb{R}),+\rangle,\, H =
   \{ f \in \mathcal{F}(\mathbb{R}) : f(x) \in \mathbb{Z}$ for every
   $x \in \mathbb{R}\}$.

   \noindent \solution 
   \begin{enumerate}[(i)]
      \item Suppose $f,g \in H$; then $f(x)\in\mathbb{Z}$ and
      $g(x)\in\mathbb{Z}$ for every $x\in\mathbb{R}$. So
      $[f+g](x)=f(x)+g(x)$ and $f(x)+g(x) \in \mathbb{Z}$ for every
      $x \in \mathbb{R}$.

      \item Suppose $f \in H$; then $[-f](x)=-f(x)$. But $-f(x) \in 
      \mathbb{Z}$ for every $x \in \mathbb{R}$.
   \end{enumerate}

   \item \textsc{Subgroups of Abelian Groups}

   \noindent In the following exercises, let $G$ be an abelian group.

   \noindent 1. If $H=\{x \in G : x = x^{-1}\}$, that is, $H$ consists of
   all the elements of $G$ which are their own inverses, prove that $H$
   is a subgroup of $G$.

   \noindent \solution Suppose $x,y \in H$; then $x=x^{-1}$ and $y=y^{-1}$.
   So $xy = x^{-1}y^{-1} = (yx)^{-1} = (xy)^{-1}$. Thus $xy \in H$; and
   by definition $x^{-1} \in H$.

   \noindent 2. Let $n$ be a fixed integer, and let $H=\{x \in G
   : x^n = e\}$. Prove that $H$ is a subgroup of $G$.

   \noindent \solution 
   \begin{enumerate}[(i)]
      \item Suppose $x,y\in H$ then $x^n=e$ and $y^n=e$ and
	 \begin{align*}
	 (xy)^n & = \underbrace{(xy)(xy)(xy)\cdots (xy)}_{n \text{ factors}}\\
	        & = \underbrace{xxx\cdots x}_{n\text{ factors}}
		    \underbrace{yyy\cdots y}_{n\text{ factors}}
		    && \text{Commutative property} \\
		& = e
	 \end{align*}
      \item Supose $x\in H$ then $x^n = e$. Let's check if the inverse is
      in $H$.  
      \begin{align*}
          e & = x^n \\
	  e & = \underbrace{xx\cdots x}_{n\text{ factors}} \\
	  x^{-1} & = \underbrace{xx\cdots x}_{n-1\text{ factors}}\\
	  \underbrace{x^{-1}x^{-1}\cdots x^{-1}}_{n\text{ factors}} & = e \\
	  (x^n)^{-1} & = e
      \end{align*}

      Therefore $(x^n)^{-1} \in H$.
   \end{enumerate}

   \noindent 3. Let $H=\{x\in G : x = y^2 \text{ for some } y \in G\}$;
   that is, let $H$ be the set of all the elements of $G$ which
   have a square root. Prove that $H$ is a subgroup of $G$.

   \noindent \solution
   \begin{enumerate}[(i)]
      \item Suppose $a,b \in H$ then there are two elements $x,y\in G$
      such that $a=x^2, b=y^2$. 
      \begin{align*}
         ab & = x^2y^2 \\
	    & = xxyy   \\
	    & = xyxy   && \text{By commutativity} \\
	    & = (xy)^2
      \end{align*}
      Therefore, $ab$ has a square root and $ab \in H$.

      \item Suppose $x \in H$. Then $x=y^2$ for some $y \in G$. 
      \begin{align*}
         x & = y^2 \\
	 x^{-1} & = (y^2)^{-1} \\
	        & = (yy)^{-1} \\
		& = y^{-1}y^{-1} \\
		& = (y^{-1})^2
      \end{align*}
      Therefore $x^{-1}$ has an inverse and since $y^{-1}\in G$ 
      we have $x^{-1} \in H$.
   \end{enumerate}

   \noindent 4. Let $H$ be a subgroup of $G$, and let 
   $K = \{x \in G : x^2 \in H\}$. Prove that $K$ is a subgroup of $G$.

   \noindent \solution 
   \begin{enumerate}[(i)]
      \item Suppose $x,y \in K$. Therefore $x,y\in G$, and $x^2,y^2 \in H$.
      Since $G$ is group we know $xy \in G$. Since $H$ is a group we
      have that $x^2y^2 \in H$. But $x^2y^2 = xxyy = (xy)^2$ (by commutativity).
      Since $xy\in G$ and $(xy)^2 \in H$ we have that $xy \in K$ (by
      the definition of $K$.

      \item Suppose $x \in K$. Therefore $x \in G, x^2 \in H$. Since
      $G$ and $H$ are groups we have that $x^{-1} \in G$ and 
      $(x^2)^{-1} \in H$. But $(x^2)^{-1} = x^{-1}x^{1} = (x^{-1})^2$.
      Since $x^{-1} \in G$ and $(x^{-1})^2 \in H$, we have
      $x^{-1} \in K$.
   \end{enumerate}
   

   \noindent 5. Let $H$ be a subgroup of $G$, and let $K$ consist of all
   the elements $x$ in $G$ such that some power of $x$ is in $H$.
   That is, $K = \{x \in G : \text{ for some integer } n > 0,\,
   x^n \in H \}$. Prove that $K$ is a subgroup of $G$.

   \noindent \solution 
   \begin{enumerate}[(i)]
      \item Suppose $x,y \in K$. Then there are two numbers $m,n$ such that
      $x^m, y^n \in H$. Since H is a group it is closed to multiplication (and
      exponentiation). Therefore, $(x^m)^n, (y^n)^m \in H$. Set q = mn. Then
      $(xy)^q = (xy)^{mn} = x^{mn}y^{mn} = (x^m)^n(y^n)^m \in H$. Therefore $xy
      \in K$. (The step $(xy)^{mn}$ to $(x^{mn}y^{mn})$ relies on commutativty.)

      \item Suppose $x \in K$. Then we know $x^n \in H$ for some $n$. Since
      $H$ is a group $(x^n)^{-1} \in H$. Therfore, $(x^{-1})^n \in H$.
      Therefore $x^{-1} \in K$.
   \end{enumerate}

   \noindent 6. Suppose $H$ and $K$ are subroupgs of $G$, and define 
   $HK$ as follows
   \[
      HK = \{ xy : x \in H \text{ and } y \in K \}
   \]
   Prove that $HK$ is a subgroup of $G$.

   \noindent \solution
   \begin{enumerate}[(i)]
      \item Suppose $a,b \in HK$. Then there are elements $x_1,x_2 \in H$
      and $y_1,y_2 \in K$ such that $a=x_1y_1$ and $b=x_2y_2$.
      Therefore, $ab = x_1y_1x_2y_2$ which equals $(x_1x_2)(y_1y_2)$
      by commutativity. Since $H$ and $G$ are groups 
      $x_1x_2 \in H$ and $y_1y_2 \in G$. Therefore $x_1x_2y_1y_2 \in HK$.

      \item Suppose $xy \in HK$. Therefore $x,x^{-1} \in H$ and
      $y,y^{-1} \in G$. Therefore $x^{-1}y^{-1} in HK$. 
      But $x^{-1}y^{-1} = (yx)^{-1} = (xy)^{-1} \in HK$.
   \end{enumerate}

   \noindent 7. Explain why parts 4--6 are not true if $G$ is not abelian.
   \noindent \solution Each of the proof relied on re-ordering factors
   (using commutativity) to prove an identity. Without those steps
   the proof cannot be made.

   \item \textsc{Subgroups of an Arbitrary Group}

   \noindent Let $G$ be a group.

   \noindent 1. If $H$ and $K$ are subgroups of a group $G$, 
   prove that $H \cap K$ is a subgroup of $G$. (Remember, that
   $x\in H\cap K \text{ iff } x \in H \text{ and } x \in K$.)

   \noindent \solution 
   \begin{enumerate}[(i)]
      \item Suppose $x,y \in H \cap K$. Therefore, $x,y \in H$ and
      $x,y\in K$. Since $H$ and $K$ are both groups we know
      $xy \in H$ and $xy \in K$. So $xy \in H \cap K$.

      \item Suppose $x \in H \cap K$. Therefore $x$ is in both $H,K$. 
      Therefore $x^{-1}$ is in both $H,K$; and so $x^{-1} \in H \cap K$.
   \end{enumerate}

   \noindent 2. Let $H$ and $K$ be subgroups of $G$. Prove that
   if $H \subseteq K$, then $H$ is a subgroup of $K$.

   \noindent \solution Um, this seems like its true by definition.
   We know that $H$ is a group and that it's a subset of $K$.
   Therefore $H$ is a subgroup of $K$. Right? I mean $H$ is
   closed with respect to it's operation since it's a group. And
   it's closed with respect to inverses since it's a group. And
   it's a subset of $K$. That's the definition in paragraph 1 of
   this chapter.

   \noindent 3. By the \emph{center} of a group $G$ we mean the set of
   all the elements of $G$ which commute with every element of $G$,
   that is,
   \[
       C = \{ a\in G : ax = xa \text{ for every } x \in G\}
   \]
   Prove that $C$ is a subgroup of $G$.
   
   \noindent \solution 
   \begin{enumerate}[(i)]
      \item Suppose $a,b \in C$. This means $ax=xa$ and $bx=xb$ for every $x
      \in G$.  Choose any $x\in G$:
      \begin{align*}
         abx & = axb && \text{Given $bx=xb$} \\
	     & = xab && \text{Given $ax=xa$} 
      \end{align*}
      So $ab \in C$.

      \item Suppose $a \in C$. This means $ax=xa$ for every $x\in G$.
      Choose any $x\in G$.
      \begin{align*}
         ax & = xa \\
	 axa^{-1} & = xaa^{-1} \\
	 axa^{-1} & = x \\
	 a^{-1}axa^{-1} & = a^{-1}x \\
	 xa^{-1} & = a^{-1}x
      \end{align*}
      Therefore $a^{-1} \in C$.
   \end{enumerate}

   \noindent 4. Let $C' = \{ a\in G : (ax)^2 = (xa)^2 
   \text{ for every } x \in G\}$.  Prove that $C'$ is a subgroup of $G$.

   \noindent \solution
   \begin{enumerate}[(i)]
      \item Suppose $a,b \in C'$. Then $(ax)^2=(xa)^2$ and $(bx)^2=(xb)^2$
      for every $x \in G$. Choose any $x \in G$ and recall $ab \in G$.
      \begin{align*}
         (abx)^2 & = abxabx \\
	         & = axbaxb && \text{commutativity} \\
		 & = xabxab && \text{commutativity} \\
		 & = (xab)^2
      \end{align*}
      Therefore $ab \in C'$.

      \item Suppose $a \in C'$. Then $ax=xa$ for all $x \in G$.
      \begin{align*}
         (ax)^2 & = (xa)^2 \\
	 axax   & = xaxa \\
	 aaxx   & = xxaa \\
	 aaxxa^{-1}a^{-1} & = xx\\
	 xxa^{-1}a^{-1} & = a^{-1}a^{-1}xx \\
	 xa^{-1}xa^{-1} & = a^{-1}xa^{-1}x \\
	 (xa^{-1})^2    & = (a^{-1}x)^2
      \end{align*}
      Therefore $a^{-1} \in C'$.
   \end{enumerate}

   \noindent 5. Let $G$ be a \emph{finite} group, and let S be a nonempty 
   subset of $G$. Suppose $S$ is closed with respect to 
   multiplication. Prove that $S$ is a subgroup of $G$. (\textsc{Hint:} It
   remains to prove that $S$ contains $e$ and is closed with respect
   to inverses. Let $S=\{a_1,\ldots,a_n\}$. If $a_i \in S$, consider
   the \emph{distinct} elements $a_ia_1,a_ia_2,\ldots,a_ia_n$.)

   \noindent \solution
   \begin{proof}
   Suppose $S=\{a_1,\ldots,a_n\}$. And choose some $a_i \in S$. Consider
   the elements $a_ia_1,a_ia_2,\ldots,a_ia_n$. Assume those elements
   are not distinct. Then there is some $m,n$ such that $a_ia_m=a_ia_n$ 
   which implies $a_m=a_n$. But we assumed the original elements were
   all distinct. Therefore the elements $a_ia_1,a_ia_2,\ldots,a_ia_n$
   must all be distinct. Let's call that set $S'$.
   Since the size of $S$ and size of $S'$ are the same then one of the
   elements $a_m$ must have mapped $a_ia_m$ to $a_i$, and therefore
   $a_m=e$. Since there is an inverse, and again since both sets are
   the same size there must be an inverse for each element.
   \end{proof}

\end{enumerate}

\end{document}
