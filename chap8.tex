\documentclass[twoside]{amsart}
\usepackage{amssymb,latexsym}
\usepackage{amsfonts}
\usepackage{xspace}
\usepackage{enumerate}
\usepackage{graphics}
\usepackage{fitch}
\newcommand{\Rationals}{\mathbb{Q}{}}
\newcommand{\Reals}{\ensuremath{\mathbb{R}}\xspace}
\newcommand{\Integers}{\ensuremath{\mathbb{Z}{}}\xspace}
\newcommand{\solution}{\textsc{Solution}\xspace}
\newcommand{\problem}{\textsc{Problem}\xspace}
\newcommand{\Blank}{\mathrel{\phantom{=}}}
\newcommand{\ltrue}{\top}
\newcommand{\lfalse}{\bot}
\newcommand{\fOfg}{\ensuremath{f \circ g}\xspace}
\newcommand{\gOff}{\ensuremath{g \circ f}\xspace}
\newcommand{\eps}{\ensuremath{\epsilon}\xspace}
\begin{document}
\title{Answers to Chapter 8 Exercises - A Book of Abstract Algebra}
\author{Michael Welch}
\date{\today}
\maketitle

This document contains selected answers to exercises from chapter 8
of A Book of Abstract Algebra.


\begin{enumerate}[A.]
   \item \textsc{Practice in Multiplying and Factoring Permutations}
   \begin{enumerate}[1]
      \item Compute each of the following products in $S_9$. (Write your
      answer as a single permutation.)

      \begin{enumerate}[(a)]
         \item (145)(37)(682)

	 \solution \[ \begin{pmatrix}
	                 1 & 2 & 3 & 4 & 5 & 6 & 7 & 8 & 9 \\
			 4 & 6 & 7 & 5 & 1 & 8 & 3 & 2 & 9
	              \end{pmatrix} \]

	 \item (17)(628)(9354)
	 
	 \solution \[
	              \begin{pmatrix}
	                 1 & 2 & 3 & 4 & 5 & 6 & 7 & 8 & 9 \\
			 7 & 8 & 5 & 9 & 4 & 2 & 1 & 6 & 3
		      \end{pmatrix}
		   \]

	 \item (71825)(36)(49)

	 \solution \[
	              \begin{pmatrix}
	                 1 & 2 & 3 & 4 & 5 & 6 & 7 & 8 & 9 \\
			 8 & 5 & 6 & 9 & 7 & 3 & 1 & 2 & 4
		      \end{pmatrix}
	           \]
	 
	 \item (12)(347)

	 \solution \[
	              \begin{pmatrix}
	                 1 & 2 & 3 & 4 & 5 & 6 & 7 & 8 & 9 \\
			 2 & 1 & 4 & 7 & 5 & 6 & 3 & 8 & 9
		      \end{pmatrix}
	           \]

	 \item (147)(1678)(74132)

	 \solution \[
	              \begin{pmatrix}
	                 1 & 2 & 3 & 4 & 5 & 6 & 7 & 8 & 9 \\
			 3 & 8 & 2 & 6 & 5 & 1 & 7 & 4 & 9
		      \end{pmatrix}
	           \]

	 \item (6148)(2345)(12493)

	 \solution \[
	              \begin{pmatrix}
	                 1 & 2 & 3 & 4 & 5 & 6 & 7 & 8 & 9 \\
			 3 & 5 & 4 & 9 & 2 & 1 & 7 & 6 & 8
		      \end{pmatrix}
	           \]
      \end{enumerate}

      \item Write each of the following permutations in $S_9$ as a product
      of disjoint cycles.

      \begin{enumerate}[(a)]
         \item \[
	          \begin{pmatrix}
		      1 & 2 & 3 & 4 & 5 & 6 & 7 & 8 & 9 \\
		      4 & 9 & 2 & 5 & 1 & 7 & 6 & 8 & 3
		  \end{pmatrix}
	       \]

	 \solution (145)(293)(67)

         \item \[
	          \begin{pmatrix}
		      1 & 2 & 3 & 4 & 5 & 6 & 7 & 8 & 9 \\
		      7 & 4 & 9 & 2 & 3 & 8 & 1 & 6 & 5
		  \end{pmatrix}
	       \]

	 \solution (17)(24)(395)(68)
	 
         \item \[
	          \begin{pmatrix}
		      1 & 2 & 3 & 4 & 5 & 6 & 7 & 8 & 9 \\
		      7 & 9 & 5 & 3 & 1 & 2 & 4 & 8 & 6
		  \end{pmatrix}
	       \]

	 \solution (17435)(296)

      \end{enumerate}

      \item Express each of the following as a product of transpostions
      in $S_8$.

      \begin{enumerate}[(a)]
         \item (137428) = (18)(12)(14)(17)(13)

	 \item (416)(8235) = (46)(41)(85)(83)(82)

	 \item (123)(456)(1574) = (13)(12)(46)(45)(14)(17)(15)

	 \item $\pi = \begin{pmatrix}
	                 1 & 2 & 3 & 4 & 5 & 6 & 7 & 8 \\
			 3 & 1 & 4 & 2 & 8 & 7 & 6 & 5
	              \end{pmatrix}$ = (1342)(58)(67) which
		      can be rewritten (12)(14)(13)(58)(67)
      \end{enumerate}

      \item If $\alpha = (3714)$, $\beta = (123)$, and $\gamma = (24135)$
      in $S_7$, express each of the following as a product of disjoint cycles:

      \begin{enumerate}[(a)]
         \item $\alpha^{-1}\beta$

	 \solution First of all $\alpha = 
	    \begin{pmatrix}
	       1 & 2 & 3 & 4 & 5 & 6 & 7 \\
	       4 & 2 & 7 & 3 & 5 & 6 & 1
	    \end{pmatrix}$ and therefore $\alpha^{-1} = 
	    \begin{pmatrix}
	       1 & 2 & 3 & 4 & 5 & 6 & 7 \\
	       7 & 2 & 4 & 1 & 5 & 6 & 3
	    \end{pmatrix}$, which can be rewritten as (4173). So notice
	    that to find the inverse of a permutation we just need to arrange
	    the cycle in reverse order. Now we need to calculate
	    $\alpha^{-1}\beta$: (4173)(123)

	    \[ \alpha^{-1}\beta = 
	       \begin{pmatrix}
	          1 & 2 & 3 & 4 & 5 & 6 & 7 \\
		  2 & 4 & 7 & 1 & 5 & 6 & 3
	       \end{pmatrix}
	    \]

	    Now we can find disjoint cycles: (124)(37)

	    \item $\gamma^{-1}\alpha = (53142)(3714)$
	    This can be rewritten as (125374).

	    \item $\alpha^2\beta = (3714)(3714)(123)$.
	    This can be rewritten as (12)(47)

	    \item $\beta^2\alpha\gamma = (123)(123)(3714)(24135)$.
	    This can be rewritten as (1735)

	    \item $\gamma^4 = (24135)(24135)(24135)(24135)$.
	    This can be rewritten as (14253).

	    \item $\gamma^3\alpha^{-1}$ = (24135)(24135)(24135)(4173).
	    This can be rewritten as $(12345)\circ(4173) = (174235)$.

	    \item $\beta^{-1}\gamma = (31)(24135)$.
	    This can be rewritten as (2435).

	    \item $\alpha^{-1}\gamma^2\alpha = (4173)(21543)(3714)$.
	    This can be rewritten as (14275).
      \end{enumerate}

      \item In $S_5$, write (12345) in five different ways as a cycle, 
      and in five different ways as a product of transpositions.

      \solution The cycles are (12345), (23451), (34512), (45123),
      (51234). The products are (15)(14)(13)(12), (21)(25)(24)(23),
      (32)(31)(35)(34), (43)(42)(41)(45), (54)(53)(52)(51).

      \item In $S_6$, express each of the following as a square of a 
      cycle (that is, express as $\alpha^2$ where $\alpha$ is a cycle):

      \begin{enumerate}[(a)]
         \item (132). This is the square of (123).
	 \item (12345). This is the square of (14253).
	 \item (13)(24). This is the square of (1234).
      \end{enumerate}


   \end{enumerate}

   \item \textsc{Powers of Permutations}

   \noindent If $\pi$ is a permutation, we write $\pi \circ \pi = \pi^2$,
   $\pi \circ \pi \circ \pi = pi^3$, etc. The convenience of this notation
   is evident.

   \begin{enumerate}[1]
      \item Compute $\alpha^{-1}, \alpha^2, \alpha^3, \alpha^4, \alpha^5$
      where
      \begin{enumerate}[(a)]
         \item $\alpha=(123)$: $\alpha^{-1}=(321)$, $\alpha^2=(132)=
	 \alpha^{-1}$,
	 $\alpha^3=\eps$, $\alpha^4=\alpha$, $\alpha^5=\alpha^2$.

	 \item $\alpha=(1234)$: $\alpha^{-1}=(4321)$, $\alpha^2=(13)(24)$,
	 $\alpha^3=(1432)=\alpha^{-1}$, $\alpha^4=\eps$, $\alpha^5=\alpha$.

	 \item $\alpha=(123456)$: $\alpha^{-1}=(654321)$ 
	 $\alpha^2=(135)(246)$, $\alpha^3=(14)(25)(36)$, $\alpha^4=
	 (153)(264)$, $\alpha^5=(165432)=\alpha^{-1}$
      \end{enumerate}

      \vspace{5pt}
      \noindent In the following problems, let $\alpha$ be a cycle
      of length $s$, say $\alpha = (a_1a_2\dots a_s)$.
      \vspace{5pt}

      \item Describe all the \emph{distinct} powers of $\alpha$. How
      many are there? Note carefully the connection with addition of
      integers modulo $s$ (page 27).

      \noindent \solution I think the answer is there are $s$ \emph{distinct}
      powers of $\alpha$: $\alpha$ and \eps are always included. So if 
      $s$ is 2 then that's all you have.
      \begin{proof}
      The first power of $\alpha$ is $\alpha$, it takes $a_1$ to $a_2$. 
      The second power of $\alpha$
      is going to take $a_1$ to $a_3$. The third power of $\alpha$ is going
      to take $a_1$ to $a_4$. The $(s-1)^{th}$ power of $\alpha$ is going to
      take $a_1$ to $a_{1+s-1} = a_s$. Finally the $s^{th}$ power of $\alpha$
      will take $a_1$ to $a_1$. So by this we have shown that there must
      be at least $s$ \emph{distinct} powers of $\alpha$.

      However we can expand this by looking at what happens to all of 
      the elements for each power.

      \begin{align*}
          \alpha &= \begin{pmatrix}
	               a_1 & a_2 & a_3 & \cdots & a_{s-1} & a_s \\
		       a_2 & a_3 & a_4 & \cdots & a_s & a_{s-1}
	            \end{pmatrix} \\
	  \alpha^2 &= \begin{pmatrix}
	                 a_1 & a_2 & a_3 & \cdots & a_{s-1} & a_s \\
		         a_3 & a_4 & a_5 & \cdots & a_1 & a_{2}
	              \end{pmatrix} \\
		   &\vdots  \\
	  \alpha^{s-1} &= \begin{pmatrix}
	                      a_1 & a_2 & a_3 & \cdots & a_{s-1} & a_s \\
		              a_s & a_1 & a_2 & \cdots & a_{s-2}& a_{s-1}
	                  \end{pmatrix} \\
	  \alpha^s &= \begin{pmatrix}
	                 a_1 & a_2 & a_3 & \cdots & a_{s-1} & a_s \\
		         a_1 & a_2 & a_3 & \cdots & a_{s-1} & a_s
	              \end{pmatrix}		  
      \end{align*}
      As you can see there are $s$ distinct powers. 

      The connection with addition module $s$ is easier to see if we
      define $\alpha$ to be $(a_0 a_1 a_2 \dots a_{s-2} a_{s-1}$. As 
      you can see $\alpha$ is still lenght $s$. Each power $n$ of $\alpha$
      has the affect of taking some $a_i$ to $a_j$ where $j=i+n$ modulo $s$.

      If we relabel $a_s$ as $a_0$ this is exactly modulo arithmetic and
      $a_s$ even shows up in the same position that $a_0$ normally would.
      \end{proof}

      \item Find the inverse of $\alpha$ and show that $\alpha^{-1} =
      \alpha^{s-1}$. 

      \noindent \solution $\alpha^{-1} = (a_s a_{s-1} a_{s-2} \dots a_3 a_2 
      a_1)$.
      So we can see that this maps every element 
      $a_i$ to $a_j$ where $j=i-1$ (modulo s, so that $a_1$ is taken to
      $a_s$). And we know that the $(s-1)^{th}$ power moves every 
      $a_i$ to $a_j$ where $j=i+s-1=i-1$.

      \noindent Prove each of the following:

      \item $\alpha^2$ is a cycle iff $s$ is odd.

      \noindent \solution. Assume that $s$ is odd. Then we have that
      every element $a_i$ for $i+2 <= s$ is taken to 
      $a_{i+2}$. 
      This account for all of
      the elements except for $a_{s-1}$ and $a_s$. These are taken
      to $a_1$ and $a_2$ respectively. So we have the cycle
      $(a_1 a_3 a_5 \dots a_{s-2} a_s a_2 a_4 \dots a_{s-3} a_{s-1})$.

      However if $s$ is even then we get the composition of
      two cycles $(a_1 a_3 \dots a_{s-3} a_{s-1})\circ (a_2 a_4 a_6 \dots
      a_{s-2} a_s$).

      So next assume that $\alpha^2$ is a cycle. We know that if $s$
      is even then $\alpha^2$ is not a cycle. Therefore $s$ must be odd.

      \item If $s$ is odd, $\alpha$ is the square of some cycle of 
      length $s$. (Find it. \textsc{Hint}: Show $\alpha = \alpha^{s+1}$.

      \noindent \solution We know that $\alpha = \alpha^{s+1}$ as we've
      repeatedly seen evidence of this. We know that $\alpha^s$ takes
      $a_1$ to $a_1$ and in general takes $a_i$ to $a_i$. So $\alpha^s=
      eps$. Therefore, $\alpha^{s+1} = \alpha^s \alpha = \eps \alpha 
      = \alpha$.

      \hspace{0.15in} Since $s$ is odd we can rewrite it in terms of some $t$ as
      $2t+1$ (where $t=(s-1)/2$) and we can rewrite $s+1$ as $2t+2$.
      So we can rewrite $\alpha$ as $\alpha^{s+1}=\alpha^{2t+2}
      =\alpha^{t+1}\alpha^{t+1}$. So $\alpha$ can be rewritten
      as $\alpha = \beta^2$ where $\beta=\alpha^{t+1}=\alpha^{((s-1)/2)+1)}$.

      \item If $s$ is even, say $s=2t$, then $\alpha^2$ is the product
      of two cycles of length $t$. (Find them.)

      \noindent \solution We know thta $\alpha^2$ takes
      $a_i$ to $a_j$ where $j=i+2$ modulo $s$. With this information it's
      possible to identify the two cycles as
      $(a_1 a_3 a_5 \dots a_{s-3} a_{s-1})$ and 
      $(a_2 a_4 a_6 \dots a_{s-2} a_{s})$.

      \item If $s$ is a multiple of $k$, say $s=kt$, then $\alpha^k$ is
      the product of $k$ cycles of length $t$.

      \noindent \solution Proceed as in part 6. We know that
      $a_i$ is taken to $a_j$ where $j=i+k$ module $s$. Therefore
      we will have $k$ cycles: $(a_1 a_{1+k} \dots a_{s-k+1})$,
      $(a_2 a_{2+k} \dots a_{s-k+2})$, $\ldots$, 
      $(a_k a_{k+k} \dots a_s)$.

      \item If $s$ is a prime number, every power of $\alpha$ is a cycle.

      \noindent \solution Let's assume that there exists an $n$ such that
      when $s$ is prime $\alpha^n$ is the product of $m$ distinct cycles.
      Then the cycles would look like

      \begin{tabular}{c}
      $(a_1 a_{1+n} a_{1+2n} \dots a_{s-2n+1} a_{s-n+1})$ \\
      $(a_2 a_{2+n} a_{2+2n} \dots a_{s-2n+2} a_{s-n+2})$ \\
      $\vdots$ \\
      $(a_m a_{m+n} a_{m+2n} \dots a_{s-2n+m} a_{s-n+m})$
      \end{tabular}

      And we would end up with $m$ cycles each of length $s/m$. But
      this contradicts are assumption that $s$ is prime and therefore
      not divisible by $m$. Therefore $m$ must be 1.

   \end{enumerate}

   \item \textsc{Even and Odd Permutations}

   \begin{enumerate}[1]
       \item Determine which of the following permutations is even, and
       which is odd.

       \begin{enumerate}[(a)]
          \item $\pi = \begin{pmatrix}
	                  1 & 2 & 3 & 4 & 5 & 6 & 7 & 8 \\
			  7 & 4 & 1 & 5 & 6 & 2 & 3 & 8
	               \end{pmatrix}$

	  \noindent \solution $\pi = (173)(2456) = (13)(17)(26)(25)(24)$
	  therefore $\pi$ is odd.

	  \item (71864) \solution This is equivalent to (74)(76)(78)(71), thus
	  it is even.

	  \item (12)(76)(345) \solution Even: (12)(76)(35)(34).

	  \item (1276)(3241)(7812)
	  \noindent \solution Odd: (16)(17)(12)(31)(34)(32)(72)(71)(78)

	  \item (123)(2345)(1357) \solution even + odd + odd = even:
	  (13) (12) (25) (24) (23) (17) (15) (13).

       \end{enumerate}
       \noindent Prove each of the following

       \item \begin{enumerate}[(a)]
	  \item The product of two even permutations is even.

	  \noindent \solution Assume we have two even permutations. Then
	  we know that we can write them as the product of $2m$ and $2n$
	  tranpositions for some $m,n >= 0$. Therefore the product of 
	  these two permutations can be written as the product of $2m 
	  + 2n = 2(m+n)$ transpositions. Therefore the product is even.

	  \item The product of two odd permutations is even.

	  \noindent \solution The two odd permutations can be written
	  as the product of $2m+1$ and $2n+1$ transpotitions for $m,n>=0$.
	  The product of these permutations can therefore be written
	  as the product of $(2m+1)+(2n+1)= 2m+2n+2 = 2(m+n+1)$ transpositions.
	  Therefore the product is even.

	  \item The product of an even permutation and an odd permutation
	  is odd.

	  \noindent \solution The two permutations can be written as
	  the product of $2m$ tranpositions and $2n+1$ transpositions
	  respectively. Therefore their product can be written
	  as the product of $2m+2n+1=2(m+n) + 1$ transpositions which
	  is an odd number. Therefore, the product is odd.
	  
       \end{enumerate}

       \item 
       \begin{enumerate}[(a)]

          \item A cycle of length $l$ is even if $l$ is odd.

	  \noindent \solution As shown on p83 we have the following
	  equality:
	  \[
	      (a_1a_2\cdots a_r) = (a_ra_{r-1})(a_ra_{r-2})\cdots
	                          (a_ra_3)(a_ra_2)(a_ra_1)
	  \]

	  We can see that a cycle of length $r$ can be written
	  as the product of $r-1$ transpositions. Therefore, if
	  the cycle has a length that is odd the cycle will be even 
	  (e.g. If the cycle has a length $l$ that is odd, then $l$
	  can be written as $2m+1$ for some $m$. This cycle can
	  then be written as the product of $l-1=(2m+1)-1=2m$ 
	  transpositions which is even).

	  \item A cycle of length $l$ is odd if $l$ is even.

	  \noindent \solution Same solution as part 1 except starting
	  with $l$ is even.
       \end{enumerate}

       \item 
       \begin{enumerate}[(a)]
          \item If $\alpha$ and $\beta$ are cycles of length $l$ and $m$,
	  respectively, then $\alpha \beta$ is even or odd depending
	  on whether $l+m-2$ is even or odd.

	  \noindent \solution $\alpha$ and $\beta$ can each be written
	  as the product of $l-1$ and $m-1$ transpositions, respectively.
	  So their product can be written as the product of $(l-1)+(m-1)
	  = l+m-2$ transpositions. Therefore if this number is even the
	  product is even, if it is odd the product is odd.

	  \item If $\pi = \beta_1 \cdots \beta_r$, where each $\beta_i$
	  is a cycle of length $l_i$, then $\pi$ is even or odd
	  depending on whether $l_1 + l_2 + \dots + l_r - r$ is even or odd.

	  \noindent \solution Proceed as in the previous part. Each
	  $\beta_i$ can be written as the product of $l_i - 1$ transpositions.
	  Therefore $\pi$ is written as the product of $q$ transpositions
	  where $q$ is calculated as:
	  \[
	     q = \sum_{i=1}^{r} (l_i - 1) = l_1 + l_2 + \dots + l_r - r
	  \]

       \end{enumerate}


       

   \end{enumerate}

   \item \textsc{Disjoint Cycles}

   \noindent In each of the following, let $\alpha$ and $\beta$ be disjoint
   cycles, say

   \begin{align*}
      \alpha &= (a_1a_2 \dots a_s) & && \beta &= (b_1b_2\dots b_r)
   \end{align*}

   \noindent Prove parts 1-3:

   \begin{enumerate}[1]
      \item For every positive integer $n$, $(\alpha \beta)^n =
      \alpha^n\beta^n$.

      \noindent \solution We can expand $(\alpha \beta)^n$:

      \begin{align*}
         (\alpha \beta)^n &= \underbrace{(\alpha \beta)(\alpha \beta) \cdots
	     (\alpha \beta)}_{n \text{ times}}
      \end{align*}

      Now, since disjoint cycles are commutative we can rearrange all
      of the terms to be

      \begin{align*}
         (\alpha \beta)^n &= \underbrace{\alpha \alpha \cdots 
	    \alpha}_{n \text{ times}} \underbrace{\beta \beta \cdots 
	    \beta}_{n \text{ times}} \\
		          &= \alpha^n \beta ^n
      \end{align*}

      \item If $\alpha \beta = \eps$, then $\alpha = \eps$ and
      $\beta = \eps$.

      \noindent \solution See the discussion of disjoint cycles on p82.
      We know that $\alpha$ moves the $a$'s while $\beta$ moves the $b$'s.
      Assume that $\alpha$ is not $\eps$. Then there is some $h$ and
      some $k$ such that $\alpha$ moves $a_h$ to $a_k$. And therefore
      $\alpha \beta$ does the same. But we know that $\alpha \beta = \eps$
      which is contradicted by our assumption. Therefore $\alpha=\eps$.
      By the same argument $\beta=\eps$.

      \item If $(\alpha \beta)^t = \eps$, then $\alpha^t = \eps$ and
      $\beta^t = \eps$.

      \noindent \solution First, we can use part 1 to note that
      $(\alpha \beta)^t = \alpha^t \beta^t$. Let $\gamma = \alpha^t$ and
      $\psi = \beta^t$. Then we have the fact that $\gamma \psi = \eps$.
      By part 2 we have the fact that $\gamma = \eps$ and $\psi = \eps$.
      Therefore $\alpha^t = \eps$ and $\beta^t = \eps$.

      \item Find a transposition $\gamma$ such that $\alpha \beta \gamma$ is
      a cycle.

      \noindent \solution Choose some $a_i$ and some $b_j$ and let
      $\gamma = (a_ib_j)$. To illustrate let's set $\gamma = (a_1b_1)$.
      Then $\alpha \beta \gamma = b_1 a_2 a_3 \cdots a_s a_1 b_2 b_3
      \cdots b_r)$.

      \item Let $\gamma$ be the same transposition as in the preceeding 
      exercise. Show that $\alpha \gamma \beta$ and $\gamma \alpha \beta$
      are cycles.

      \noindent \solution $\alpha \gamma \beta =
      (a_1 a_2 \cdots a_s)(a_1b_1)(b_1 b_2 \cdots b_t)$ which can be
      expanded to $(a_1 b_1 b_2 \cdots b_t a_2 \cdots a_s)$.

      \noindent \solution $\gamma \alpha \beta = (a_1b_1)(a_1 a_2 \cdots a_s)
      (b_1 b_2 \cdots b_t)$ which can be expanded to 
      $(a_1 a_2 \cdots a_s b_1 b_2 \cdots b_t)$.

      \item Let $\alpha$ and $\beta$ be cycles of odd length (not disjoint).
      Prove that if $\alpha^2 = \beta^2$, then $\alpha = \beta$.

      \noindent \solution Trivial. Prove by contradiction. Assume 
      $\alpha \ne \beta$.

   \end{enumerate}

   \item \textsc{Conjugate Cycles}

   \noindent Prove each of the following in $S_n$:

   \begin{enumerate}[1]
       \item Let $\alpha = (a_1, \ldots , a_s)$ be a cycle and let $\pi$
       be a permutation in $S_n$. Then $\pi \alpha \pi^{-1}$ is the cycle
       $(\pi (a_1), \ldots , \pi (a_s))$.

       \noindent \solution Trivial actually, once you see the pattern.
       Let's see what $\pi \alpha \pi^{-1} (\pi(a_1))$ equals. Let's
       rewrite this so it's clear what is going on: 
       $\pi(\alpha(\pi^{-1}(\pi(a_1))))$. This is easily reduced because
       the innermost term is applying $\pi$ to $a_1$ but right after
       that we undo the result by applying $\pi^{-1}$ which leaves us
       back with $a_1$. So we've reduced the equation to $\pi(\alpha(a_1))$.
       This easily reduces (by defn of $\alpha$) to $\pi(a_2)$. To recap:
       \begin{align*}
           \pi \alpha \pi^{-1} (\pi(a_1)) &= \pi(\alpha(\pi^{-1}(\pi(a_1)))) \\
                   &= \pi(\alpha(a_1)) \\
                   &= \pi(a_2)
       \end{align*}

       So using the same technique we find out that 
       $\pi \alpha \pi^{-1}(\pi(a_2))=\pi(a_3)$ and so on all the way
       to $\pi \alpha \pi^{-1}(\pi(a_s))=\pi(a_1)$. This gives us the cycle
       $(\pi(a_1),\dots,\pi(a_s))$.

\vspace{5pt}
       \noindent If $\alpha$ is any cycle and $\pi$ is any permutation,
       $\pi\alpha\pi^{-1}$ is called a \emph{conjugate} of $\alpha$. In
       the following parts, let $\pi$ denote any permutation in $S_n$.
\vspace{5pt}

       \item Conclude from part 1: Any two cycles of the same length are 
       conjugates of each other.

       \noindent \solution Let $\alpha=(a_1,\dots,a_s)$ and $\beta=
       (b_1,\dots,b_s)$ be two cycles of the same length, $s$. If $\beta$
       is a conjugate of $\alpha$ then we should be able to find some
       permutation $\pi$ such that $\beta=\pi \alpha \pi^{-1}$. This
       is actually trivial. Select $\pi$ to be the following:

       \begin{center}
       \begin{tabular}{cc}
           $x$ & $\pi(x)$ \\
           $a_1$ & $b_1$ \\
           $a_2$ & $b_2$ \\
           \vdots & \vdots \\
           $a_s$ & $b_s$ \\
       \end{tabular}
       \end{center}

       Now we have $\pi \alpha \pi^{-1}$ is the cycle (by part 1):
       $(\pi(a_1), \pi(a_2), \dots, \pi(a_s)) = (b_1, b_2, \dots, b_s) = 
       \beta$.
       Therefore we have shown that $\beta$ is a conjugate of $\alpha$.

       \item If $\alpha$ and $\beta$ are disjoint cycles, then
       $\pi\alpha\pi^{-1}$ and $\pi\beta\pi{-1}$ are disjoint cycles.

       \noindent \solution First let's expand the terms:
       $\alpha'=\pi\alpha\pi^{-1}=(a'_1,\dots,a'_s)=(\pi(a_1), \dots,
       \pi(a_s))$ and $\beta'=\pi\beta\pi^{-1}=(b'_1,\dots,b'_s)=(\pi(b_1),
       \dots, \pi(b_s))$.  Assume that $i^{th}$ element in $\alpha'$ is the
       same value as the $j^{th}$ element in $\beta'$. Then we'd have the fact
       that $a'_i=b'_j=\pi(a_i)=\pi(b_j)$. But for that to be the case we'd
       have to have the fact that $a_i=b_j$. But this contradicts our
       assumption that $\alpha$ and $\beta$ are disjoint. We conclude that
       $\alpha'$ and $\beta'$ must also be disjoint.

       \item Let $\sigma$ be a product $a_1 \cdots a_t$ of $t$ disjoint cycles
       of lengths $l_1,\dots,l_t$, respectively. Them $\pi\sigma\pi^{-1}$
       is also a product of $t$ disjoint cycles of lengths $l_1,\dots,l_t$.

       \noindent \solution Also, trivial. Can be seen by inspection.
       $\pi\sigma\pi^{-1}=\pi a_1 a_2 \cdots a_t \pi^{-1}$ There will
       be $l_1$ values from the output of $\pi^{-1}$ that are in $a_1$.
       When we feed the output of $a_1$ thru $\pi$ we will get a new
       cycle of length $l_1$.
       Likewise there will be $l_i$ values from the output of $\pi^{-1}$
       that are in the cycle $a_i$. And if you feed these values thru
       $\pi$ you'll get a new cycle of length $l_i$. And since the inputs
       to $\pi$ from each of these cycles is disjoint, the output must
       be as well. Therefore, we can see we'll have the same number of
       disjoint cycles with the same lengths. Also note that some
       values in the output of $\pi^{-1}$ may not be in any of the cycles of
       $\sigma$ which means they are left unchanged by $\sigma$. Moreover,
       these values will then be fed direclty into $\pi$ which will
       return the original values. So the values that are not in the
       cycles of $\sigma$ are unchanged by $\pi\sigma\pi^{-1}$.

       \item Let $\alpha_1$ and $\alpha_2$ be cycles of the same length.
       Let $\beta_1$ and $\beta_2$ be cycles of the same length. Let $\alpha_1$
       and $\beta_1$ be disjoint, and let $\alpha_2$ and $\beta_2$ be 
       disjoint. There is a permutation $\pi \in S_n$ such that
       $\alpha_1 \beta_1 = \pi \alpha_2 \beta_2 \pi^{-1}$.

       \noindent \solution Let us define $\alpha_1 = (a_{1,1}, a_{1,2}, 
       \ldots , a_{1,s_\alpha})$, $\alpha_2 = (a_{2,1}, a_{2,2},
       \ldots, a_{2,s_\alpha})$, $\beta_1 = (b_{1,1}, \ldots , 
       b_{1,s_\beta})$, and $\beta_2 = (b_{2,1}, \ldots, b_{2,s_\beta})$.
       We will pick $\pi$ to be the following:

       \begin{center}
       \begin{tabular}{c|c}
          $x$              &   $\pi(x)$         \\ \hline
          $a_{2,1}$        &   $a_{1,1}$        \\
          $\vdots$         &   $\vdots$         \\
          $a_{2,s_\alpha}$ &   $a_{1,s_\alpha}$ \\
          $b_{2,1}$        &   $b_{1,1}$        \\
          $\vdots$         &   $\vdots$         \\
          $b_{2,s_\beta}$ &   $b_{1,s_\beta}$ \\
       \end{tabular}
       \end{center}

       Now let's see what happens when we calculate
       $\pi\alpha_2\beta_2\pi^{-1}$ of different values starting
       with $\alpha_{1,1}$. 

       \begin{align*}
           \pi\alpha_2\beta_2\pi^{-1}(a_{1,1}) &= 
                   \pi\alpha_2\beta_2(a_{2,1}) \\
               &=  \pi\alpha_2(a_{2,1}) && \text{$\beta_2$ doesn't affect any
                        $\alpha_2$ values.}\\
               &=  \pi(a_{2,2})          \\
               &=  a_{1,2}
       \end{align*}

       Using the same process we find that $\pi\alpha_2\beta_2
       \pi^{-1}(a_{1,2}) = a_{1,3}$. If we continue with this
       we'll see that we get the cycle $\alpha_1$. Next we pick
       $b_{1,1}$ and get out $b_{1,2}$ and we see that we get the
       cycle $\beta_1$. Any values that are not in any of the cycles
       will go thru the function unchanged. So we demonstrate
       that there is a $\pi$ to make the equation true.

   \end{enumerate}


   \item \textsc{Order of Cycles}

   \begin{enumerate}[1]

      \item Prove in $S_n$: if $\alpha = (a_1, \ldots, a_s)$ is a cycle
      of length $s$, then $\alpha^s = \epsilon$, $\alpha^{2s}=\epsilon$,
      and $\alpha^{3s}=\epsilon$. Is $\alpha^k = \epsilon$ for any positive
      integer $k < s$? (Explain.)

      \noindent \solution We've proven this before. It's obvious. If
      you apply $\alpha$ to $a_i$ you get $a_{i+1}$. If you apply it
      $s$ times you get $a_i$. If you apply $\alpha$ to $a_i$ $s$ times you
      get $a_i$. So $\alpha^s=\alpha^{2s}=\alpha^{3}=\epsilon$.

\vspace{10pt}
      If $\alpha$ is any permutation, the lease positive integer $n$
      such that $\alpha^n=\epsilon$ is called the \emph{order} of
      $\alpha$.
\vspace{10pt}

      \item  Prove in $S_n$: If $\alpha=(a_1,\ldots,a_s)$ is any cycle 
      of length $s$, the order of $\alpha$ is $s$.

      \noindent \solution By Part 1 we know that $\alpha^s=\epsilon$, therefore
      we know the order of $\alpha$ is less than or equal to $s$. If we
      pick any $n$ such that $n<s$ we get a function that takes $a_i$ to
      $a_{i+n}$ which is not equal to $\epsilon$. Therefore the order is $s$.

      \item Find the order of each of the following permutations:

      \begin{enumerate}[(a)]
          \item (12)(345)

          \noindent \solution The answer is $2*3=6$. This can be seen by 
          noticing that every power of 2 causes the first cycle to 
          become $\epsilon$ for the values of $1$ and $2$. The same
          happens for every power of 3 and the second cycle. The 
          least common multiple of 2 and 3 is 6. Also the two cycles
          are disjoint.

          \item (12)(3456) 
          \noindent \solution The answer is 4.

          \item (1234)(56789)
          \noindent \solution The answer is 20.
      \end{enumerate}

      \item What is the order of $\alpha\beta$, if $\alpha$ and $\beta$
      are disjoint cycles of lengths 4 and 6, respectively. (Explain why.
      Use the fact that disjoint cycles commute.)

      \noindent \solution The answer is 12.  Because the least common 
      multiple of 4 and 6 is 12.  Ok, to see this remember that 
      $(\alpha\beta)^12=\alpha^12\beta^12$. $alpha^12=\epsilon$ and 
      $\beta^12=\epsilon$.

      \item What is the order of $\alpha\beta$, if $\alpha$ and $\beta$
      are disjoint cycles of length $r$ and $s$, respectively.
      (Venture a guess, explain, but do not attempt a rigourous proof.)

      \noindent \solution The answer is the least common multiple of 
      $r$ and $s$, which is obvious and doesn't need a proof. 
      Hmm, I maybe am missing something. This seems pretty obvious.
   \end{enumerate}

   \item \textsc{Even/Odd Permutations in Subgroups of $S_n$}

   Prove each of the following in $S_n$:
   \begin{enumerate}[1]
       \item Let $\alpha_1,\ldots,\alpha_r$ be distinct even
       permutations, and $\beta$ an odd permutation. Then
       $\alpha_1\beta,\ldots,\alpha_r\beta$ are $r$ \emph{distinct}
       odd permutations. (See Exercise C2.)

       \noindent \solution By C2 we know each is odd. Next let's assume
       that 2 of them are the same. So we have $\alpha_i\beta
       = \alpha_j\beta$. But we can then say $\alpha_i\beta\beta^{-1}
       = \alpha_j\beta\beta^{-1}$. And this reduces to $\alpha_i=\alpha_j$.
       This contradicts the statement in the problem that
       we started with distinct permutations. Therefore, it must
       be the case that none of the resulting permutations can be the
       same.

       \item If $\beta_1,\ldots,\beta_r$ are distinct odd permutations,
       then $\beta_1\beta_1, \beta_1\beta_2, \ldots, \beta_1\beta_r$ are
       $r$ \emph{distinct} even permutations.

       \noindent \solution Again by C2 we know they are even. We'll use
       the same proof by contradiction technique. Assume they are not
       all distinct. Then we have for some $i$ and some $j$ the fact
       that $\beta_1\beta_i=\beta_1\beta_j$ and therefore
       $\beta_1^{-1}\beta_1\beta_i=\beta_1^{-1}\beta_1\beta_j$, which
       means that $\beta_i=\beta_j$ which contradicts our assumption.
       Therefore, they must all be distinct.

       \item In $S_n$, there are the same number of odd permutations as even
       permutations. (\textsc{Hint}: Use part 1 to prove that the number of
       even permutations is $\le$ the number of odd permutations.
       Use part 2 to prove the reverse of the inequality.)

       \noindent \solution Assume there are $r$ even permutations. Then the
       rest, assume there are $t$ of them, must be odd. But by part 1, we know
       that we have to have at least $r$ odd permutations. 
       Therefore $t \ge r$ or  $r \le t$.

       Assuming that we have $t$ odd permutations, then by part 2
       we know that we have at least $t$ even permutations. So we
       have $r \ge t$. So we have $r \le t$ and $r \ge t$. Therefore $r=t$.

       \item The set of all even permutations is a subgroup of $S_n$.
       (It is denoted by $A_n$ and is called the \emph{alternating group}
       on $n$ symbols.)

       \noindent \solution We must show that the subgroup $A_n$ is closed
       with respect to $\circ$ and with respect to function inverses.
       We know from C2 that the product of two even permutations is even,
       so $A_n$ is closed with respect to $\circ$. Next we use some logic
       from p85. If a permuation, $\pi$, is even then it's inverse,
       $\pi^{-1}$ must be even as well. For if it were odd, then if
       we mulitply them we would get an odd permutation (by C2) for 
       $\epsilon$. Therefore $\pi^{-1}$ must be even and therefore
       $A_n$ is closed with respect to function inverses. Therefore,
       $A_n$ is a subgroup of $S_n$.

       \item Let $H$ be any subgroup of $S_n$. $H$ either contains only
       even permutations, or $H$ contains the same number of odd as even
       permutations. (Use parts 1 and 2.)

       \noindent \solution So we know it's possible from part 4 for a subgroup
       to have only even permutations. So this proof boils down to the fact
       that if there is at least one odd permutation then there must
       be an equal number of even and odd permutations. The proof follows the
       same as \#3 except the distinct permutations used are on the
       subgroup not on the group.

   \end{enumerate}

   \item \textsc{Generators of $A_n$ and $S_n$}

   Remember that in any group $G$, a set $S$ of elements of $G$ is
   said to \emph{generate} $G$ if every element of $G$ can
   be expressed as a product of elements in $S$ and inverses of elements
   in $S$. (See page 47.)

   \begin{enumerate}[1]
      \item Prove that the set $T$ of all the transpositions in $S_n$
      generates $S_n$.

      \noindent \solution The set $T$ of all the transpositions is 
      \[\{(12),(13),\ldots,(1n),(23),\ldots,(2n),\ldots,((n-1)n)\}\]

      Any permutation in $S_n$ can be written as a product of disjoint
      cycles. So we will prove that every cycle in $S_n$ can be generated by
      $T$.

      Pick a cycle, $\alpha$, at random from $S_n$ and say it equals
      $(a_1,\ldots,a_s)$. This cycle can be produced by the product
      of the transpositions:
       \[(a_1a_s),(a_1a_{s-1}),\ldots,(a_1a_3),
      (a_1a_2) \]

      So now, pick any permutation, $\pi$, at random from $S_n$. Factor
      it into disjoint cycles. Factor each cycle into transpositions
      as just proved you could do. You have now generated $\pi$
      from elements in $T$.

      \item Prove that the set $T_1=\{(12),(13),\ldots,(1n)\}$
      generates $S_n$.

      \noindent \solution Again, we will prove that $T_1$ generates
      any cycle in $S_n$ and therefore generates $S_n$. Choose
      any cycle, $\alpha=(1,a_1,\ldots,a_s)$ that contains the value 1
      and arrange the terms so 1 comes first. Now we can generate
      this cycle with $(1a_s)(1a_{s-1})\ldots(1a_2)(1a_1)$.

      Now pick any cycle $\alpha=(a_1a_2\ldots a_s)$ that does not contain
      the value 1 anywhere. Then this cycle can be generated by
      $(1a_1)(1a_s)(1a_{s-1})\ldots(1a_2)(1a_1)$.

      So we have shown that no matter what cycle we pick we can generate
      it with elements from $T_1$ alone. So like previous part, any 
      permutation can be factored down to disjoint cycles and each cycle
      factored down to transpositions from $T_1$.



   \end{enumerate}


   
 
\end{enumerate}

\end{document}
