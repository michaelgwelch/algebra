\documentclass[twoside]{amsart}
\usepackage{amssymb,latexsym}
\usepackage{amsfonts}
\usepackage{xspace}
\usepackage{enumerate}
\usepackage{graphics}
\usepackage{fitch}
\newcommand{\Rationals}{\mathbb{Q}{}}
\newcommand{\Reals}{\ensuremath{\mathbb{R}}\xspace}
\newcommand{\Integers}{\ensuremath{\mathbb{Z}{}}\xspace}
\newcommand{\solution}{\textsc{Solution}\xspace}
\newcommand{\problem}{\textsc{Problem}\xspace}
\newcommand{\Blank}{\mathrel{\phantom{=}}}
\newcommand{\ltrue}{\top}
\newcommand{\lfalse}{\bot}
\newcommand{\fOfg}{\ensuremath{f \circ g}\xspace}
\newcommand{\gOff}{\ensuremath{g \circ f}\xspace}
\newcommand{\eps}{\ensuremath{\epsilon}\xspace}
\newcommand{\iso}{\cong}
\newcommand{\niso}{\ncong}
\newcommand{\blank}{\vspace{5pt}}
\newcommand{\ind}{\hspace{.35in}}
\newcommand{\degree}{\ensuremath{^\circ}}
\newcommand{\real}{\mathop{\mathrm{real}}}
\newcommand{\img}{\mathop{\mathrm{img}}}
\newcommand{\first}{\mathop{\mathrm{first}}}
\newcommand{\second}{\mathop{\mathrm{second}}}
\newcommand{\abs}{\mathop{\mathrm{abs}}}
\renewcommand{\qed}{\blacksquare}
\newcommand{\itm}{\blank\item}
\newcommand{\sol}{\blank\noindent\solution}
\begin{document}
\title{Answers to Chapter 10 Exercises - A Book of Abstract Algebra}
\author{Michael Welch}
\date{\today}
\maketitle

This document contains selected answers to exercises from chapter 10
of A Book of Abstract Algebra.


\begin{enumerate}[A.]
   
   \item \textsc{Laws of Exponents}

   Let $G$ be a group and $a \in G$.

   \begin{enumerate}[1]
      \blank
      \item Prove that $a^m a^n = a^{m+n}$ in the following cases.
      \begin{enumerate}[(a)]
         \item $m=0$
         \item $m<0$ and $n>0$
         \item $m<0$ and $n<0$
      \end{enumerate}

      \blank \noindent \solution
      \begin{enumerate}[(a)]
         \item If $m=0$ then we have $a^m a^n = a^0 a^n = \epsilon a^n = a^n$.
         In addition, $a^{m+n} = a^{0+n} = a^n$. Therefore, $a^m a^n = a^{m+n}$.

         \item $m<0$ and $n>0$. 
         Let $m' = -m$ and therefore $m'$ is positive. Then 
         we have the following:
         \[a^m = a^{-m'} = \underbrace{a^{-1}a^{-1}\cdots 
         a^{-1}}_{m'\text{ times}}\] Therefore: \[a^m a^n =
         \underbrace{a^{-1}a^{-1}\cdots a^{-1}}_{m'\text{ times}}
         \underbrace{aa\cdots a}_{n\text{ times}}\] 
         Now there are three case: $m'>n$, $m'<n$ and $m'=n$.
         If $m' > n$
         then $n$ of the $a^{-1}$ terms cancel out $n$ of the $a$
         terms leaving $m'-n$ $a^{-1}$ terms: \[a^m a^n = \underbrace{
         a^{-1} a^{-1} \cdots a^{-1}}_{m'-n\text{ times}}\] By
         the definition of exponents we can rewrite this as
         $a^{-(m'-n)} = a^{-m'+n} = a^{m+n}$. If
         $m' < n$ then $m'$ terms of $a^{-1}$ cancel out
         $m'$ terms of $a$ leaving $n-m'$ terms of $a$:
         \[a^m a^n = \underbrace{a a \cdots a}_{n-m'\text{ times}}\]
         This can be rewritten as $a^m a^n = a^{n-m'} = a^{-m' + n} 
         = a^{m+n}$. Finally, if $m'=n$ then all the $a^{-1}$ terms
         cancel out all of the $a$ terms leaving $a^m a^n = \epsilon
         = a^0$. In addition if $m'=n$ then $m=-m'=-n$ and
         $m+n=-n+n = 0$, so $a^{m+n} = \epsilon = a^ma^n$. $\qed$

         \item $m<0$ and $n<0$. Let $m'=-m$ and $n'=-n$ (so $m'$ and
         $n'$ are positive). Then
         \[a^m a^n = a^{-m'} a^{-n'} = \underbrace{a^{-1} a^{-1}
         \cdots a^{-1}}_{m'\text{ times}} \underbrace{a^{-1} a^{-1} \cdots
         a^{-1}}_{n'\text{ times}} = (a^{-1})^{(m'+n')} \]

         Now recall that the text on p104 proved that $a^{-n} =
         (a^{-1})^n = (a^n)^{-1}$ when $n$ is positive. In our
         case the term $m'+n'$ is positive so we can use
         this identity to state that $(a^{-1})^{(m'+n')} = 
         a^{-(m'+n')}$. Now putting it all together we have
         \[
         a^m a^n = a^{-(m'+n')} = a^{(-m' - n')} = a^{(m+n)} 
         \]
         $\qed$

      \end{enumerate}

      \blank
      \item Prove that $(a^m)^n = a^{mn}$ in the following cases:
      \begin{enumerate}[(a)]
         \item $m=0$
         \item $n=0$
         \item $m<0$ and $n>0$
         \item $m>0$ and $n<0$
         \item $m<0$ and $n<0$
      \end{enumerate}

      \blank \noindent \solution
      \begin{enumerate}[(a)]
         \item $m=0$: $(a^m)^n = (a^0)^n = \epsilon^n = \epsilon
         = a^{(0*n)} = a^{mn}$. $\qed$

         \item $n=0$: $(a^m)^n = (a^m)^0 = \epsilon = a^{(m*0)} =
         a^{mn}$. $\qed$

         \item $m<0$ and $n>0$: Let $m' = -m$ so that $m'$ is positive.
         Then we have
         \[
            (a^m)^n = (a^{-m'})^n = (\underbrace{a^{-1} a^{-1} \cdots
            a^{-1}}_{m'\text{ times}})^n = \underbrace{a^{-1} a^{-1} \cdots
            a^{-1}}_{m'n\text{ times}} = a^{-(m'n)} = a^{mn}
         \] $\qed$

         \item $m>0$ and $n<0$: Let $n' = -n$ so that $n'$ is positive.
         Then we have
         \[
            (a^m)^n = (a^m)^{-n'} = \underbrace{(a^m)^{-1} (a^m)^{-1} 
            \cdots (a^m)^{-1}}_{ n'\text{ times}}
         \]
         Recall that the text on p104 proved $(a^m)^{-1} = (a^{-1})^m$ when
         $m >0$. So we have
         \[
           (a^m)^n = \underbrace{(a^{-1})^m (a^{-1})^m \cdots (a^{-1})^m}_{
           n'\text{ times}} = \underbrace{a^{-1} a^{-1} \cdots a^{-1}}_{
           mn'\text{ times}} = (a^{-1})^{mn'}
         \]

         Again we can use Theorem 1 part (iii) because $mn'$ is positive to
         get $(a^m)^n = (a^{-1})^{mn'} = a^{-mn'} = a^{mn}$. $\qed$

         \item $m<0$ and $n<0$. Let $m' = -m$ and $n' = -n$ so that both
         $m', n'$ are positive. Then we have
         \begin{align*} 
            (a^m)^n &= (a^{-m'})^{-n'} \\
                    &= ((a^{m'})^{-1})^{-n'} \text{ Theorem 1(iii) $m'>0$} \\
                    &= \underbrace{((a^{m'})^{-1})^{-1} ((a^{m'})^{-1})^{-1}
                        \cdots ((a^{m'})^{-1})^{-1}}_{n'\text{ times}} \\
                    &= \underbrace{a^{m'} a^{m'} \cdots a^{m'}}_{n'\text{ 
                    times}} \\
                    &= a^{m'n'} \\
                    &= a^{(-m)(-n)} \\
                    &= a^{mn}
         \end{align*}
         $\qed$

      \end{enumerate}

      \blank
      \item Prove that $(a^n)^{-1} = a^{-n}$ in the following cases:
      \begin{enumerate}[(a)]
         \item $n=0$
         \item $n<0$
      \end{enumerate}
      
      \blank \noindent \solution 
      \begin{enumerate}[(a)]
         \item $n=0$: $(a^n)^{-1} = (a^0)^{-1} = \epsilon = a^{-0} = a^{-n}
         \quad \qed$ 

         \item $n<0$: Let $n' = -n$ so that $n'$ is positive. Then
         we have 
         \[
         (a^n)^{-1} = (a^{-n'})^{-1} = (\underbrace{a^{-1} a^{-1} \cdots
         a^{-1}}_{n' \text{ times}})^{-1} = ((\underbrace{a a \cdots
         a}_{n'\text{ times}})^{-1})^{-1} = a^{n'} = a^{-n} \quad \qed
         \]

      \end{enumerate}
   \end{enumerate}

   \blank
   \item \textsc{Examples of Orders of Elements}

   \begin{enumerate}[1]
      \blank
      \item What is the order of 10 in $\Integers_{25}$?

      \blank \noindent \solution The order of 10 is 5. It's the first
      multiple of 10 that is a multiple of 25: $\{10, 20, 5, 15, 0\}$.

      \blank
      \item What is the order of 6 in $\Integers_{16}$?

      \blank \noindent \solution The multiples of 6 are
      $\{6, 12, 2, 8, 14, 4, 10, 0\}$. Therefore the order of 6
      in $\Integers_{16}$ is 8.

      \blank
      \item What is the order of 
      \[
         f = 
            \begin{pmatrix}
               1 & 2 & 3 & 4 & 5 & 6 \\
               6 & 1 & 3 & 2 & 5 & 4
            \end{pmatrix}
      \]
      in $S_6$?

      \blank \noindent \solution $f = (1642)$. We can already guess
      that the order of $f$ is 4. Let's check. The powers of $f$
      are $\{(1642), (14)(26), (1246), \eps\}$. Yep, the power of $f$
      is 4.

      \blank
      \item What is the order of 1 in $\Reals^*$? What is the order
      of 1 in $\Reals$?

      \sol $\Reals^*$ is infinite, however 1 is the identity element in 
      $\Reals^*$ so it's order is 1 in $\Reals^*$. It's a different 
      story in $\Reals$. It is also infinite, but 1 is not the
      identity element in this group. Therefore it's order is infinity.

      \itm If $A$ is the set of all the real numbers $x \ne 0, 1, 2$,
      what is the order of 
      \[
         f(x) = \frac{2}{2-x}
      \]
      in $S_A$?

      \sol We will calculate the powers of f.

      \[
         f(x) = \frac{2}{2-x}
      \]
      \[
         f^2(x) = \cfrac{2}{2-\cfrac{2}{2-x}} 
                = \cfrac{2}{\left(\cfrac{4-2x-2}{2-x}\right)} 
                = \frac{2-x}{1-x}
      \]
      \[
         f^3(x) = \cfrac{2}{2-\cfrac{2-x}{1-x}}
                = \cfrac{2}{\left(\cfrac{2-2x-2+x}{1-x}\right)}
                = \cfrac{2(x-1)}{x}
      \]
      \[
         f^4(x) = \cfrac{2}{2-\cfrac{2(x-1)}{x}}
                = \cfrac{2}{\left(\cfrac{2x-2x+2}{x}\right)}
                = \frac{2x}{2} = x
      \]

      Fortunately $A$ does not include $0, 1, 2$ as each of those 
      is not defined for one of the powers of $f$. We can
      see that the order of $f$ is 4.

      \itm Can as element of an \emph{infinite} group have 
      \emph{finite} order? Explain.

      \sol Obviously yes it can. The identity element of an infinite
      group is the trivial example. But the previous problem is also
      an example. $S_A$ in the previous problem is an infinte
      set, yet $f$ has finite order.

      \itm In $\Integers_{24}$, list all the elements $(a)$ of order 2;
      $(b)$ of order 3; $(c)$ of order 4; $(d)$ of order 6.

      \sol $(a)$: 12; $(b)$ 8, 16; $(c)$ 6, 18; $(d)$ 4, 20

      
   \end{enumerate}


\end{enumerate}

\end{document}
