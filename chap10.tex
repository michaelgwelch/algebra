\documentclass[twoside]{amsart}
\usepackage{amssymb,latexsym}
\usepackage{amsfonts}
\usepackage{xspace}
\usepackage{enumerate}
\usepackage{graphics}
\usepackage{fitch}
\newcommand{\Rationals}{\mathbb{Q}{}}
\newcommand{\Reals}{\ensuremath{\mathbb{R}}\xspace}
\newcommand{\Integers}{\ensuremath{\mathbb{Z}{}}\xspace}
\newcommand{\solution}{\textsc{Solution}\xspace}
\newcommand{\problem}{\textsc{Problem}\xspace}
\newcommand{\Blank}{\mathrel{\phantom{=}}}
\newcommand{\ltrue}{\top}
\newcommand{\lfalse}{\bot}
\newcommand{\fOfg}{\ensuremath{f \circ g}\xspace}
\newcommand{\gOff}{\ensuremath{g \circ f}\xspace}
\newcommand{\eps}{\ensuremath{\epsilon}\xspace}
\newcommand{\iso}{\cong}
\newcommand{\niso}{\ncong}
\newcommand{\blank}{\vspace{5pt}}
\newcommand{\ind}{\hspace{.35in}}
\newcommand{\degree}{\ensuremath{^\circ}}
\newcommand{\real}{\mathop{\mathrm{real}}}
\newcommand{\img}{\mathop{\mathrm{img}}}
\newcommand{\first}{\mathop{\mathrm{first}}}
\newcommand{\second}{\mathop{\mathrm{second}}}
\newcommand{\abs}{\mathop{\mathrm{abs}}}
\begin{document}
\title{Answers to Chapter 10 Exercises - A Book of Abstract Algebra}
\author{Michael Welch}
\date{\today}
\maketitle

This document contains selected answers to exercises from chapter 10
of A Book of Abstract Algebra.


\begin{enumerate}[A.]
   
   \item \textsc{Laws of Exponents}

   Let $G$ be a group and $a \in G$.

   \begin{enumerate}[1]
      \blank
      \item Prove that $a^m a^n = a^{m+n}$ in the following cases.
      \begin{enumerate}[a]
         \item $m=0$
         \item $m<0$ and $n>0$
         \item $m<0$ and $n<0$
      \end{enumerate}

      \blank \noindent \solution
      \begin{enumerate}[a]
         \item If $m=0$ then we have $a^m a^n = a^0 a^n = \epsilon a^n = a^n$.
         In addition, $a^{m+n} = a^{0+n} = a^n$. Therefore, $a^m a^n = a^{m+n}$.

         \item 
      \end{enumerate}


   \end{enumerate}


\end{enumerate}

\end{document}
