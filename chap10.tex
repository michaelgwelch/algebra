\documentclass[twoside]{amsart}
\usepackage{amssymb,latexsym}
\usepackage{amsfonts}
\usepackage{xspace}
\usepackage{enumerate}
\usepackage{graphics}
\usepackage{fitch}
\newcommand{\Rationals}{\mathbb{Q}{}}
\newcommand{\Reals}{\ensuremath{\mathbb{R}}\xspace}
\newcommand{\Integers}{\ensuremath{\mathbb{Z}{}}\xspace}
\newcommand{\solution}{\textsc{Solution}\xspace}
\newcommand{\problem}{\textsc{Problem}\xspace}
\newcommand{\Blank}{\mathrel{\phantom{=}}}
\newcommand{\ltrue}{\top}
\newcommand{\lfalse}{\bot}
\newcommand{\fOfg}{\ensuremath{f \circ g}\xspace}
\newcommand{\gOff}{\ensuremath{g \circ f}\xspace}
\newcommand{\eps}{\ensuremath{\epsilon}\xspace}
\newcommand{\iso}{\cong}
\newcommand{\niso}{\ncong}
\newcommand{\blank}{\vspace{5pt}}
\newcommand{\ind}{\hspace{.35in}}
\newcommand{\degree}{\ensuremath{^\circ}}
\newcommand{\real}{\mathop{\mathrm{real}}}
\newcommand{\img}{\mathop{\mathrm{img}}}
\newcommand{\first}{\mathop{\mathrm{first}}}
\newcommand{\second}{\mathop{\mathrm{second}}}
\newcommand{\abs}{\mathop{\mathrm{abs}}}
%\renewcommand{\qed}{\blacksquare}
\newcommand{\itm}{\blank\item}
\newcommand{\sol}{\blank\noindent\solution}
\newcommand{\ord}{\mathop{\mathrm{ord}}}
\begin{document}
\title{Answers to Chapter 10 Exercises - A Book of Abstract Algebra}
\author{Michael Welch}
\date{\today}
\maketitle

This document contains selected answers to exercises from chapter 10
of A Book of Abstract Algebra.


\begin{enumerate}[A.]
   
   \item \textsc{Laws of Exponents}

   Let $G$ be a group and $a \in G$.

   \begin{enumerate}[1]
      \blank
      \item Prove that $a^m a^n = a^{m+n}$ in the following cases.
      \begin{enumerate}[(a)]
         \item $m=0$
         \item $m<0$ and $n>0$
         \item $m<0$ and $n<0$
      \end{enumerate}

      \blank \noindent \solution
      \begin{enumerate}[(a)]
         \item If $m=0$ then we have $a^m a^n = a^0 a^n = \epsilon a^n = a^n$.
         In addition, $a^{m+n} = a^{0+n} = a^n$. Therefore, $a^m a^n = a^{m+n}$.

         \item $m<0$ and $n>0$. 
         Let $m' = -m$ and therefore $m'$ is positive. Then 
         we have the following:
         \[a^m = a^{-m'} = \underbrace{a^{-1}a^{-1}\cdots 
         a^{-1}}_{m'\text{ times}}\] Therefore: \[a^m a^n =
         \underbrace{a^{-1}a^{-1}\cdots a^{-1}}_{m'\text{ times}}
         \underbrace{aa\cdots a}_{n\text{ times}}\] 
         Now there are three case: $m'>n$, $m'<n$ and $m'=n$.
         If $m' > n$
         then $n$ of the $a^{-1}$ terms cancel out $n$ of the $a$
         terms leaving $m'-n$ $a^{-1}$ terms: \[a^m a^n = \underbrace{
         a^{-1} a^{-1} \cdots a^{-1}}_{m'-n\text{ times}}\] By
         the definition of exponents we can rewrite this as
         $a^{-(m'-n)} = a^{-m'+n} = a^{m+n}$. If
         $m' < n$ then $m'$ terms of $a^{-1}$ cancel out
         $m'$ terms of $a$ leaving $n-m'$ terms of $a$:
         \[a^m a^n = \underbrace{a a \cdots a}_{n-m'\text{ times}}\]
         This can be rewritten as $a^m a^n = a^{n-m'} = a^{-m' + n} 
         = a^{m+n}$. Finally, if $m'=n$ then all the $a^{-1}$ terms
         cancel out all of the $a$ terms leaving $a^m a^n = \epsilon
         = a^0$. In addition if $m'=n$ then $m=-m'=-n$ and
         $m+n=-n+n = 0$, so $a^{m+n} = \epsilon = a^ma^n$. $\qed$

         \item $m<0$ and $n<0$. Let $m'=-m$ and $n'=-n$ (so $m'$ and
         $n'$ are positive). Then
         \[a^m a^n = a^{-m'} a^{-n'} = \underbrace{a^{-1} a^{-1}
         \cdots a^{-1}}_{m'\text{ times}} \underbrace{a^{-1} a^{-1} \cdots
         a^{-1}}_{n'\text{ times}} = (a^{-1})^{(m'+n')} \]

         Now recall that the text on p104 proved that $a^{-n} =
         (a^{-1})^n = (a^n)^{-1}$ when $n$ is positive. In our
         case the term $m'+n'$ is positive so we can use
         this identity to state that $(a^{-1})^{(m'+n')} = 
         a^{-(m'+n')}$. Now putting it all together we have
         \[
         a^m a^n = a^{-(m'+n')} = a^{(-m' - n')} = a^{(m+n)} 
         \]
         $\qed$

      \end{enumerate}

      \blank
      \item Prove that $(a^m)^n = a^{mn}$ in the following cases:
      \begin{enumerate}[(a)]
         \item $m=0$
         \item $n=0$
         \item $m<0$ and $n>0$
         \item $m>0$ and $n<0$
         \item $m<0$ and $n<0$
      \end{enumerate}

      \blank \noindent \solution
      \begin{enumerate}[(a)]
         \item $m=0$: $(a^m)^n = (a^0)^n = \epsilon^n = \epsilon
         = a^{(0*n)} = a^{mn}$. $\qed$

         \item $n=0$: $(a^m)^n = (a^m)^0 = \epsilon = a^{(m*0)} =
         a^{mn}$. $\qed$

         \item $m<0$ and $n>0$: Let $m' = -m$ so that $m'$ is positive.
         Then we have
         \[
            (a^m)^n = (a^{-m'})^n = (\underbrace{a^{-1} a^{-1} \cdots
            a^{-1}}_{m'\text{ times}})^n = \underbrace{a^{-1} a^{-1} \cdots
            a^{-1}}_{m'n\text{ times}} = a^{-(m'n)} = a^{mn}
         \] $\qed$

         \item $m>0$ and $n<0$: Let $n' = -n$ so that $n'$ is positive.
         Then we have
         \[
            (a^m)^n = (a^m)^{-n'} = \underbrace{(a^m)^{-1} (a^m)^{-1} 
            \cdots (a^m)^{-1}}_{ n'\text{ times}}
         \]
         Recall that the text on p104 proved $(a^m)^{-1} = (a^{-1})^m$ when
         $m >0$. So we have
         \[
           (a^m)^n = \underbrace{(a^{-1})^m (a^{-1})^m \cdots (a^{-1})^m}_{
           n'\text{ times}} = \underbrace{a^{-1} a^{-1} \cdots a^{-1}}_{
           mn'\text{ times}} = (a^{-1})^{mn'}
         \]

         Again we can use Theorem 1 part (iii) because $mn'$ is positive to
         get $(a^m)^n = (a^{-1})^{mn'} = a^{-mn'} = a^{mn}$. $\qed$

         \item $m<0$ and $n<0$. Let $m' = -m$ and $n' = -n$ so that both
         $m', n'$ are positive. Then we have
         \begin{align*} 
            (a^m)^n &= (a^{-m'})^{-n'} \\
                    &= ((a^{m'})^{-1})^{-n'} \text{ Theorem 1(iii) $m'>0$} \\
                    &= \underbrace{((a^{m'})^{-1})^{-1} ((a^{m'})^{-1})^{-1}
                        \cdots ((a^{m'})^{-1})^{-1}}_{n'\text{ times}} \\
                    &= \underbrace{a^{m'} a^{m'} \cdots a^{m'}}_{n'\text{ 
                    times}} \\
                    &= a^{m'n'} \\
                    &= a^{(-m)(-n)} \\
                    &= a^{mn}
         \end{align*}
         $\qed$

      \end{enumerate}

      \blank
      \item Prove that $(a^n)^{-1} = a^{-n}$ in the following cases:
      \begin{enumerate}[(a)]
         \item $n=0$
         \item $n<0$
      \end{enumerate}
      
      \blank \noindent \solution 
      \begin{enumerate}[(a)]
         \item $n=0$: $(a^n)^{-1} = (a^0)^{-1} = \epsilon = a^{-0} = a^{-n}
         \quad \qed$ 

         \item $n<0$: Let $n' = -n$ so that $n'$ is positive. Then
         we have 
         \[
         (a^n)^{-1} = (a^{-n'})^{-1} = (\underbrace{a^{-1} a^{-1} \cdots
         a^{-1}}_{n' \text{ times}})^{-1} = ((\underbrace{a a \cdots
         a}_{n'\text{ times}})^{-1})^{-1} = a^{n'} = a^{-n} \quad \qed
         \]

      \end{enumerate}
   \end{enumerate}

   \blank
   \item \textsc{Examples of Orders of Elements}

   \begin{enumerate}[1]
      \blank
      \item What is the order of 10 in $\Integers_{25}$?

      \blank \noindent \solution The order of 10 is 5. It's the first
      multiple of 10 that is a multiple of 25: $\{10, 20, 5, 15, 0\}$.

      \blank
      \item What is the order of 6 in $\Integers_{16}$?

      \blank \noindent \solution The multiples of 6 are
      $\{6, 12, 2, 8, 14, 4, 10, 0\}$. Therefore the order of 6
      in $\Integers_{16}$ is 8.

      \blank
      \item What is the order of 
      \[
         f = 
            \begin{pmatrix}
               1 & 2 & 3 & 4 & 5 & 6 \\
               6 & 1 & 3 & 2 & 5 & 4
            \end{pmatrix}
      \]
      in $S_6$?

      \blank \noindent \solution $f = (1642)$. We can already guess
      that the order of $f$ is 4. Let's check. The powers of $f$
      are $\{(1642), (14)(26), (1246), \eps\}$. Yep, the power of $f$
      is 4.

      \blank
      \item What is the order of 1 in $\Reals^*$? What is the order
      of 1 in $\Reals$?

      \sol $\Reals^*$ is infinite, however 1 is the identity element in 
      $\Reals^*$ so it's order is 1 in $\Reals^*$. It's a different 
      story in $\Reals$. It is also infinite, but 1 is not the
      identity element in this group. Therefore it's order is infinity.

      \itm If $A$ is the set of all the real numbers $x \ne 0, 1, 2$,
      what is the order of 
      \[
         f(x) = \frac{2}{2-x}
      \]
      in $S_A$?

      \sol We will calculate the powers of f.

      \[
         f(x) = \frac{2}{2-x}
      \]
      \[
         f^2(x) = \cfrac{2}{2-\cfrac{2}{2-x}} 
                = \cfrac{2}{\left(\cfrac{4-2x-2}{2-x}\right)} 
                = \frac{2-x}{1-x}
      \]
      \[
         f^3(x) = \cfrac{2}{2-\cfrac{2-x}{1-x}}
                = \cfrac{2}{\left(\cfrac{2-2x-2+x}{1-x}\right)}
                = \cfrac{2(x-1)}{x}
      \]
      \[
         f^4(x) = \cfrac{2}{2-\cfrac{2(x-1)}{x}}
                = \cfrac{2}{\left(\cfrac{2x-2x+2}{x}\right)}
                = \frac{2x}{2} = x
      \]

      Fortunately $A$ does not include $0, 1, 2$ as each of those 
      is not defined for one of the powers of $f$. We can
      see that the order of $f$ is 4.

      \itm Can as element of an \emph{infinite} group have 
      \emph{finite} order? Explain.

      \sol Obviously yes it can. The identity element of an infinite
      group is the trivial example. But the previous problem is also
      an example. $S_A$ in the previous problem is an infinte
      set, yet $f$ has finite order.

      \itm In $\Integers_{24}$, list all the elements $(a)$ of order 2;
      $(b)$ of order 3; $(c)$ of order 4; $(d)$ of order 6.

      \sol $(a)$: 12; $(b)$ 8, 16; $(c)$ 6, 18; $(d)$ 4, 20
   \end{enumerate}

   \itm \textsc{Elementary Properties of Order}
   
   \noindent Let $a, b, c$ be elements of a group $G$. Prove the following:

   \begin{enumerate}[1]
      \itm $\mathrm{Ord}(a) = 1$ iff $a=e$.

      \sol First case. Assume that $\mathrm{Ord}(a) = 1$. Then we know by
      definition of order that $a^1 = \epsilon$, but $a^1=a$. Therefore $a=e$.

      Next case. Assume that $a=e$. Then trivially the order of $a$
      is 1 since $a^1=a=e$. $\qed$

      \itm If $\mathrm{Ord}(a)=n$, then $a^{n-r} = (a^r)^{-1}$.

      \sol We are given that the order of $a$ is $n$. Therefore
      $a^n = e$. So $a^{n-r} = a^n a^{-r} = a^{-r}$. And by theorem
      1 we know $a^{-r}=(a^r)^{-1}$. Therefore $a^{n-r} = a^{-r} =
      (a^r)^{-1}$. $\qed$

      \itm If $a^k=e$ where $k$ is odd, then the order of $a$ is odd.
      
      \sol Let $n$ be the order of $a$. By Theorem 5 on p 107 we know that
      $k$ is a multiple of $n$. Assume $n$ is even. Then every multiple of
      $n$ is also even which means $k$ is even. However we are told that
      $k$ is odd. Therefore $n$ must be odd as well.

      \itm $\ord(a) = \ord(bab^{-1})$.

      \sol This one is easy to see. Let's assume the order of $a$
      is $n$. Let's first see what happens when $n=1$. Then
      $a=e$ and $\ord(bab^{-1}) = bb^{-1} = e = a$. So the order of
      $a$ equals the order of $bab^{-1}$.

      Now let's assume $n=2$. Then we have $a^2 = e$ and
      $(bab^{-1})^2 = bab^{-1}bab^{-1} = ba^2b^{-1} = e$. Again
      the order of each is $n$.

      Now let's assume any positive value of $n$. We know that $a^n=e$.
      Let's examing the value $(bab^{-1})^n$.
      \[
         (bab^{-1})^n = \underbrace{(bab^{-1})(bab^{-1})\cdots
          (bab^{-1})}_{n\text{ times}}
      \]
      Now when you remove the parentheses, you will have $n-1$
      instances of $b^{-1}b$ next to each other. After those are
      all cancelled out you will have $b\underbrace{a a \cdots a}_{
      n\text{ times}}b^{-1} = ba^nb^{-1} = e$. 

      Now I really didn't prove that the order of $bab^{-1}$ is $n$.
      There could be some number $m$ smaller than $n$ such
      that $(bab^{-1})^m = e$. Let's assume there is. Then we know that
      $a^m \ne e$. And we have $(bab^{-1})^m = ba^mb^{-1} = e$. Now
      I can multiply on the right by $b$ and get $ba^m = b$. Nnow I
      can multiply on the left by $b^{-1}$ and get $a^m = b^{-1}b = e$.
      So I've just shown that $a^m = e$. But this can't be since $m < n$
      and $n$ is the order of $a$. Therefore, there can be no number
      $m$ less than $n$ that is the order of $bab^{-1}$. Therefore,
      the order of $bab^{-1}$ is $n$. $\qed$

      \itm The order of $a^{-1}$ is the same as the order of $a$.

      \sol Assume the order of $a$ is $n$. Then we need to show that
      $(a^{-1})^n = e$ and that there is no other number $m$ less than $n$
      such that $(a^{-1})^m = e$.

      So we are given that $a^n = e$. So I can multiply each side
      by $a^{-n}$ and get $e = a^{-n} = (a^{-1})^n$. Which is the
      first thing we wanted to show. Next assume there exists some
      $m < n$ such that $(a^{-1})^m = e$. We know that $a^m \ne e$.
      If $(a^{-1})^m = e$ then I can multiply both sides by $a^m$
      to get $a^{-m}a^m = e = a^m$. Which is a contradiction. Therefore,
      no $m$ can exist and $n$ is the order of $a^{-1}$. $\qed$

      \itm The order of $ab$ is the same as the order of $ba$.

      \sol Assume the order of $ab$ = $n$, so $(ab)^n = e$. 
      \[
         (ab)^n = \underbrace{(ab)(ab) \cdots (ab)}_{n\text{ times}} = e
      \]
      If I multiply on the right by $a$ and change how I group the
      terms I get
      \[
         \underbrace{(ab)(ab) \cdots (ab)}_{n\text{ times}} a =
            a\underbrace{(ba)(ba) \cdots (ba)}_{n\text{ times}} = ae
      \]
      and using cancellation law I can eliminate the extra $a$ on
      each side and I'm left with
      \[
         \underbrace{(ba)(ba) \cdots (ba)}_{n\text{ times}} = (ba)^n = e
      \]

      Now I can use the same trick as the last two problems to prove
      that there is no nuber $m$ smaller than  $n$  such that $(ba)^m = e$.
      Let's assume that $m < n$ and $(ba)^m = e$. Then I have
      \[
         \underbrace{(ba)(ba) \cdots (ba)}_{m\text{ times}}b = 
         b\underbrace{(ab)(ab) \cdots (ab)}_{m\text{ times}} = be
      \]
      and using cancellation laws we have $(ab)^m = e$ which can't be
      becaues the order of $ab$ is $n$. Therefore the order
      of $ba$ equals the order of $ab$. $\qed$

      \itm $\ord(abc) = \ord(cab) = \ord(bca)$.

      \sol Assume the order of $abc$ is $n$. Then $(abc)^n = e$.
      Let's examine $(cab)^nc$. And use the same expansion trick
      from last time.

      \[
         (cab)^nc = c(abc)^n = c
      \]

      Using cancellation we get $(cab)^n = e$. We can use the same trick
      as last time to prove that $n$ is the smallest number for which
      this is true.  And we can use the same trick for $bca$. $\qed$

      \itm Let $x=a_1 a_2 \ldots a_n$, and let $y$ be a product of the same
      factors, permutated cyclically. (That is, $y = a_k a_{k+1} \ldots
      a_n a_1 \ldots a_{k-1}$.) Then $\ord(x) = \ord(y)$.

      \sol Use the exact same tricks from the last two problems. $\qed$

   \end{enumerate}

   \itm \textsc{Further Properties of Order}
   
   \noindent Let $a$ be any element of finite order of a group $G$. Prove
   the following.

   \begin{enumerate}[1]
      \itm If $a^p = e$ where $p$ is a prime number, then $a$ has order
      $p$. $(a \ne e.)$

      \sol We know that there is no number $m$ greater than $p$ that is
      the order of $a$ by definition of order. So let's assume the order of
      $a$ is not $p$ and there is some $m$ less than $p$ such that
      $a^m = e$. Now let's use the division algorithm to divide $p$
      by $m$. We have $p = mq + r$ which means
      \[
         a^p = a^{mq} a^r = (a^m)^q a^r = e a^r = a^r
      \]

      So $a^p = a^r$ where $0 \leq r < m$. But we know that $a^p = e$ so
      $a^r = e$. Now we know that $r = 0$ because by the definition of order
      there can be no positive value less than $m$ such that $a$ raised
      to that value is $e$. Therefore $r=0$. Which means that $p = mq$. 
      Since $p$ is prime that means that $m$ must be equal to 1. But this
      would imply that $a^1 = e$ which would mean that $a=e$. But we said
      that $a$ is any element of finite order of $G$. So this means that
      in general there is no $m < p$ for which $a^m = e$. Indeed this is
      only true if we choose $a$ to be $e$. In which case $p = 1$. In the
      all case $p$ is the order of $a$. $\qed$

      \itm The order of $a^k$ is a divisor (factor) of the order of $a$.

      \sol Let $n$ be the order of $a$ and let $m$ be the order of 
      $a^k$. Since $m$ is the order of $a^k$ we have 
      $(a^k)^m = e$. We also can see that $(a^k)^n = (a^n)^k = e$.
      Now by Theorem 5 $n$ is a multiple of $m$. (This was hard for 
      me to see. Let $b = a^k$ to make it more clear. We have
      $b^m = e$ and $b^n = e$. Since $m$ is the order of $b$ it
      is by definition smaller than $n$ and by Theorem 5 $n$ is a 
      multiple of $m$.) $\qed$

      \itm If $\ord(a) = km$, then $\ord(a^k) = m$.

      \sol $(a^{km}) = (a^k) ^m = e$. Therefore the order of $a^k$ is $m$.
      $\qed$

      \itm If $\ord(a) = n$ where $n$ is odd, then $\ord(a^2) = n$.

      \sol I will show two things: 1) $(a^2)^n = e$ and 2)
      There is no number $m < n$ which is the order of $a^2$.

      The first part is trivial: $(a^2)^n = (a^n)^2 = e^2 = e$.

      The second part will be proved by contradiction. Assume there is
      some $m < n$ such that $(a^2)^m = e$. Then, by Theorem 5, $n$
      is a multiple of $m$. In other words there is some $q > 1$ 
      such that $n = qm$.

      Now assume $q=2$. Then $n=2q$. But this is impossible because we
      are given that $n$ is odd. So even if $n$ is a multiple of $m$, 
      it cannot be the case that $q=2$.

      Now assume $q > 2$. Since $n=qm$ then $m=n/q$.
      Then  $(a^2)^m = (a^2)^{n/q} = a^{(2/q)n} = e$. But $2/q < 1$
      and $(2/q)n < n$.  But we know that $n$ is the order of $a$
      so it cannot be possible that $a^{(2/q)n} = e$.

      So we have shown that there is no possible value for $q$. Therefore,
      our assumption that there is a value of $m < n$ for which
      $(a^2)^m = e$ is false. Therefore $n$ is the order of $a^2$. $\qed$
      
      \itm If $a$ has order $n$, and $a^r = a^s$, then $n$ is a factor
      of $r-s$.

      \sol Given $a^n = e$, $a^r = a^s$. Therefore
      $a^{r-s} = a^r a^{-s} = a^s a^{-s} = e$. By theorem 5, $r-s$ is
      a multiple of $n$. Therefore $n$ is a factor of $r-s$. $\qed$

      \itm If $a$ is the \emph{only} element of order $k$ in $G$, then
      $a$ is in the center of $G$. (\textsc{Hint}: Use Exercise C4.
      Also, see Chapter 4, Exercise C6.)

      \sol Given $\ord(a)=k$. Therefore, for any element $b \in G$
      (by part C4), we have $\ord(bab^{-1}) = k$. Therefore $(bab^{-1})^k = e$.

      By Chapter 4 C6 $ab = ba$ iff $bab^{-1} = a$. 

      STUMPED!!!

      \itm If the order of $a$ is not a multiple of $m$, then
      the order of $a^k$ is not a multiple of $m$.

      \sol Given that the order of $a$ is not a multiple of $m$.
      Assume that order of $a^k$ is a multiple of $m$. Then
      for some integer $q_1 \geq 1$ we have $\ord(a^k) = q_1m$.
      By part 2 we know that the order of $a$ is a multiple of
      the order of $a^k$. Therefore there is some $q_2$ such
      that $\ord(a) = q_2 \ord(a^k) = q_2 q_1 m$. Therefore
      order of $a$ is a multiple of $m$. But this contradicts
      our premise. So order of $a^k$ must not be a multiple of $m$.$\qed$.

      \itm If $\ord(a) = mk$ and $a^{rk} = e$, then $r$ is a multiple of $m$.

      \sol We know by theorem 5 that $rk$ is a multiple of $mk$. 
      So $rk = qmk$ for some $q \ne 0$.  So $r = qm$. So $r$ is a multiple
      of $m$. $\qed$

   \end{enumerate}
   
   \itm \textsc{Relationship between $\ord(ab)$, $\ord(a)$, and $\ord(b)$}

   \noindent Let $a$ and $b$ be elements of a group $G$. Let $\ord(a)=m$ and
   $\ord(b)=n$; let $\mathrm{lcm}(m,n)$ denote the least common multiple
   of $m$ and $n$. Prove parts 1-5.

   \begin{enumerate}[1]
      \itm If $a$ and $b$ commute, then $\ord(ab)$ is a divisor of 
      $\mathrm{lcm}(m,n)$. 

      \sol Let $o$ be the order of $ab$. Let $l$ be the least common
      multiple of $m$ and $n$.

      First let's calculate $(ab)^l$. $(ab)^l = a^lb^l$ because
      $a$ and $b$ commute. Since $l$ is a multiple of $m$ and $n$
      we know there exist integers $q_m$ and $q_n$ such that
      $mq_m = l$ and $nq_n = l$. So we have 
      
      \begin{align*}
         (ab)^l &= a^l b^l \\
                &= a^{mq_m} b^{nq_n} \\
                &= (a^m)^{q_m} (b^n)^{q_n} \\
                &= e^{q_m} e^{q_n} \\
                &= e
      \end{align*}

      Since $o$ is the order of $(ab)$, by theorem 5, we know that $l$ is
      a multiple of $o$. Therefore $\ord(ab)$ is a divisor of $\mathrm{lcm}
      (m,n)$. $\qed$

      \itm If $m$ and $n$ are relatively prime, then no power of $a$
      can be equal to any power of $b$ (except for $e$). (\textsc{Remark}: Two
      integers are said to be relatively prime if they have no common factors
      except $\pm 1$.) (\textsc{Hint}: Use exercise D2.)

      \sol The hint really gives it away. Okay let's assume the opposite.
      Let's assume that there exist some numbers $s$, and $t$ such
      that $a^s = b^t = c$ where $c \ne e$. In other words, we have found a
      power of $a$ that is equal to a power of $b$ other than $e$. 
      We know from D2 that
      $\ord(c) = \ord(a^s)$ is a factor of the $\ord(a)$ which is $m$.
      Likewise we know that the $\ord(c) = \ord(b^t)$ is a factor
      of the $\ord(b)$ which is $n$. So we know that the
      $\ord(c)$ is a factor of $m$ and $n$. 

      Then we have found a factor common to both $m$ and $n$ which 
      contradicts our givens. Therefore, we know there exists no $s$ or
      $t$. $\qed$

      \itm If $m$ and $n$ are relatively prime, then the products 
      $a^ib^j (0 \le i < m, 0 \le j < n)$ are all distinct.

      \sol To prove this, let's assume that there exists some $r,r',s,s'$ 
      where $(r,s)$ and $(r',s')$ are distinct pairs of integers 
      and $r, r'$ range over $[0,m-1]$ and
      $s,s'$ range over $[0,n-1]$ and $a^rb^s = a^{r'} b^{s'}$.

      \blank
      Multiply each side on the left by $a^{-r'}$ and on the right by
      $b^{-s}$ and we get $a^{r-r'} = b^{s'-s}$.

      We proved in previous exercise that no power of $a$ can equal
      any power of $b$ (except for $e$).  Therefore it is not the case
      that $a^rb^s = a^{r'} b^{s'}$.

      \itm Let $a$ and $b$ commute. If $m$ and $n$ are relativley prime,
      then $\ord(ab) = mn$. (\textsc{Hint}: Use part 2.)

      \sol First we show that $(ab)^{mn} = e$:
      \begin{align*}
         (ab)^{mn} &= a^{mn}b^{mn} \\
                   &= (a^m)^n (b^n)^m \\
                   &= e^n e^m \\
                   &= e
      \end{align*}

      Now we will show that there doesn't exist any number $p$ less than
      $mn$ for which $(ab)^p = e$. Assume that there is. Then $(ab)^p = e$.
      Then we have $a^pb^p = e$ or $a^p = b^{-p}$. But by part 2 we know
      that no power of $a$ can equal any power of $b$ except for $e$.

      \itm Let $a$ and $b$ commute. There is an element $c$ in $G$ whose
      order is $\mathrm{lcm}(m,n)$. (\textsc{Hint}: User part 4, above, 
      together with Exercise D3. Let $c=a^ib$ where $a^i$ is a certain
      power of $a$.)

      \sol First let's prove this if $m$ and $n$ are relatively prime.
      Then we choose $c = ab$. The order of $ab$ is the order of $c$ and by
      part 4 is $mn$. But since $m$ and $n$ are relatively prime we know
      that the least common multiple of $m$ and $n$ is $mn$. So we
      have found $c$ if $m$ and $n$ are relatively prime.

      \blank
      Now let's assume that $m$ and $n$ are not relatively prime. This
      means they share at least one factor. Let $l = \mathrm{lcm}( 
      m,n)$. Then we know that $mn = ql$ for some integer $q > 1$.
      Now we choose $c = a^ib$. By D3 the order of $a^i$ is $m/i$.


      \sol $c = a^ib$ where $a^i$ is some power of $a$. Proof.
      Let $l = \mathrm{lcm}(m,n)$. So $l = nr_n = mr_m$ for some $r_m, r_n$.
      \begin{align*}
         c^l &= (a^ib)^l \\
             &= (a^i)^lb^l \\
             &= (a^m)^{ir_m} (b^n)^{r_n} \\
             &= e^{ir_m} e^{r_n} \\
             &= e
      \end{align*}

      Now assume there is some number $p < l$ such that $c^p = e$.
      \begin{align*}
         c^p &= e \\
         c^p &= (a^ib)^p \\
             &= (a^i)^p b^p \\
         (a^i)^p b^p &= e 
      \end{align*}

      STUMP

      \itm Give an example to show that part 1 is not true if $a$ and $b$
      do not commute.

   \end{enumerate}

   \itm Let $a$ be an element of order 12 in a group $G$.

   \begin{enumerate}[1]
      \itm What is the smallest positive integer $k$ such that $a^{8k} = e$?
      (\textsc{Hint}: Use Theorem 5 and explain carefully!)

      \sol By Theorem 5 if $k$ is chosen such that $a^{8k} = e$ then 
      $8k$ must be a multiple of 12. So we know that $8k = 12q$
      for some value of $q$. We want the smallest value of $k$ so we want
      the least common multiple of $8$ and $12$ which is 24. So $k = 3$
      and $q=2$. 

      \itm What is the order of $a^8$?

      \sol The order of $a^8$ is 3. We know from previous exercise that 3
      is the smallest value of $k$ such that $(a^8)^k = a^{8k} = e$. 

      \itm What are the orders of $a^9$, $a^{10}$, $a^5$?

      \sol The order of $a^9$ is 4: $(a^9)^4 = a^{36} = (a^{12})^3 = e$.
      The order of $a^{10}$ is 6: $(a^{10})^6 = a^{60} = (a^{12})^5 = e$.
      The order of $a^5$ is 12: $(a^5)^{12} = a^{60} = (a^{12})^5 = e$.
      In each of these cases there is no smaller exponent that works.

      \itm Which powers of $a$ have the same order as $a$? [That is, for
      what values of $k$ is $\ord(a^k)$ = 12?]

      \sol The powers of $a$ that have order 12 are: $a, a^5, a^7, a^{11}$.

      \itm Let $a$ be an element of order $m$ in any group $G$. What 
      is the order of $a^k$?

      \sol The order of $a^k$ is $\mathrm{lcm}(k,m)/k$.

      \itm Let $a$ be an element of order $m$ in any group $G$. For what
      values of $k$ is $\ord(a^k)=m$. 

      \sol The $\ord(a^k)=m$ for values of $k$ that are relatively prime
      to $m$.
   \end{enumerate}



\end{enumerate}

\end{document}
