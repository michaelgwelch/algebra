\documentclass[twoside]{amsart}
\usepackage{amssymb,latexsym}
\usepackage{amsfonts}
\usepackage{xspace}
\usepackage{enumerate}
\usepackage{graphics}
\usepackage{fitch}
\newcommand{\Rationals}{\mathbb{Q}{}}
\newcommand{\Reals}{\ensuremath{\mathbb{R}}\xspace}
\newcommand{\Integers}{\ensuremath{\mathbb{Z}{}}\xspace}
\newcommand{\solution}{\textsc{Solution}\xspace}
\newcommand{\problem}{\textsc{Problem}\xspace}
\newcommand{\Blank}{\mathrel{\phantom{=}}}
\newcommand{\ltrue}{\top}
\newcommand{\lfalse}{\bot}
\newcommand{\fOfg}{\ensuremath{f \circ g}\xspace}
\newcommand{\gOff}{\ensuremath{g \circ f}\xspace}
\newcommand{\eps}{\ensuremath{\epsilon}\xspace}
\newcommand{\iso}{\cong}
\newcommand{\niso}{\ncong}
\newcommand{\blank}{\vspace{5pt}}
\newcommand{\ind}{\hspace{.35in}}
\newcommand{\degree}{\ensuremath{^\circ}}
\newcommand{\real}{\mathop{\mathrm{real}}}
\newcommand{\img}{\mathop{\mathrm{img}}}
\newcommand{\first}{\mathop{\mathrm{first}}}
\newcommand{\second}{\mathop{\mathrm{second}}}
\newcommand{\abs}{\mathop{\mathrm{abs}}}
\renewcommand{\qed}{\blacksquare}
\begin{document}
\title{Answers to Chapter 10 Exercises - A Book of Abstract Algebra}
\author{Michael Welch}
\date{\today}
\maketitle

This document contains selected answers to exercises from chapter 10
of A Book of Abstract Algebra.


\begin{enumerate}[A.]
   
   \item \textsc{Laws of Exponents}

   Let $G$ be a group and $a \in G$.

   \begin{enumerate}[1]
      \blank
      \item Prove that $a^m a^n = a^{m+n}$ in the following cases.
      \begin{enumerate}[(a)]
         \item $m=0$
         \item $m<0$ and $n>0$
         \item $m<0$ and $n<0$
      \end{enumerate}

      \blank \noindent \solution
      \begin{enumerate}[(a)]
         \item If $m=0$ then we have $a^m a^n = a^0 a^n = \epsilon a^n = a^n$.
         In addition, $a^{m+n} = a^{0+n} = a^n$. Therefore, $a^m a^n = a^{m+n}$.

         \item $m<0$ and $n>0$. 
         Let $m' = -m$ and therefore $m'$ is positive. Then 
         we have the following:
         \[a^m = a^{-m'} = \underbrace{a^{-1}a^{-1}\cdots 
         a^{-1}}_{m'\text{ times}}\] Therefore: \[a^m a^n =
         \underbrace{a^{-1}a^{-1}\cdots a^{-1}}_{m'\text{ times}}
         \underbrace{aa\cdots a}_{n\text{ times}}\] 
         Now there are three case: $m'>n$, $m'<n$ and $m'=n$.
         If $m' > n$
         then $n$ of the $a^{-1}$ terms cancel out $n$ of the $a$
         terms leaving $m'-n$ $a^{-1}$ terms: \[a^m a^n = \underbrace{
         a^{-1} a^{-1} \cdots a^{-1}}_{m'-n\text{ times}}\] By
         the definition of exponents we can rewrite this as
         $a^{-(m'-n)} = a^{-m'+n} = a^{m+n}$. If
         $m' < n$ then $m'$ terms of $a^{-1}$ cancel out
         $m'$ terms of $a$ leaving $n-m'$ terms of $a$:
         \[a^m a^n = \underbrace{a a \cdots a}_{n-m'\text{ times}}\]
         This can be rewritten as $a^m a^n = a^{n-m'} = a^{-m' + n} 
         = a^{m+n}$. Finally, if $m'=n$ then all the $a^{-1}$ terms
         cancel out all of the $a$ terms leaving $a^m a^n = \epsilon
         = a^0$. In addition if $m'=n$ then $m=-m'=-n$ and
         $m+n=-n+n = 0$, so $a^{m+n} = \epsilon = a^ma^n$. $\qed$

         \item $m<0$ and $n<0$. Let $m'=-m$ and $n'=-n$ (so $m'$ and
         $n'$ are positive). Then
         \[a^m a^n = a^{-m'} a^{-n'} = \underbrace{a^{-1} a^{-1}
         \cdots a^{-1}}_{m'\text{ times}} \underbrace{a^{-1} a^{-1} \cdots
         a^{-1}}_{n'\text{ times}} = (a^{-1})^{(m'+n')} \]

         Now recall that the text on p104 proved that $a^{-n} =
         (a^{-1})^n = (a^n)^{-1}$ when $n$ is positive. In our
         case the term $m'+n'$ is positive so we can use
         this identity to state that $(a^{-1})^{(m'+n')} = 
         a^{-(m'+n')}$. Now putting it all together we have
         \[
         a^m a^n = a^{-(m'+n')} = a^{(-m' - n')} = a^{(m+n)} 
         \]
         $\qed$

      \end{enumerate}


   \end{enumerate}


\end{enumerate}

\end{document}
