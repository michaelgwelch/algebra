\documentclass[twoside]{amsart}
\usepackage{amssymb,latexsym}
\usepackage{xspace}
\usepackage{enumerate}
\usepackage{graphics}
\newcommand{\Rationals}{\mathbb{Q}{}}
\newcommand{\Reals}{\mathbb{R}{}}
\newcommand{\Integers}{\mathbb{Z}{}}
\newcommand{\Solution}{\textsc{Solution}\xspace}
\newcommand{\Problem}{\textsc{Problem}\xspace}
\newcommand{\Blank}{\mathrel{\phantom{=}}}
\newcommand{\ltrue}{\top}
\newcommand{\lfalse}{\bot}
\begin{document}
\title{Answers to Chapter 3 Exercises - A Book of Abstract Algebra}
\author{Michael Welch}
\date{\today}
\maketitle

This document contains selected answers to exercises from chapter 4
of A Book of Abstract Algebra.

Remark on notation. In the exercises in this document, the exponential
notation $a^n$ is used in the following sense: if $a$ is any element of
a group $G$, then $a^2$ means $aa$, and $a^3$ means $aaa$, and, in
general, $a^n$ is the product of $n$ factors of $a$, for any positive
integer $n$.

\begin{enumerate}[A.]
   \item \textsc{Solving Equations in Groups}

   \noindent Let $a$, $b$, $c$, and $x$ be elements of a group $G$. In each of
   the following, solve for $x$ in terms of $a$, $b$, and $c$.

   \noindent   1. $axb = c$
      \begin{align*}
         axb & = c \\
	 a^{-1}axb & = a^{-1}c \\
	 xbb^{-1} & = a^{-1}cb^{-1} \\
	 x & = a^{-1}cb^{-1}
      \end{align*}

   \noindent   2. $x^2b = xa^{-1}c$
      \begin{align*}
         x^2b & = xa^{-1}c    \\
	 x^2bb^{-1} & = xa^{-1}cb^{-1} \\
	 x^2        & = xa^{-1}cb^{-1} \\
	 x^{-1}x^2  & = x^{-1}xa^{-1}cb^{-1}\\
	 x^{-1}xx   & = a^{-1}cb^{-1} \\
	 x          & = a^{-1}cb^{-1}
      \end{align*}

   Solve simultaneously:

   \noindent 3. $x^2a = bxc^{-1}$ and $acx = xac$
   \begin{align*}
      acx & = xac      & \text{Given} \\
      xacx & = x^2ac   & \text{Multiply by $x$} \\
      xacxc^{-1} & = x^2a \\
      xacxc^{-1} & = bxc^{-1} & \text{Substitute} \\
      xacx       & = bx  \\
      xac        & = b  \\
      x          & = bc^{-1}a^{-1}
   \end{align*}

   \noindent 4. $ax^2=b$ and $x^3=e$
   \begin{align*}
      ax^2 & = b     & \text{Given} \\
      ax^3 & = bx    & \text{Multiply, on right, by x} \\
      a    & = bx    & \text{Subst other given} \\
      x    & = ab^{-1}
   \end{align*}

   \noindent 5. $x^2 = a^2$ and $x^5 = e$
   \begin{align*}
      x^2 & = a^2   & \text{Given} \\
      x^2 x^2 & = a^2 x^2 & \text{Multiply on right by $x^2$}\\
      x^4     & = a^4     & \text{Because $x^2 = a^2$.}\\
      x^5     & = a^4x \\
      e       & = a^4x \\
      x       & = (a^4)^{-1}
   \end{align*}

   \noindent 6. $(xax)^3 = bx$ and $x^2 a = (xa)^{-1}$.
   \begin{align}
      (xax)^3  & = bx \\
      xaxxaxxax & = bx \\
      x^2a      & = (xa)^{-1} \\
      xxaxa     & = e  & \text{from 3} \\
      xaxxa     & = e  & \text{from 3} \\
      e xxax    & = bx & \text{subst 5 into 2} \\
      exxaxa    & = bxa & \text{mult by a on right} \\
      ee        & = bxa & \text{subst 4 into 7} \\
      b^{-1}    & = xa \\
      b^{-1}a^{-1} & = x  \\
      x & = (ab)^{-1}
   \end{align}

   \item \textsc{Rules of Algebra in Groups}
   For each of the following rules, either prove that it is true in 
   every group $G$, or give a counterexample to show that it is false in 
   some groups. (All the counterexamples you need may be found in the group
   of matrices $\{I,A,B,C,D,K\}$ described on page 28.)

   The matrices from page 28 and the matrix multiplication table
   are recreated in figure~\ref{tab:matrices} and table~\ref{tab:matmult}
   respectively.

   \begin{figure}
   \caption{The matrices of $G$.}
   \label{tab:matrices}
   \begin{align*}
     I & = \left( \begin{array}{rr} 
              1 & 0 \\ 
	      0 & 1
	   \end{array} \right) & 
     A & = \left( \begin{array}{rr}
              0 & 1 \\
	      1 & 0
	   \end{array} \right)  &
     B & = \left( \begin{array}{rr}
              0 & 1 \\
	      -1 & -1
	   \end{array}  \right) \\ \\
     C & = \left( \begin{array}{rr}
              -1 & -1 \\
	       0 &  1
	   \end{array} \right) &
     D & = \left( \begin{array}{rr}
              -1 & -1 \\
	       1 &  0
	   \end{array} \right) &
     K & = \left( \begin{array}{rr}
               1 &  0 \\
	      -1 & -1
	   \end{array} \right)
   \end{align*}
   \end{figure}

   \begin{table}
   \caption{Matrix multiplication of the matrices in $G$}
   \label{tab:matmult}
   \begin{tabular}{c|cccccc}
     $*$ & $I$ & $A$ & $B$ & $C$ & $D$ & $K$ \\ \hline
     $I$ & $I$ & $A$ & $B$ & $C$ & $D$ & $K$ \\
     $A$ & $A$ & $I$ & $C$ & $B$ & $K$ & $D$ \\
     $B$ & $B$ & $K$ & $D$ & $A$ & $I$ & $C$ \\
     $C$ & $C$ & $D$ & $K$ & $I$ & $A$ & $B$ \\
     $D$ & $D$ & $C$ & $I$ & $K$ & $B$ & $A$ \\
     $K$ & $K$ & $B$ & $A$ & $D$ & $C$ & $I$
   \end{tabular}
   \end{table}
   
   \noindent 1. If $x^2=e$, then $x=e$.

   \noindent \Solution This rule is false. A counter example is
   $A*A=I$ yet $A\ne I$.

   \noindent 2. If $x^2=a^2$ then $x=a$.

   \noindent \Solution This rule is false. A counter example is $A*A=I=I*I$
   yet $A \ne I$.

   \noindent 3. $(ab)^2 = a^2b^2$.

   \noindent \Solution This rule is false. A counter example is given
   by letting $a=A$ and $b=B$. $(AB)^2=C^2=I \ne D = ID = A^2B^2$.

   \noindent 4. If $x^2=x$, then $x=e$.

   \noindent \Solution This rule is true.
   \begin{proof}
      \begin{align*}
         x^2 & = x \\
	 xx  & = x \\
	 xxx^{-1} & = xx^{-1} \\
	 xe       &= e \\
	 x        & = e \qedhere
      \end{align*}
   \end{proof}

   \noindent 5. For every $x \in G$, there is some $y \in G$ such that
   $x = y^2$. (This is the same as saying that every element of $G$ has
   a ``square root''.)

   \noindent \Solution This rule is false. In the group G in
   figure~\ref{tab:matrices}, it is clear that the matrices 
   $A$, $C$, and $K$ do not have a ``square root''.

   \noindent 6. For any two elements $x$ and $y$ in $G$, there is an element
   $z$ in $G$ such that $y=xz$.

   \noindent \Solution. This rule is true.
   \begin{proof}
   We want to make the equality $y=xz$ hold. Since $G$ is a group we know
   that $x$ has an inverse. So we can write $z=x^{-1}y$. Also since
   $G$ is a group we know that $x^{-1}y$ is defined and unique. Therefore,
   $z$ exists.
   \end{proof}

   \item \textsc{Elements That Commute}

   \noindent If $a$ and $b$ are in $G$ and $ab=ba$, we say that $a$ and $b$ 
   \emph{commute}. Assming that $a$ and $b$ commute, prove the following:

   \noindent 1. $a^{-1}$ and $b^{-1}$ commute.
   \begin{proof}
      \begin{align*}
         a^{-1}b^{-1} & = (ba)^{-1} && \text{Theorem 3} \\
	              & = (ab)^{-1} && \text{Given $a,b$ commute} \\
		      & = b^{-1}a^{-1} && \text{Theorem 3} \qedhere
      \end{align*}
   \end{proof}

   \noindent 2. $a$ and $b^{-1}$ commute. (\textsc{Hint}: First show
   that $a=b^{-1}ab$.
   \begin{proof}
   Since $ab=ba$, it is true that $ba=ab$ and $b^{-1}ba=b^{-1}ab$. This
   can be reduced and we have that $a=b^{-1}ab$. Finally, we can
   multiply by $b^{-1}$ on both sides to get $ab^{-1}=b^{-1}a$.
   \end{proof}

   \noindent 3. $a$ commutes with $ab$.
   \begin{proof}
   We know that $a(ab) = a(ba)$ by commutativity, and $a(ba)=(ab)a$ by
   associativity. So we have $a(ab)=(ab)a$.
   \end{proof}

   \noindent 4. $a^2$ commutes with $b^2$.
   \begin{proof}
      \begin{align*}
         a^2b^2 & = (aa)(bb) && \text{Definition}\\
	        & = a(a(bb)) && \text{Associativity}\\
		& = a((ab)b) && \text{Associativity}\\
		& = a((ba)b) && \text{Commutativity Given} \\
		& = a(b(ab)) && \text{Associativity} \\
		& = a(b(ba)) && \text{Commutativity Given} \\
		& = (ab)(ba) && \text{Associativity} \\
		& = (ba)(ba) && \text{Commutativity} \\
		& = b(a(ba)) && \text{Associativity} \\
		& = b((ab)a) && \text{Associativity} \\
		& = b((ba)a) && \text{Commutativity} \\
		& = b(b(aa)) && \text{Associativity} \\
		& = (bb)(aa) && \text{Associativity} \\
		& = b^2a^2   && \text{Definition} \qedhere
      \end{align*}
   \end{proof}

   \noindent 5. $xax^{-1}$ commutes with $xbx^{-1}$, for any $x\in G$.
   \begin{proof}
   \begin{align*}
      xax^{-1}xbx^{-1} & = xabx^{-1} \\
                       & = xbax^{-1} && \text{Commutativity} \\
		       & = x(be)ax^{-1} && \text{Identity} \\
		       & = x(b(x^{-1}x)ax^{-1} && \text{Inverses}\\
		       & = xbx^{-1}xax^{-1} && \text{Associativity}\qedhere
   \end{align*}
   \end{proof}

   \noindent 6. $ab=ba$ iff $aba^{-1}=b$.
   \begin{proof}
   First we will assume $ab=ba$. Given this we can derive
   $abe=aba^{-1}a=ba$ and this can be reduced to $aba^{-1}=b$.
   It is trivial to reverse: $aba^{-1}=b$, therefore $aba^{-1}a=ba$,
   therefore $ab=ba$.
   \end{proof}
   
   \noindent 7. $ab=ba$ iff $aba^{-1}b^{-1}=e$.
   \begin{proof}
   Assume $ab=ba$. Then we know (by 6) that $aba^{-1}=b$. Therefore,
   $aba^{-1}b^{-1}=e$. The reverse direction is trivial.
   \end{proof}

   \item \textsc{Group Elements and Their Inverses}

   \noindent Let $G$ be a group. Let $a$, $b$, $c$ denote elements of
   $G$, and let $e$ be the neutral element of $G$.

   \noindent 1. Prove that if $ab=e$, then $ba=e$.
   \begin{proof}
      \begin{align*}
         ab & = e && \text{Given} \\
	 bab & = be && \\
	 bab & = a^{-1}e && \text{Theorem 2}\\
	 ba  & = a^{-1}eb^{-1} \\
	 ba  & = a^{-1}b^{-1} \\
	 ba  & = a^{-1}a && \text{Theorem 2} \\
	 ba  & = e \qedhere
      \end{align*}
   \end{proof}

   \noindent 2. Prove that if $abc=e$, then $cab=e$ and $bca=e$.
   \begin{proof}
   We are given that $abc=e$. By associativity we have $(ab)c=e$.
   By D1 we have that $c(ab)=cab=e$. We can do this one more time:
   $(ca)b=e$ and therefore $b(ca)=bca=e$.
   \end{proof}

   \noindent 3. State a generalization of 1 and 2
   
   \noindent \Solution If $a_1 a_2 \cdots a_{n-2} a_{n-1} a_n=e$ then 
   $a_n a_1 a_2 \cdots a_{n-2} a_{n-1}=e$. We have proven it for
   $n=2,3$. We can group the first
   $n-1$ elements in parentheses. We then let $a$ equal these terms
   and $b=a_n$ and then apply D1 to get the desired
   result.

   \noindent Prove the following:

   \noindent 4. If $xay=a^{-1}$, then $yax=a^{-1}$.
   \begin{proof}
      \begin{align*}
         xay  & = a^{-1} \\
	 xaya & = e \\
	 axay & = e && \text{D3} \\
	 yaxa & = e && \text{D3} \\
	 yax  & = a^{-1}
      \end{align*}
   \end{proof}

   \noindent 5. Let $a$, $b$, $c$ each be equal to its own inverse.
   If $ab=c$, then $bc=a$ and $ca=b$.
   \begin{proof}
      \begin{align*}
         ab & = c    && \text{Given} \\
	 abc^{-1} & = e \\
	 abc      & = e && \text{Given $c=c^{-1}$} \\
	 cab      & = e && \text{D1}\\
	 ca       & = b^{-1} \\ 
	 ca       & = b  && \text{Given $b=b^{-1}$}\\
	 bc       & = a^{-1} && \text{From third line}\\
	 bc       & = a      && \text{Given $a=a^{-1}$} \qedhere
      \end{align*}
   \end{proof}

\end{enumerate}
\end{document}
