\documentclass[twoside]{amsart}
\usepackage{amssymb,latexsym}
\usepackage{xspace}
\usepackage{enumerate}
\usepackage{graphics}
\newcommand{\Rationals}{\mathbb{Q}{}}
\newcommand{\Reals}{\mathbb{R}{}}
\newcommand{\Integers}{\mathbb{Z}{}}
\newcommand{\Solution}{\textsc{Solution}\xspace}
\newcommand{\Problem}{\textsc{Problem}\xspace}
\newcommand{\Blank}{\mathrel{\phantom{=}}}
\newcommand{\ltrue}{\top}
\newcommand{\lfalse}{\bot}
\begin{document}
\title{Answers to Chapter 3 Exercises - A Book of Abstract Algebra}
\author{Michael Welch}
\date{\today}
\maketitle

This document contains selected answers to exercises from chapter 4
of A Book of Abstract Algebra.

Remark on notation. In the exercises in this document, the exponential
notation $a^n$ is used in the following sense: if $a$ is any element of
a group $G$, then $a^2$ means $aa$, and $a^3$ means $aaa$, and, in
general, $a^n$ is the product of $n$ factors of $a$, for any positive
integer $n$.

\begin{enumerate}[A.]
   \item \textsc{Solving Equations in Groups}

   \noindent Let $a$, $b$, $c$, and $x$ be elements of a group $G$. In each of
   the following, solve for $x$ in terms of $a$, $b$, and $c$.

   \noindent   1. $axb = c$
      \begin{align*}
         axb & = c \\
	 a^{-1}axb & = a^{-1}c \\
	 xbb^{-1} & = a^{-1}cb^{-1} \\
	 x & = a^{-1}cb^{-1}
      \end{align*}

   \noindent   2. $x^2b = xa^{-1}c$
      \begin{align*}
         x^2b & = xa^{-1}c    \\
	 x^2bb^{-1} & = xa^{-1}cb^{-1} \\
	 x^2        & = xa^{-1}cb^{-1} \\
	 x^{-1}x^2  & = x^{-1}xa^{-1}cb^{-1}\\
	 x^{-1}xx   & = a^{-1}cb^{-1} \\
	 x          & = a^{-1}cb^{-1}
      \end{align*}

   Solve simultaneously:

   \noindent 3. $x^2a = bxc^{-1}$ and $acx = xac$
   \begin{align*}
      acx & = xac      & \text{Given} \\
      xacx & = x^2ac   & \text{Multiply by $x$} \\
      xacxc^{-1} & = x^2a \\
      xacxc^{-1} & = bxc^{-1} & \text{Substitute} \\
      xacx       & = bx  \\
      xac        & = b  \\
      x          & = bc^{-1}a^{-1}
   \end{align*}



\end{enumerate}
\end{document}
